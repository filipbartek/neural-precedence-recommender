% !TEX root = main.tex

% Terminology

\todo[inline,author=FB]{Get inspiration from PAAR paper.}

\subsection{Saturation-based theorem proving}
\label{sec:saturation}

A \emph{\acrfull{fol} problem} consists of a set of axiom formulas and a conjecture formula.
In a \emph{refutation-based} \emph{\acrfull{atp}},
proving that the axioms entail the conjecture
is reduced to proving that the axioms together with the negated conjecture entail a \emph{contradiction}.
The most popular \gls{fol} \glspl{atp}, such as Vampire \cite{10.1007/978-3-642-39799-8_1}, E \cite{10.1007/978-3-030-29436-6_29}, or SPASS \cite{DBLP:conf/cade/WeidenbachDFKSW09},
start the proof search by converting the input \gls{fol} formulas to an equisatisfiable representation in 
% \emph{quantifier-free first-order 
\emph{\acrfull{cnf}} \cite{DBLP:books/el/RV01/NonnengartW01,Harrison2009}.
We denote a problem in \gls{cnf} as $P = (\Sigma, \mathit{Cl})$,
where $\Sigma$ is a list of all non-logical (predicate and function)
\emph{symbols} in the problem called the \emph{signature},
and $\mathit{Cl}$ is the set of the clauses of the problem (including the negated conjecture).

Given a problem $P$ in \gls{cnf},
a \emph{saturation-based} \gls{atp} searches for a refutational proof
by iteratively applying the \emph{inference rules} from the given \emph{calculus}
to infer new clauses entailed by $\mathit{Cl}$.
As soon as the empty clause, denoted by $\square$, is inferred,
the prover concludes that the premises entail the conjecture,
the sequence of inferences taken constituting a proof.
In case the premises do not entail the conjecture,
the proof search continues until
the set of the inferred clauses is saturated with respect to the inference rules.
In the common setting of time-restricted proof search, a time limit may end the process prematurely.

The common saturation-based \glspl{atp} structure the proof search
by maintaining two classes of the inferred clauses: processed and unprocessed \cite{10.1007/978-3-030-29436-6_29}.
In each \emph{iteration of the saturation loop}, one clause (so-called \emph{given clause})
is selected from the unprocessed set
and all the inferences between the given clause and the processed clauses are performed.
The resulting new clauses are added to the unprocessed set.

\todo[inline,author=FB]{Add terminology: $\bot$.}

\subsection{Superposition calculus}

The superposition calculus is of special interest
because it is used in the most successful \gls{fol} \glspl{atp}.

The inference rules of superposition calculus are constrained by a \gls{sot}.

\todo[inline,author=FB]{Mention: Vampire, superposition calculus, inference rules (resolution, superposition), simplification ordering on terms, KBO, symbol precedence. How does the STO influence the rules? Mention transfinite KBO (lcm predicate), which is only relevant to literal selection. Cite Voronkov: transfinite KBO (see section Notes on implementation). We use TKBO to increase the impact of predicate precedence. Vampire sets all KBO weights to 1.}

\subsubsection{\Gls{sot}}

An ordering on terms is called a \emph{simplification ordering} if it satisfies the following conditions:

...

influences the proof search in Vampire on two levels.
First, the inferences on each clause are limited
to the selected literals.
\todo{Cite Bachmair Ganzinger or Handbook of AR, chapter 2. For literal selection. Inspiration: Selecting the selection.}
In each clause,
either a negative literal or all the maximal literals are selected.
The maximality is evaluated
according to the simplification ordering.
% Note: The selection function Total does not use the simplification ordering.
Second, the simplification ordering orients some of the equalities
to prevent superposition and equality factoring
from inferring redundant complex conclusions.
In each of these two roles,
the simplification ordering may impact the direction and,
in effect, the length of the proof search.

\subsubsection{\Acrfull{kbo}}

% Example: It seems that a classical example for KBO an orienting equations is group theory axioms.




Recall the two views:
\begin{enumerate}
\item
	One is the view of complete literal selections (we influence which literals are maximal)

\item
	The other is that of the term rewriting world

\end{enumerate}
(The full landscape also contains LPO).




\subsubsection{Symbol precedences}

\todo[inline,author=FB]{Define terminology.}

\subsection{Neural networks}

A \emph{feed-forward neural network} is a directed acyclic graph
of differentiable \emph{modules} that operate on numeric vectors.
The common modules include multiplication by a matrix, and nonlinearity such as ReLU or sigmoid.\footnote{
These are, respectively, $f(x) = \max\{0,x\}$ and $g(x) = 1/(1+e^{-x}$). }
Some of the modules are parameterized by trainable parameters or hyperparameters.
In a typical usage, the neural network is trained by \emph{gradient descent}.
In this setting, the trained network outputs a single numeric value called \emph{loss}.
Since the network is differentiable,
a gradient of the loss is computed with respect to all the trainable parameters of the network.
The gradients are typically computed by \emph{backpropagation}.
The values of the parameters are then updated by taking a small step in the direction that minimizes the loss.

\todo{Maybe we don't need to make GCN a preliminary; we can define it later instead.}
\todo[inline,author=FB]{GCN? Backprop?, gradient descent, loss, gradient, epoch}
