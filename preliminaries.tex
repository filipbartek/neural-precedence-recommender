% Terminology

\todo[inline,author=FB]{Get inspiration from PAAR paper.}

\subsection{Saturation-based theorem proving}

A \emph{\acrfull{fol} problem} consists of a set of premise formulas and a conjecture formula.
In a \emph{refutation-based} \emph{\acrfull{atp}},
proving that the premises entail the conjecture
is reduced to proving that the premises together with the negated conjecture entail a \emph{contradiction}.
The most popular \gls{fol} \glspl{atp}, such as Vampire \cite{10.1007/978-3-642-39799-8_1}, E \cite{10.1007/978-3-030-29436-6_29}, or SPASS \cite{},
start the proof search by converting the input \gls{fol} formulas to an equisatisfiable representation in \emph{quantifier-free \gls{cnf}}.\cite{Harrison2009}
We denote a problem in \gls{cnf} as $P = (\Sigma, \mathrm{Cl})$,
where $\Sigma$ is a list of all non-logical symbols in the problem,
and $\mathrm{Cl}$ is the set of the clauses of the problem (including the negated conjecture).

Given a problem $P$ in \gls{cnf},
a \emph{saturation-based} \gls{atp} searches for a refutational proof
by iteratively applying the \emph{inference rules} from the given \emph{calculus}
to infer new clauses entailed by $\mathrm{Cl}$.
As soon as the empty clause, denoted by $\square$, is inferred,
the prover concludes that the premises entail the conjecture,
the sequence of inferences taken constituting a proof.
In case the premises do not entail the conjecture,
the proof search continues until
the set of the inferred clauses is saturated with respect to the inference rules.
In the common setting of time-restricted proof search, a timeout may end the process prematurely.

\todo[inline,author=FB]{Add terminology: saturation iteration, $\bot$.}

%\vampire{} is a saturation-based prover.
%Given a \gls{fol} problem,
%\vampire{} converts the problem into \gls{cnf}.
%and proceeds to saturate the set of clauses
%by repeated application of inference rules.

\subsection{Superposition calculus}

\todo[inline,author=FB]{Mention: Vampire, superposition calculus, inference rules (resolution, superposition), simplification ordering on terms, KBO, symbol precedence. How does the STO influence the rules? Mention transfinite KBO (lcm predicate), which is only relevant to literal selection. Cite Voronkov: transfinite KBO (see section Notes on implementation). We use TKBO to increase the impact of predicate precedence. Vampire sets all KBO weights to 1.}

%\vampire{} searches for a proof by applying the rules
%of the superposition inference system
%\cite{10.1007/978-3-642-39799-8_1}.

\subsubsection{\Gls{sot}}

% AWR candidate
Simplification ordering on terms
influences the proof search in Vampire on two levels.
First, the inferences on each clause are limited
to the selected literals.
\todo{Cite Bachmair Ganzinger or Handbook of AR, chapter 2. For literal selection. Inspiration: Selecting the selection.}
In each clause,
either a negative literal or all the maximal literals are selected.
The maximality is evaluated
according to the simplification ordering.
% Note: The selection function Total does not use the simplification ordering.
Second, the simplification ordering orients some of the equalities
to prevent superposition and equality factoring
from inferring redundant complex conclusions.
In each of these two roles,
the simplification ordering may impact the direction and,
in effect, the length of the proof search.

\subsubsection{\Acrfull{kbo}}

% Example: It seems that a classical example for KBO an orienting equations is group theory axioms.

\subsubsection{Symbol precedences}

\todo[inline,author=FB]{Define terminology.}

\subsection{\Acrlongpl{gcn}}

\todo[inline,author=FB]{GCN? Backprop?, gradient descent, loss, gradient, epoch}
