\documentclass[runningheads]{llncs}
\usepackage[utf8]{inputenc}
\usepackage[T1]{fontenc}
\usepackage[english]{babel}
\usepackage[binary-units=true,detect-weight=true]{siunitx}

% How to write a paper:
% https://mj.ucw.cz/papers/jakpsat.pdf
% Jones 2016: https://www.microsoft.com/en-us/research/academic-program/write-great-research-paper/
% AWR: https://jazyky.fel.cvut.cz/vyuka/RPP/BE9M04AKP/

% PAAR paper: https://github.com/filipbartek/learning-precedences-from-elementary-symbol-features/releases/download/paar2020%2Fceur-2/Learning_Precedences_from_Simple_Symbol_Features.pdf

\usepackage{amsfonts}
\usepackage{amsmath}
\usepackage{amssymb}
\usepackage{todonotes}

%\usepackage[numbers]{natbib}
%\bibliographystyle{plainnat}

% Multi-letter identifier
\newcommand{\mli}[1]{\mathit{#1}}

\DeclareMathOperator{\re}{\mathbb{R}}
\DeclareMathOperator{\nat}{\mathbb{N}}
\DeclareMathOperator{\logit}{logit}
\DeclareMathOperator{\sigmoid}{sigmoid}
\DeclareMathOperator*{\argmin}{argmin}
\DeclareMathOperator{\argsort}{argsort}
\newcommand{\Perm}[1]{\mathrm{Perm}(#1)}
\newcommand{\inv}[1]{#1^{-1}}
\newcommand{\card}[1]{\left|#1\right|}
\newcommand{\SquareBracket}[1]{\left[#1\right]}
\newcommand{\Parentheses}[1]{\left(#1\right)}
\newcommand{\DotProd}[2]{\left<#1,#2\right>}
\newcommand{\Better}[2]{#1 \prec #2}
\newcommand{\Prob}[1]{\mathrm{Prob}(#1)}

\DeclareMathOperator{\symbols}{\Sigma}
\newcommand{\Problems}[1]{\mathcal{P}_{#1}}
\DeclareMathOperator{\cnf}{\Problems{\mathrm{CNF}}}
\DeclareMathOperator{\ProblemsTptp}{\Problems{0}}
\DeclareMathOperator{\ProblemsTrain}{\Problems{\mathrm{train}}}
\DeclareMathOperator{\ProblemsTrainEx}{\ProblemsTrain'}
\DeclareMathOperator{\ProblemsVal}{\Problems{\mathrm{val}}}
\DeclareMathOperator{\ProblemsValEx}{\ProblemsVal'}
\newcommand{\signature}[1]{\Sigma_#1}
\newcommand{\Precedences}[1]{\Perm{\card{\signature{P}}}}
\newcommand{\fv}{\mli{fv}}

\DeclareMathOperator{\Cost}{\mli{c}}
\DeclareMathOperator{\CostSym}{\mli{\Cost_{sym}}}
\DeclareMathOperator{\CostPrec}{\mli{\Cost_{prec}}}
\DeclareMathOperator{\CostPrecPair}{\mli{LogOdds_{pair}}}

\newcommand{\definiendum}[1]{#1}
\newcommand{\vampire}{Vampire}
\newcommand{\frequency}{\texttt{frequency}}
\newcommand{\atping}{automated theorem proving}

\usepackage{glossaries}

\newacronym{ann}{ANN}{artificial \acrlong{nn}}
\newacronym{atp}{ATP}{automated theorem prover}
\newacronym{casc}{CASC}{CADE ATP System Competition}
% http://www.tptp.org/CASC/
\newacronym{ciirc}{CIIRC}{Czech Institute of Informatics, Robotics and Cybernetics}
\newacronym{cnf}{CNF}{clause normal form}
\newacronym{csv}{CSV}{comma-separated values}
\newacronym{ctu}{CTU}{Czech Technical University in Prague}
\newacronym{dag}{dag}{directed acyclic graph}
\newacronym{fel}{FEL}{Faculty of Electrical Engineering}
\newacronym{fof}{FOF}{first-order form}
% http://www.tptp.org/TPTP/TR/TPTPTR.shtml
\newacronym{fol}{FOL}{first-order logic}
\newacronym{gcn}{GCN}{graph convolutional network}
\newacronym{gnn}{GNN}{graph \acrlong{nn}}
\newacronym{hin}{HIN}{heterogeneous information network}
\newacronym{json}{JSON}{JavaScript Object Notation}
\newacronym{kbo}{KBO}{Knuth-Bendix ordering}
\newacronym{lpo}{LPO}{lexicographic path ordering}
\newacronym{ml}{ML}{machine learning}
\newacronym{nn}{NN}{neural network}
\newacronym{relu}{ReLU}{Rectified Linear Unit}
\newacronym{rgcn}{R-GCN}{relational \acrlong{gcn}}
\newacronym{tkbo}{TKBO}{transfinite \acrlong{kbo}}
\newacronym{tptp}{TPTP}{Thousands of Problems for Theorem Provers}
% http://www.tptp.org/
\newacronym{ucb}{UCB}{Upper Confidence Bound}
\newacronym{ucb1}{UCB1}{\gls{ucb}1}

\newglossaryentry{sot}{
name={simplification ordering on terms},
description={},
plural={simplification orderings on terms}
}

% !TEX root = main.tex

\title{Neural Precedence Recommender}

% > The corresponding author, i.e., the author responsible for checking the final proof and for signing the copyright form on behalf of all of the authors, should be clearly marked in the header of the paper.
% > The inclusion of the corresponding author’s email address is  mandatory.

% > We strongly recommend that all authors include their email addresses in their papers.

\author{
Filip Bártek\inst{1,2}\orcidID{0000-0002-1822-2651} \and
Martin Suda\inst{1}\orcidID{0000-0003-0989-5800}
}

\authorrunning{F. Bártek \and M. Suda}

% > The affiliated institutions, including town/city and country
\institute{
\acrlong{ciirc}\\
\acrlong{ctu}\\
Jugoslávských partyzánů 1580/3, 160 00 Praha 6 -- Dejvice, Czech Republic\\
\email{\{filip.bartek,martin.suda\}@cvut.cz}\\
% TODO: Consider removing Martin's email.
% TODO: Consider changing Martin's email to martin.suda@gmail.com.
%\url{https://www.ciirc.cvut.cz/}
\and
\acrlong{fel}\\
\acrlong{ctu}\\
Technická 2, 166 27 Praha 6 -- Dejvice, Czech Republic\\
%\url{http://www.fel.cvut.cz/}
}


\usepackage{svg}

\usepackage[pdf]{graphviz}
\usepackage{tikz}

\usepackage{hyperref}
\hypersetup{
pdftitle={Neural Precedence Recommender},
pdfauthor={Filip Bártek, Martin Suda},
pdfkeywords={saturation-based theorem proving, simplification ordering, symbol precedence, machine learning, graph convolutional network}
}

\usepackage{cleveref}

\begin{document}

\maketitle

\begin{abstract}
% AWR: Guidelines (Unit 7, page 5):
% At most 300 words
% Content:
% Purpose of the study
% Research problems
% Basic design of the study
% Summary of interpretations and conclusions

% AWR: Parts of an abstract (checklist) (Unit 7, page 6):
% Motivation
% Problem statement
% Approach
% Results
% Conclusions

This paper describes and evaluates a symbol precedence heuristic based on a \acrlong{gcn}.
When trained on proof attempts with random symbol precedences,
the heuristic outperforms the state of the art by more than 20~\%.
For the purpose of the evaluation, the symbol precedences predicted by the heuristic
are used to instantiate Knuth-Bendix term simplification orderings.
These orderings in turn guide the proof search in the state-of-the-art \acrlong{atp} \vampire{}
by constraining the superposition calculus inferences.
The evaluation is performed on \acrlong{fol} problems from the TPTP problem library.

% Motivation and purpose of the study are excluded because the audience at CADE need not be reminded of the motivation for making the provers faster.

\end{abstract}

\section{Introduction}

% General motivation for FOL ATPing:
% Vampire is used as a sledgehammer in Isabelle/HOL:
% 1. https://people.mpi-inf.mpg.de/~jblanche/life.pdf
% 2. https://www.cl.cam.ac.uk/~lp15/papers/Automation/paar.pdf
% Formal methods for verification (survey): https://arxiv.org/pdf/1912.03028.pdf

\todo[author=Filip,inline]{Consider an alternative approach: Start with an example problem, rewrite in two ways, describe how do state-of-the-art provers differentiate between these two ways (by symbol precedence), ...}

% Jones 2016 guidelines:
% 1. Problem (explain by example)
% 2. Contributions (refutable, forward references to sections)

% AWR structure (Unit 7, page 1):
% 1. Attention-getter (lead-in) [1-2 sentences]
% 2. Set up for the thesis [minimum: 2-3 sentences]
% 3. Thesis statement (essay map) [1 sentence]

% AWR guidelines (Unit 7, page 5):
% Whet the reader's appetite
% Set the context
% State why the main idea is important
% State your thesis/claim

\todo[inline,author=FB]{Try to use the first page for something more flashy than exposure of saturation-based automated proving. Ideas: catchy puzzle-like example, abstract description of the crucial challenge etc.}

Choosing the next inference to apply is the most important decision point in the state-of-the-art
\todo{FB: Add "saturation-based"?}
\glspl{atp} for \gls{fol}.
Since the most popular \glspl{atp} use the superposition calculus with literal selection,
the \gls{sot}, such as the \gls{kbo}, plays a key role in guiding the proof search.
The \gls{kbo} is typically specified by a symbol precedence,
that is a permutation of the non-logical symbols of the input problem.
Each of the commonly used precedence heuristics orders the symbols
lexicographically according to a small number of simple symbol properties,
such as the number of occurrences in the input problem,
% vampire --symbol_precedence frequency
the arity of the symbol, or
% vampire --symbol_precedence arity
%occurrence in the conjecture,
% vampire --symbol_precedence_boost goal
%number of occurrences in unit clauses, or
% vampire --symbol_precedence_boost units
whether or not the symbol was introduced by the Tseytin transformation or skolemization.
% vampire --introduced_symbol_precedence top
Since none of the simple heuristics fits all the problems,
choosing a symbol precedence that guides the proof search on a problem efficiently towards the refutation
remains to be an intriguing task.
This paper explores the possibility of solving this task by training a neural networks
on the results of running the \gls{atp} Vampire \cite{10.1007/978-3-642-39799-8_1} with randomly sampled precedences.

A \gls{fol} problem consists of \gls{fol} formulas divided into two categories:
one or more premises and one conjecture.
% We assume that \gls{fol} is expanded to FOL.
% For this reason the article is "an".
When presented with a \gls{fol} problem,
a \definiendum{saturation-based} \gls{atp} negates the conjecture (as a first step of proof by contradiction)
and converts all the formulas into an equisatisfiable set of clauses in \gls{cnf}.
\todo[author=Filip]{Are there any saturation-based ATPs that don't use CNF?}
The prover proceeds to iteratively infer new provable clauses
until a contradiction, represented by the empty clause, is inferred,
or until no new clauses can be inferred.
A proof of the fact that the premises and the negated conjecture entail a contradiction
constitutes a proof that the premises entail the conjecture.
\todo[author=Filip]{Cite some source that explains refutation-based automated theorem proving. Perhaps Harris?}
\todo[author=Martin]{prvni odstavec uvodu bych pro CADE publikum vynechal a zacal rovnou s druhym}

The state-of-the-art \gls{fol} \glspl{atp},
such as \vampire{} \cite{10.1007/978-3-642-39799-8_1} and E \cite{10.1007/978-3-030-29436-6_29},
are saturation-based and
infer new provable clauses by applying the rules of the \definiendum{superposition calculus}.
\todo[author=Filip]{Cite some source that explains superposition calculus.}
The inferences under superposition calculus are constrained by a \definiendum{simplification ordering on the terms}.
\todo[author=Filip]{Do we mean terms in sensu stricto (applications of functions), or generalized terms (applications of functions and predicates)?}
The two classes of simplification orderings commonly used in practice are
\gls{lpo} \cite{Kamin1980} and \gls{kbo} \cite{Knuth1983}.
In both classes the ordering is specified by a \definiendum{symbol precedence},
\todo[author=Filip]{This may be an oversimplification: KBO is further parameterized by symbol weights.}
which is a permutation of predicate and function symbols of the input problem.

The typical approach to constructing a symbol precedence orders the symbols by a property
that is relatively cheap and straightforward to compute,
such as the frequency (the number of occurrences in the input problem) \cite{E-manual} or the arity.
\todo[author=Filip]{Cite the original source of the arity heuristic.}
The structure of the problem contains much information that is relevant for the choice of symbol precedence,
but is completely ignored by the symbol properties that inform the symbol precedences.
\todo[author=Filip]{Why do we believe this?}
\todo[author=Martin]{Klicovou vetu ... by asi chtelo rozvest treba do celeho odstavce. Takhle vyzni dost krypticky a neni moc jasne, co se snazi vyjadrit: "The structure of the problem" neni predem vysvetleno. "completely ignored" - "ignored" trosku zavani nejakym agentem, pritom je zde predmetem neziva vec / abstraktum}

This paper describes the architecture (\cref{sec:architecture}) and an empirical evaluation (\cref{sec:evaluation})
of a symbol precedence heuristic based on a \gls{gcn} \cite{kipf2017semisupervised}
and trained on proof attempts with random symbol precedences.
The experiments presented in \cref{sec:evaluation} demonstrate an improvement of success rate by 20 \%
compared to the state-of-the-art heuristic.
\todo[author=Martin]{Poslednimyu ostaveci by taky asi jeste mohlo neco predhazet, aby to mohl shrnout: "the architecture" ma urcity clen, ale predtim zadna "architecture" jeste neni. "graph convolution network" taky vlastne zminujes poprve}

% Filip: I decided not to follow the AWR structure in favor of Martin and CADE.

\iffalse % AWR draft

The performance of the state-of-the-art \acrfullpl{atp} is severely limited by some of the design decisions that constitute the provers' architecture.
For example, the \frequency{} symbol precedence heuristic of the \acrfull{fol} \acrshort{atp} \vampire{} is easily outperformed by the best of 10 random precedences.\cite{}
The gap in performance suggests that a more elaborate precedence heuristic with a superior performance exists.
% what is precedence heuristic?
% Thesis statement
Training a model from proof attempts with random precedences yields a heuristic that significantly outperforms the state-of-the-art heuristic.

\fi


\section{Preliminaries}

% !TEX root = main.tex

\subsection{Saturation-based theorem proving}
\label{sec:saturation}

A \emph{\acrfull{fol} problem} consists of a set of axiom formulas and a conjecture formula.
\todo{FB: Clarify that we use FOL with equality.}
In a \emph{refutation-based} \emph{\acrfull{atp}},
proving that the axioms entail the conjecture
is reduced to proving that the axioms together with the negated conjecture entail a \emph{contradiction}.
The most popular \gls{fol} \glspl{atp}, such as \Vampire{} \cite{DBLP:conf/cav/KovacsV13}, E \cite{Schulz2019}, or SPASS \cite{DBLP:conf/cade/WeidenbachDFKSW09},
start the proof search by converting the input \gls{fol} formulas to an equisatisfiable representation in 
% \emph{quantifier-free first-order 
\emph{\acrfull{cnf}} \cite{DBLP:books/el/RV01/NonnengartW01,Harrison2009}.
We denote the problem in \gls{cnf} as $P = (\Sigma, \mathit{Cl})$,
% Handbook of AR, Chapter 3: signature $\Sigma = (P_\Sigma, F_\Sigma)$
\todo{FB: What about the variables? Handbook of AR (chapter 3) defines the set Var.}
where $\Sigma$ is a list of all non-logical (predicate and function)
\emph{symbols} in the problem called the \emph{signature},
and $\mathit{Cl}$ is the set of clauses of the problem (including the negated conjecture).

Given a problem $P$ in \gls{cnf},
a \emph{saturation-based} \gls{atp} searches for a refutational proof
by iteratively applying the \emph{inference rules} from the given \emph{calculus}
to infer new clauses entailed by $\mathit{Cl}$.
As soon as the empty clause, denoted by $\square$, is inferred,
the prover concludes that the premises entail the conjecture.
The sequence of those inferences leading up from the input clauses $\mathit{Cl}$ to the discovered $\square$ constitutes a proof.
If the premises do not entail the conjecture,
the proof search continues until
the set of inferred clauses is saturated with respect to the inference rules.
In the standard setting of time-restricted proof search, a time limit may end the process prematurely.

Since the space of derivable clauses is typically very large,
the efficacy of the prover depends on the order in which the inferences are applied.
The standard saturation-based \glspl{atp} order the inferences
by maintaining two classes of inferred clauses: processed and unprocessed \cite{Schulz2019}.
In each \emph{iteration of the saturation loop}, one clause (so-called \emph{given clause})
is selected from the unprocessed set
and all the inferences between the given clause and the processed clauses are performed.
The resulting new clauses are added to the unprocessed set.
Finishing the proof in few iterations of the saturation loop is important
because the number of inferred clauses typically grows exponentially during the proof search.

\todo{MS: Celkove hezky! Jen mi tu nejak chybi konec... A taky proc se to vsechno rika.
V PAAR clanku, pokud si dobre vzpominam, bylo hlavnim cilem zavest znaceni pro
 ``pocet iteraci nez se najde dukaz''. aby se pak pouzilo dale\dots
Nicmene ctu ted beginning->end, tak nevim, co chystas dale.\\
FB: Doplnil jsem motivaci do prvni vety odstavce.}

\todo[inline,author=FB]{Add terminology: $\bot$.}

\subsection{Superposition calculus}

The \emph{superposition calculus} is of particular interest
because it is used in the most successful contemporary \gls{fol} \glspl{atp}.
The inferences of the superposition calculus are constrained by a \emph{\gls{sot}} \cite{DBLP:journals/logcom/BachmairG94}.

The \gls{sot} influences the superposition calculus in two ways.
First, the inferences on each clause are limited
to the selected literals.
\todo{Cite Bachmair Ganzinger or Handbook of AR, chapter 2. For literal selection. Inspiration: Selecting the selection.}
\todo{MS: the original paper is \cite{DBLP:journals/logcom/BachmairG94}. 
Stoji za precteni, at se v tom trochu vyznas!
Mimochodem, mozna ze ``simplification ordering'' neni ten uplne nejpresnejsi nazev. 
Pripadne na ten clenek mrkni rychle uz ted. V tehle sekci by se hodily nejake citace,
aby bylo jasne, od koho terminy prejimame.
(Weidenbachova kapitola v Handbooku taku (Combining superposition, sorts and splitting)
taky tusim definuje presne vlastnosti toho usporadani a dava jim jmeno.}
In each clause,
either a negative literal or all maximal literals are selected.
The maximality is evaluated
according to the simplification ordering.
% Note: The selection function Total does not use the simplification ordering.
\todo{MS: Ta logika je (pro pripomenuti):
1) pravidla jsou sound i bez restrikci, (proto selection Total je v pohode)
2) nicmene se ukazuje, ze cim mene toho selectem a tedy cim mene bude provedeno inferenci, tim lepe typicky pro search (vetsi sance dokazat (i tezke) problemy)
3) na treti stranu, nemuzem nedelat nic - to by nebylo complete.
4) Pravidla o selekci literalu, ktera popisujes, vyjadruji
minimalni selekci, o ktere se podarilo dokazat (BachmairGanzinger), ze zaruci uplnost. 
}
Second, the simplification ordering orients some of the equalities
to prevent superposition and equality factoring
from inferring redundant complex conclusions.
In each of these two roles,
the simplification ordering may impact the direction and,
in effect, the length of the proof search.

The \emph{\acrfull{kbo}} \cite{Knuth1983}, a commonly used simplification ordering scheme,
is parameterized by symbol weights and a \emph{symbol precedence},
a permutation\footnote{
The definition of KBO does not require the precedence to be total. 
However, for use in ATPs, the more symbols and thus also terms 
we can compare, the better. % That is why we define a precedence as a permutation.
} of the non-logical symbols of the input problem.
In this work, we focus on the task of finding a symbol precedence which
leads to a good performance of an ATP
when plugged into the \gls{kbo},
leaving all the symbol weights at the default 
value 1 as set by the \gls{atp} \Vampire{}.

\subsection{\Acrlongpl{nn}}

A \emph{feedforward \acrlong{ann}} \cite{DBLP:books/daglib/0040158} is a \acrlong{dag} of \emph{modules}.
Each module is an operation that consumes a numeric (input) \emph{vector} and outputs a numeric vector.
Each of the components of the output vector is called a \emph{unit} of the module.
The output of each module is differentiable with respect to the input almost everywhere.
\todo{FB: Should we explain why "almost everywhere" suffices?}

The standard modules include
the \emph{fully connected layer}, which performs an affine transformation,
and non-linear \emph{activation functions} such as the \emph{\gls{relu}} or \emph{sigmoid}.\footnote{
These are, respectively, $f(x) = \max\{0,x\}$ and $g(x) = \frac{1}{1+e^{-x}}$.}
A fully connected layer with a single unit is called the \emph{linear unit}.

Some of the modules are parameterized by numeric \emph{parameters}.
For example, the fully connected layer that transforms the input $x$ by the affine transformation $Wx + b$
is parameterized by the weight matrix $W$ and the bias vector $b$.
If the output of a module is differentiable with respect to a parameter,
that parameter is considered \emph{trainable}.

In a typical scenario, the \acrlong{nn} is trained by \emph{gradient descent} on a \emph{training set} of \emph{examples}.
% Note: In case of GCN, the example is actually a set of edge matrices.
In such a setting, the network outputs a single numeric value called \emph{loss} when evaluated on a \emph{batch} of examples.
The loss of a batch is typically computed as a weighted sum of the losses of the individual examples.
Since each of the modules is differentiable with respect to its input and trainable parameters,
the gradient of the loss with respect to all trainable parameters of the neural network
can be computed using the \emph{back-propagation} algorithm \cite{DBLP:books/daglib/0040158}.
The trainable parameters are then updated by taking a small step
against the gradient---in the direction that is expected to reduce the loss.
An \emph{epoch} is a sequence of iterations that updates the trainable parameters
using each example in the training set exactly once.

A \emph{\gls{gcn}} is a special case of feedforward \acrlong{nn}.
The modules of a \gls{gcn} transform messages that are passed along the edges of a graph encoded in the input example.
A particular architecture of a \gls{gcn} used prominently in this work is discussed in \cref{sec:gcn}.


\section{Architecture}
\label{sec:architecture}

% !TEX root = main.tex

The precedence recommender is a system that takes
a \gls{cnf} problem $P = (\Sigma, \mathit{Cl})$ as the input,
and produces a precedence $\pi^*$ over the symbols $\Sigma$ as the output.
For the recommender to be useful, it needs to produce a precedence
that is likely to yield a proof in as few iterations of the saturation loop as possible.

The recommender described in this section
first uses a neural network to compute a cost value for each symbol of the input problem,
and then orders the symbols by their costs in a non-increasing order.
In this manner, the task of finding good precedences is reduced to the task
of training a good symbol cost function,
as discussed in \cref{sec:training}.
%The architecture is designed so that the recommender can be trained
%on the results of \gls{atp} executions on various problems with random precedences.
%Since the recommender contains a neural network,
%it is parameterized by weight tensors
%whose values can be trained by gradient descent.

The recommender consists of modules that perform specific subtasks,
each of which is described in detail in one of the following\iftoggle{LONG}{
sections (see also \cref{fig:architecture}):
\begin{itemize}
\item \Cref{sec:graphifier}: The graph constructor transforms the input \gls{cnf} problem into a problem graph.
\item \Cref{sec:gcn}: The \gls{gcn} computes symbol embeddings from the problem graph.
\item \Cref{sec:output}: The output layer extracts symbol costs from the symbol embeddings.
\item \Cref{sec:sorting}: The final symbol precedence is obtained by sorting the symbols by their costs.
\end{itemize}
}{
sections (see also \cref{fig:architecture}).
}

\begin{figure}[h]
\caption{Recommender architecture overview.
When recommending a precedence, the input is problem $P$ and the output is precedence $\pi^*$.
When training, the input is problem $P$ and precedences $\pi$ and $\rho$,
and the output is the loss value.
The trainable modules and the edges along which the loss gradient is propagated are emphasized by bold lines.}
\label{fig:architecture}
\centering
\digraph[scale=0.5]{ArchitectureOverview}{
graph [splines=ortho, ranksep=0.25];
node [shape=box, fontsize=14, width=0, height=0];

{ rank = same;
Problem [label="Problem P"];
pi [label=<Precedence &pi;>];
rho [label=<Precedence &rho;>];
}

Graphifier [style=rounded, label="Graph constructor"];
g [label="Graph"];
GCN [style=<rounded,bold>, label="GCN"];
SymbolEmbedding [label="Symbol embeddings"];
SymbolCostModel [style=<rounded,bold>, label="Output layer"];
SymbolCost [label="Symbol costs"];

{ rank = same;
Sort [style=rounded, label="Sort"];
LossFunction [style=rounded, label="Loss function"];
}

{ rank = same;
Precedence [label=<Precedence &pi;*>];
Loss [label="Loss value"];
}

Problem -> Graphifier -> g -> GCN;
GCN -> SymbolEmbedding -> SymbolCostModel -> SymbolCost [style=bold];
SymbolCost -> Sort;
Sort -> Precedence;

pi -> LossFunction:nw;
rho -> LossFunction:ne;
SymbolCost -> LossFunction [style=bold];
LossFunction -> Loss [style=bold];
}

\iffalse

\usetikzlibrary{shapes}
\tikzstyle{object} = [rectangle, draw]
\tikzstyle{input} = [ellipse, draw]
\tikzstyle{output} = [ellipse, draw]

\begin{tikzpicture}[node distance = 0.5 and 1, ->]
% https://tex.stackexchange.com/a/332796/202639

\node (problem) [input] {\gls{cnf} problem $P$};
\node (symbol embeddings) [object] [below=of problem] {Symbol embeddings};
\node (symbol costs) [object] [below=of symbol embeddings] {Symbol costs};
\node (symbol precedence) [output] [below=of symbol costs] {Symbol precedence};

\draw (problem) to node [left] {GCN} (symbol embeddings);
\draw (symbol embeddings) to node [left] {MLP} (symbol costs);
\draw (symbol costs) to node [left] {Order symbols by their costs} (symbol precedence);

\node (PrecBetter) [input] [right=of problem] {Precedence $\PrecBetter$};
\node (PrecBetterInv) [object] [below=of PrecBetter] {$\inv{\PrecBetter}$};
\draw (PrecBetter) to node [right] {Invert} (PrecBetterInv);

\node (PrecWorse) [input] [right=of PrecBetter] {Precedence $\PrecWorse$};
\node (PrecWorseInv) [object] [below=of PrecWorse] {$\inv{\PrecWorse}$};
\draw (PrecWorse) to node [right] {Invert} (PrecWorseInv);

\node (PrecDiff) [object] [right=2 of symbol costs] {$\inv{\PrecWorse} - \inv{\PrecBetter}$};
\node (PrecDiffNormalized) [object] [below=of PrecDiff] {Normalized};
\node (PrecPairCost) [object] [below=of PrecDiffNormalized] {Precedence pair cost};
\node (loss) [output] [below=of PrecPairCost] {Loss};

\draw (PrecBetterInv) to (PrecDiff);
\draw (PrecWorseInv) to (PrecDiff);
\draw (PrecDiff) to node [right] {Normalize} (PrecDiffNormalized);
\draw (PrecDiffNormalized) to (PrecPairCost);
\draw (PrecPairCost) to (loss);

\draw (symbol costs) to (PrecPairCost);

\end{tikzpicture}

\fi

\end{figure}
\todo{FB: Improve the diagram. Consider using tikz. See the disabled tikz figure in the source file.}
\todo[author=R2]{The small bold arrows are quite difficult to distinguish from the small non-bold arrows.}

\subsection{Graph constructor: From \gls{cnf} to graphs}
\label{sec:graphifier}

As the first step of the recommender processing pipeline,
the input problem is converted from a \gls{cnf} representation
to a \emph{heterogeneous (directed) graph} \cite{Zhou2018}.
% Zhou uses the term "heterogeneous graph".
%\footnote{Data mining literature often uses the term \gls{hin} \cite{Shi2015} for heterogeneous graphs.
%We prefer to conform to the terminology common in literature studying \glspl{gnn}.}
Each of the nodes of the graph is labeled with a node type,
and each edge is labeled with an edge type,
defining the heterogeneous nature of the graph.
Each node corresponds to one of the elements that constitute the \gls{cnf} formula,
such as a clause, an atom or a predicate symbol.
Each such category of elements corresponds to one node type.
The edges represent the (oriented) relations between the elements,
for example the incidence relation between a clause and one of its (literals') atoms,
or the relation between an atom and its predicate symbol.
$\mathcal{R}$ denotes the set of all relations in the graph.
\Cref{fig:CnfSchema} shows the types of nodes and edges used in our graph representation.
\Cref{fig:GcnExample} shows an example of a graph representation of a simple problem.

\newcommand{\ntype}[1]{\texttt{#1}}
\newcommand{\etype}[1]{\texttt{#1}}
\newcommand{\epos}{\etype{pos}}
\newcommand{\eneg}{\etype{neg}}

\begin{figure}[h]
\caption{\gls{cnf} graph schema}
\label{fig:CnfSchema}
\centering

\digraph[scale=0.5]{CnfSchema}{
graph [ranksep=0.3];
node [fontsize=14, shape=box, height=0, width=0];
edge [fontsize=14];

problem;
clause;
%{ rank = same;
predicate;
atom;
equality;
%}
argument;
term;
function;
variable;

problem -> clause [label=< contains >];
clause -> atom:nw [label=< pos >];
clause -> atom:ne [label=< neg >];
clause -> equality:nw [label=< pos >];
clause -> equality:ne [label=< neg >];
clause -> variable [label=< binds >];
atom -> predicate [label=< atom_applies>];
atom -> argument [label=< atom_has >];
equality -> term [label=< equalizes >];
equality -> variable [label=< equalizes >];
argument -> argument [label=< precedes >];
argument -> term [label=< is >];
argument -> variable [label=< is >];
term -> argument [label=< term_has >];
term -> function [label=< term_applies >];
}

\iffalse
\tikzstyle{token} = [rectangle, draw]

\begin{tikzpicture}[node distance = 1 and 2, ->]
% https://tex.stackexchange.com/a/332796/202639

% Node and edge types:
% https://docs.google.com/spreadsheets/d/1PCPHEgk6vLxpdpcvB_PGoLx7p4DID6WtvVWy2mDuv4A/edit?usp=sharing

\node (formula) [token] {\ntype{problem}};
% TODO: Consider removing the Formula nodes.
\node (clause) [token, below=of formula] {\ntype{clause}};
\node (atom) [token, below left=of clause] {\ntype{atom}};
\node (equality) [token, below right=of clause] {\ntype{equality}};
\node (predicate) [token, left=of atom] {\ntype{predicate}};
\node (argument) [token, below=of atom] {\ntype{argument}};
\node (term) [token, below=of argument] {\ntype{term}};
\node (function) [token, left=of term] {\ntype{function}};
\node (variable) [token, right=of term] {\ntype{variable}};

\draw (formula) to node [right] {\etype{contains}} (clause);
\draw (clause) to [bend right] node [above] {\epos{}} (atom);
\draw (clause) to [bend left] node [below] {\eneg{}} (atom);
\draw (clause) to [bend left] node [above] {\epos{}} (equality);
\draw (clause) to [bend right] node [below] {\eneg{}} (equality);
\draw (clause) to node [above] {\etype{binds}} (variable);
\draw (atom) to node [above] {\etype{atom\_applies}} (predicate);
\draw (atom) to node [left] {\etype{atom\_has}} (argument);
\draw (equality) to node {\etype{equalizes}} (term);
\draw (equality) to node {\etype{equalizes}} (variable);
\draw (argument) to [bend right] node [left] {\etype{is}} (term);
\draw (argument) to [loop left] node [left] {\etype{precedes}} (argument);
\draw (argument) to node [below] {\etype{is}} (variable);
\draw (term) to node [above] {\etype{term\_applies}} (function);
\draw (term) to [bend right] node [right] {\etype{term\_has}} (argument);

\end{tikzpicture}
\fi

\end{figure}
\todo{FB: Clean up the diagram.}

\begin{figure}[h]
\caption{Graph representation of the \gls{cnf} formula $a=b \wedge f(a,b) \neq f(b,b)$.}
\label{fig:GcnExample}
\centering
\digraph[scale=0.5]{GcnExample}{
graph [ranksep=0.3];
node [fontsize=14, shape=record, height=0, width=0];
edge [fontsize=14, dir=both, arrowtail=empty];

problem [label=<problem|a=b &and; f(a,b)&ne;f(b,b)>];

{ rank = same;
c0 [label="clause|a=b"];
c1 [label=<clause|f(a,b)&ne;f(b,b)>];
}

problem -> c0;
problem -> c1;

{ rank = same;
t0 [label="equality atom|a=b"];
t1 [label="equality atom|f(a,b)=f(b,b)"];
}

c0 -> t0 [label=" + "];
c1 -> t1 [label=" &ndash; "];

{ rank = same;
tfab [label="term|f(a,b)"];
tfbb [label="term|f(b,b)"];
}

{ rank = same;
ta [label="term|a"];
tb [label="term|b"];
}

ff [label="function|f", style=bold];
fa [label="function|a", style=bold];
fb [label="function|b", style=bold];
tfab0 [label="argument|1", style=dotted];
tfab1 [label="argument|2", style=dotted];
tfbb0 [label="argument|1", style=dotted];
tfbb1 [label="argument|2", style=dotted];
t0 -> ta;
t0 -> tb;
t1 -> tfab;
t1 -> tfbb;
tfab -> ff;
tfab -> tfab0;
tfab0 -> tfab1;
tfab0 -> ta;
tfab1 -> tb;
tfbb -> ff;
tfbb -> tfbb0;
tfbb0 -> tfbb1;
tfbb0 -> tb;
tfbb1 -> tb;
ta -> fa;
tb -> fb;

% Technical details:
%ff -> ff [dir=forward];
%fa -> fa [dir=forward];
%fb -> fb [dir=forward];
}
\end{figure}
\todo[author=R2]{There are unneeded differences in the notations used in the two figures; for example, pos/neg vs +/-, equality vs equality atom, the various (and non-specified) types of boxes around the nodes in fig. 3...}

The graph representation has the following properties:
\begin{itemize}
\item Lossless: The original problem can be faithfully reconstructed from the corresponding graph representation
(up to logical equivalence).
\item Signature agnostic: Renaming the symbols and variables in the input problem yields an isomorphic graph.
\item For each relation $r \in \mathcal{R}$, its inverse $\inv{r}$ is also present in the graph,
typically\todo{MS: jaka je vyjimka?} represented by a different edge type.
\todo{MS: tohle jsem nepochopil. Nemuzeme rict, ze (az na nejakou vyjimkou) jsou edges v obou smerech?\\
FB: Chci vyjasnit, ze inverzni edge ma jiny edge type.}
%\item A singleton node of type \ntype{problem} is connected to all the clauses of the problem.
\item The polarity of the literals is expressed by the type of the edge (\epos{} or \eneg{})
connecting the respective atom to the clause it occurs in.
\item For every non-equality atom and term, the order of its arguments is captured by a sequence of \ntype{argument} nodes chained by edges \cite{Rawson2020}.
\item The two operands of equality are not ordered.
This reflects the symmetry of equality.
\item Sub-expression sharing \cite{Chvalovsky2019,Olsak2019,Rawson2020}:
\todo[inline]{MS: (Almost)}
\todo[inline]{MS: Predpokladam ze dva ruzne ground literaly p(c) tam budou taky jen jendnou?\\
FB: Literaly nemaji nody. Atomy maji nody a polarita literalu je zakodovana v typu hrany, ktery jej spojuje s klauzuli.
MS: O to my, neslo. Slo o to, jestli se sdily ground termy mezi klauzulemi, ted uz dobry! :)}
Identical atoms and terms share a node representation.
\todo{FB: Cite some text about sub-expression sharing. Perhaps Rawson2020?}
\todo{MS: Ale mezitim se to tu zmenilo a ja prestavam chapat, co je ``Redundant''!}
%Note that since each variable is bound by a clause,
%ground terms are shared across clauses,
%but non-ground terms are only shared within the context of a clause.
\todo{FB: Remove as of little interest?}
\end{itemize}
\todo{FB: Reference appendix if we describe the representation in more detail there.}
\todo{FB: Mention other encodings that have been proposed by Mirek and Michael and compare ours to theirs.}

\subsection{\Gls{gcn}: From graphs to symbol embeddings}
\label{sec:gcn}

For each symbol in the input problem $P$,
we seek to find a vector representation, i.e., an \emph{embedding},
that captures the symbol's properties that are relevant
for correctly ranking the symbol in the symbol precedences over $P$.

The symbol embeddings are output by a \gls{rgcn} \cite{Schlichtkrull2017},
which is a stack of \emph{graph convolutional layers}.
Each layer consists of a collection of differentiable modules---one module per edge type.
\todo[author=R1]{Are there just so many nodes in the layer as the types of graph edges in Figure 2, or is each edge in the whole graph (a large set) represented by a collection of modules?}
The computation of the \gls{gcn} starts with assigning each node an initial embedding
and then iteratively updates the embeddings by passing them through the convolutional layers.

The initial embedding $h_a^{(0)}$ of a node $a$ is a concatenation of two vectors:
a \emph{feature vector} specific for that node (typically empty)
and a trainable vector shared by all nodes of the same type.
In our particular implementation,
feature vectors are used in nodes that correspond to clauses and symbols.
Each clause node has a feature vector with a one-hot encoding of the role of the clause,
which can be either axiom, assumption, or negated conjecture \cite{TptpSyntax,Sutcliffe2017}.
Each symbol node has a feature vector with two bits of data:
whether the symbol was introduced into the problem during preprocessing (most notably clausification),
and whether the symbol appears in a conjecture clause.

One pass through the convolutional layer
updates the node embeddings by passing a message along each of the edges.
For an edge of type $r \in \mathcal{R}$ going from source node $s$ to destination node $d$ at layer $l$,
the message is composed by converting the embedding of the source node $h_s^{(l)}$
using the module associated with the edge type $r$.
In the simple case that the module is a fully connected layer with weight matrix $W_r^{(l)}$ and bias vector $b_r^{(l)}$,
the message is $W_r^{(l)} h_s^{(l)} + b_r^{(l)}$.
% Notation is from Kipf: RGCN paper.
Each message is then divided by the normalization constant
$c_{s,d} = \sqrt{\card{\mathcal{N}_s^r}} \sqrt{\card{\mathcal{N}_d^r}}$ \cite{kipf2017semisupervised},
where $\mathcal{N}_a^r$ is the set of neighbors of node $a$ under the relation $r$.
\todo{FB: Mention that we mean both in- and out-neighbors?}

Once all the messages are computed,
they are aggregated at the destination nodes to form the new node embeddings.
Each node $d$ aggregates all the incoming messages of a given edge type $r$ by summation,
then passes the sum through an activation function $\sigma$ such as the \gls{relu},
and finally aggregates the messages across the edge types by summation,
yielding the new embedding $h_d^{(l+1)}$.

The following formula captures the complete update of embedding of node $d$ by layer $l$:
$$
h_d^{(l+1)} =
\sum_{r \in \mathcal{R}} \sigma \Parentheses{\sum_{s \in \mathcal{N}_d^r} \frac{1}{c_{s,d}} (W_r^{(l)} h_s^{(l)} + b_r^{(l)})}
$$

%$$
%h_i^{(l+1)} =
%\mathrm{LayerNorm} \Parentheses{h_i^{(l)} + \sum_{r \in \mathcal{R}} \sigma \Parentheses{\sum_{j \in %\mathcal{N}_i^r} \frac{1}{c_{ji}} h_j^{(l)} W_r^{(l)}}}
%$$
% Inspiration: https://ufal.mff.cuni.cz/~straka/courses/npfl114/1920/slides.pdf/npfl114-07.pdf - slide 27 - Transformer
%\todo{FB: Mention inspiration: Transformer. See Straka's slides.}

\todo{Compare our GCN to Michael, Mirek etc.}

\subsection{Output layer: From symbol embeddings to symbol costs}
\label{sec:output}
% Terminology: Deep Learning Book

The symbol cost of each symbol is computed by passing the symbol's embedding through a linear output unit,
which is an affine transformation with no activation function.

It is possible to use a more complex output layer in place of the linear unit,
e.g., a feedforward network with one or more hidden layers.
Our experiments showed no significant improvement when a hidden layer was added,
likely because the underlying \gls{gcn} learns a sufficiently complex transformation.

Let $\theta$ denote the vector of all the parameters of the whole \acrlong{nn} consisting of the \gls{gcn} and the output unit.
Given an input problem $P$ with signature $\Sigma = (s_1, \ldots, s_n)$,
we denote the cost of symbol $s_i$ predicted by the \acrlong{nn} as $c(i, P; \theta)$.
In the rest of this text,
we refer to the predicted cost of $s_i$ simply as $c(i)$
because the problem $P$ and the parameters $\theta$ are fixed in each respective context.

\subsection{Sort: From symbol costs to precedence}
\label{sec:sorting}

The symbol precedence heuristics commonly used in the \glspl{atp} sort the symbols by some numeric property
that is inexpensive to compute,
such as the number of occurrences in the input problem, or the symbol arity.
In our precedence recommender,
we use a \gls{gcn}-based symbol cost model $c$ in place of the simple symbol property,
\todo[author=R1]{What exactly is the GCN-bases symbol cost? Is it the cost calculated by
the model for each symbol, or are different copies of symbols in different occurrences given
different costs?}
and generate a precedence by sorting the symbols by their predicted costs.
An advantage of this scheme is that sorting is a fast operation.

Moreover, as we show in \cref{sec:training}, it is possible
to train the underlying symbol costs by gradient descent.

%Note that a precedence that minimizes $C$
%orders the symbols with the lowest cost as the last, prioritizing them for early inferences.

\section{Training procedure}
\label{sec:training}

In \cref{sec:architecture} we described the structure of a recommender system that generates a symbol precedence for an arbitrary input problem.
The efficacy of the recommender depends on the quality of the underlying symbol cost function $c$.
In theory, the symbol cost function can assign the costs so that
sorting the symbols by their costs yields an optimum precedence.
This is because, at least in principle, all the information necessary 
to determine the optimum precedence is present in the graph representation of the input problem
thanks to the lossless property of the graph encoding.
\todo{MS: Hlavne tady ale jakoby chybi druha pulky ty vety. Neco jako ``In practice, however, ...'' Nemyslis?} 
Our approach to defining an appropriate symbol cost function is based on statistical learning
from executions of an \gls{atp} on a set of problems with random precedences.

To train a useful symbol cost function $c$,
we define a precedence cost function $C$ using the symbol cost function $c$
in a manner than ensures that minimizing $C$ corresponds to sorting the symbols by $c$.
Finding a precedence that minimizes $C$ can then be done efficiently and precisely.
We proceed to train $C$ on the proxy task of ranking the precedences.
% We have:
% Data allows us to tell whether pi < rho.
% We need the symbols sorted by c be a good precedence.

\subsection{Precedence cost}

%Sorting a vector $x$ of length $n$ in a non-increasing order corresponds to finding a permutation $\pi$ that minimizes
%the sum of the components of $x$ weighted by their positions in $\pi$.
%By defining the cost of a precedence as a measure that is minimized by sorting,
%we obtain a practical notion that can ultimately be trained by gradient descent.

We extend the notion of cost from symbols to precedences
by taking the sum of the symbol costs
weighted by their positions in the given precedence $\pi$:
\todo[inline]{FB: Consider accessing elements $\pi$ as vector elements $\pi_i$ instead of $\pi(i)$. Keep in mind that $\inv{\pi}$ must be updated as well. The motivation for using $\pi(i)$ is that $\inv{\pi}(i)$ looks clearer than $\inv{\pi}_i$. How about using $(\inv{\pi})_i$? We may also omit the inversion completely.}
\todo{FB: Consider using the terminology of RankNet: $C(\pi) = o_i$, $S(\pi, \rho) = o_{ij}$, $\loss = C_{ij}$}
\begin{equation*} \label{eq:cost_model}
C(\pi) = Z_n \sum_{i=1}^n i \cdot c(\pi(i))% We call the normalization factor Z_n to follow convention from Foundations of Machine Learning, p. 122.
\end{equation*}
$Z_n = \frac{2}{n(n+1)}$ is a normalization factor
that ensures the commensurability of precedence costs across signature sizes.
More precisely, normalizing by $Z_n$ makes the expected value of the precedence cost
independent of the signature size $n$:
\begin{multline*}
\mathbb{E}_\pi [C(\pi)]
= \mathbb{E}_\pi \SquareBracket{Z_n \sum_{i=1}^n i \cdot c(\pi(i))}
= Z_n \sum_{i=1}^n i \cdot \mathbb{E}_\pi [c(\pi(i))] \\
= Z_n \Parentheses{\sum_{i=1}^n i} \mathbb{E}_i [c(i)]
= \frac{2}{n(n+1)} \frac{n(n+1)}{2} \mathbb{E}_i [c(i)]
= \mathbb{E}_i [c(i)]
\end{multline*}
\todo[inline,author=R2]{In the equations showing that the expected value of the precedence cost is independent of the size of the signature, there is a (minor) problem in the second line: the i occurring in $E_i[c(i)]$ is not defined and probably unneeded if, as hinted by the equations, the expected value of each symbol is the same (which should also be stated somewhere).}

When $C$ is defined in this way,
the precedence produced by the recommender (see \cref{sec:sorting}) minimizes $C$.
\todo{Marting == for the camera ready version ==
Write here that there could be alternavi formulas to \eqref{eq:cost_model}
and we pick this one as the arguably simplest, for the lack of a better choice.}

\begin{lemma}
Precedence cost $C$ is minimized by any precedence that sorts the symbols by their costs in non-increasing order:
$$
\argmin_\rho C(\rho) = \argsort^- (c(1), \ldots, c(n))
$$
where $\argsort^-(x)$ is the set of all permutations $\pi$
that sort vector $x$ in non-increasing order ($x_{\pi(1)} \geq x_{\pi(2)} \geq \ldots \geq x_{\pi(n)}$).
\end{lemma}

\begin{proof}
We prove direction ``$\argmin_\rho C(\rho) \subseteq \argsort^- (c(1), \ldots, c(n))$'' by contradiction.
Let $\pi$ minimize $C$ and let $\pi$ not sort the costs in non-increasing order.
Then there exist $k < l$ such that $c(\pi(k)) < c(\pi(l))$.
Let $\bar{\pi}$ be a precedence obtained from $\pi$ by swapping the elements $k$ and $l$.
Then we obtain
%$\bar{\pi} = (\pi(1), \ldots, \pi(k-1), \pi(l), \pi(k+1), \ldots, \pi(l-1), \pi(k), \pi(l+1), \ldots, \pi(n))$.
\begin{align*}
\frac{C(\bar{\pi}) - C(\pi)}{Z_n}
&= kc(\bar{\pi}(k)) + lc(\bar{\pi}(l)) - kc(\pi(k)) - lc(\pi(l)) \\
&= kc(\pi(l)) + lc(\pi(k)) - kc(\pi(k)) - lc(\pi(l)) \\
&= k(c(\pi(l)) - c(\pi(k))) - l(c(\pi(l)) - c(\pi(k))) \\
&= (k-l) (c(\pi(l)) - c(\pi(k))) \\
&< 0
\end{align*}
The final inequality is because $k-l < 0$ and $c(\pi(l)) - c(\pi(k)) > 0$.
Clearly, $Z_n > 0$ for any $n \geq 0$.
\todo{FB: Isn't the inclusion of $Z_n$ too confusing?}
Thus, $C(\bar{\pi}) < C(\pi)$, which is contradicts the assumption that $\pi$ minimizes $C$.

It is easy to see that all the precedences that sort the symbol costs in a non-increasing order
have the same precedence cost.
Since $\emptyset \neq \argmin_\rho C(\rho) \subseteq \argsort^- (c(1), \ldots, c(n))$,
each of the precedences in $\argsort^- (c(1), \ldots, c(n))$ has the cost $\min_\rho C(\rho)$.
It follows that $\argsort^- (c(1), \ldots, c(n)) \subseteq \argmin_\rho C(\rho)$.
\qed
\end{proof}

\todo[inline]{FB: Discuss that sorting would minimize even if we weighted by a strictly monotonous function other than identity.}

\todo[inline]{FB: Why is the equality predicate always the first in the precedence?
MS: Andrei Voronokov reasons. It's typically better in practice to postpone equational reasoning,
so we want equalities to be generally small.}

\subsection{Learning to rank precedences}
\label{sec:ranking}

Our ultimate goal is to train the precedence cost function $C$ so that it is minimized by the best precedence,
measuring the quality of a precedence by the number of iterations of the saturation loop taken to solve the problem.

Approaching this task directly, as a regression problem,
runs into the difficulty of establishing sensible target cost values for the precedences in the training dataset,
especially when a wide variety of input problems is covered.
Approaching the task as binary classification of precedences
seems possible, but it is not clear which precedences should be a priori
labeled as positive and which as negative, to give a guarantee
that a precedence minimizing the precedence cost (i.e. the one obtained by sorting)
would be among the best in any good sense.
\todo{FB: I think we could ensure that the precedence that minimizes $C$ is relatively good, for example the best of the observed. I hope that ranking generalizes better to unseen precedences, but I am not sure whether it does and I haven't done an experiment to confirm that.}
% such as learning to detect the precedences that yield a successful proof search within the allocated time.

We cast the task as an instance of score-based ranking problem \cite{Mohri2018,Burges2005}
% Mohri2018: p. 239, chapter 10 Ranking
by training a classifier to decide which of a \emph{pair} of precedences is better, based on their costs.
We train the classifier in a way that ensures that better precedences are assigned lower costs.
The motivation for learning to order pairs of precedences
is that it allows learning on easy problems,
and that it may allow the system to generalize to precedences that are better than any of those seen during training.
\todo{FB: We should test this hypothesis by training on one problem in isolation.}
\todo[inline]{MS: ta druha myslenka, o ktere jsem mluvil pred obedem, se deje nekde v tomhle odstavci.
Jasne je, ze chceme vypichnout uceni ze dvojic precedenci jako jeden z klicovych napadu. Trochu min jasne je, proc by teda dvojice mely byt dobre.
Ja mel pocit, ze kdyz jsme se vloni snazili na tenhle problem napasovat regresi, 
pusobilo ``neprirozene'' vnucovat modelu konkretni cisla z vampiru. Jasne, nejak to resila normalizace
(a muselo se davat pozor aby cisla davala smysl nejen napric problemy (jeden resi vsechno rychle, jiny obcas neresi -- penalizace)
ale i kvuli ruznym velikostem signatury. Coz teda tady mame porad.) ale porad tam bylo to, ze nejak z vnejsku
modelu narizujeme, na ktere hodnoty ma cilit.
My tady ale vime, ze nam na konkretnich hodnotach sybol costu nezalezi, 
a ze ani nemusi byt rozumne porovnavatelne mezi problemy,
jediny, co je dulezity, je, aby setridena permutace byla ta (to the best of model's knowledge) nejlepsi.
Takze ja bych shrnul, ze s dvojicema obchazime nutnost stanovovat konkretni target values pro regresi (``problem volby spolecnych jednotek velicin''?)
a zaroven rafinovane menime ulohu z regresivni na classifikacni.\\
FB: Regrese $C$ neni jedina alternativa. Mohli bychom klasifikovat jednotlive precedence (viz text). Mohli bychom relaxovat pozadavek, ze setridena permutace je nejlepsi; stacila by "dost dobra". Zkusil jsem nejak vyargumentovat, proc je klasifikace dvojic lepsi nez klasifikace jednotlivych precedenci.}

\todo{Intuice: snazime se nastavit costy tak, aby co nejvic dvojic dopadlo spravne.}
\todo{Some pairs of precedences are not informative. We hope that the non-systematic examples will cancel each other out.}

\todo[inline]{MS: Tady je asi hlavni ``dira'' ve vysvetlovani...
Na to, abys vzal symbol costy a zavolal sort nepotrebujes definovat
cost precedenci a neco o nem dokazovat.

Nevim, jestli to nechces rict az pozdeji,
nebot precedence cost opravdu potrebujem az pro uceni!

Teda vlastne by mi nevadilo o tom mluvit zde, (az na to, ze rozcestnik drive rika
``Sorting converts the symbol costs into a symbol precedence.
It is only used in generating mode.'')

ale to vyzneni, kde to vypada, ze to delame kvuli tomu sortu, je matouci.

Predstavuju si odstavec, kde rikame, ze precedence cost je dulezity koncept,
protoze ma jednak tu vlastnost, ze je minimalizovan sortenim
ale druhak prirazuje cost i nesetridenym permutacim a tak ho muzem pouzit pri uceni.
(Coz by splnoval i jiny vzorecek nez \eqref{eq:cost_model}), namely
jina striktne monotonni funkce na pronasobeni nez to $i$ tam,
ale berem proste tenhle, protoze neni jasne, proc by jiny mel byt lespi,
a tenhle je nejjednodussi.)}

\subsubsection{Training data}

Each training example has the form $(P, \PrecBetter, \PrecWorse)$,
where $P = (\Sigma, \mathit{Cl})$ is a problem
and $\PrecBetter, \PrecWorse$ are precedences over $\Sigma$
such that the prover using $\PrecBetter$ solves $P$ in fewer iterations of the saturation loop than with $\PrecWorse$,
denoted as $\Better{\PrecBetter}{\PrecWorse}{P}$.
\todo{FB: Is this sufficient?}
\todo{FB: Describe the distribution of problems and precedences.}

\subsubsection{Loss function}

Let $(P, \PrecBetter, \PrecWorse)$ be a training example ($\Better{\PrecBetter}{\PrecWorse}{P}$).
The precedence cost classifies this example correctly if $C(\PrecBetter) < C(\PrecWorse)$,
or alternatively $S(\PrecBetter, \PrecWorse) = C(\PrecWorse) - C(\PrecBetter) > 0$.
% We have positive S score for correctly predicted examples.
% The loss function transorms that into a loss value close to 0.
We approach this problem as an instance of binary classification with the logistic loss \cite{Mohri2018},
% p. 128
a loss function routinely used in classification tasks in \acrlong{ml}:
\todo{FB: Consider exposing the form with inverse precedences to single out the $c$ values and expose the $c$ gradient better.}
\begin{multline*}
\loss(P, \PrecBetter, \PrecWorse)
= - \log \sigmoid S(\PrecBetter, \PrecWorse)
= - \log \sigmoid (C(\PrecWorse) - C(\PrecBetter)) \\
= - \log \sigmoid Z_n \sum_{i=1}^n i (c(\PrecWorse(i)) - c(\PrecBetter(i)))
%= - \log \sigmoid Z_n \sum_{i=1}^n c(i) (\inv{\PrecWorse}(i) - \inv{\PrecBetter}(i))
\end{multline*}
% FB: Note: I have included the full expansion to make it visible how the loss depends on $c$.
% To a skilled NN practitioner, it should now be obvious that the loss is differentiable w.r.t. $c$.

Note that the classifier cannot simply train $S$ to output a positive number on all pairs of precedences
because $S$ is defined as a difference of two precedence costs.
Intuitively, by training on the example $(P, \PrecBetter, \PrecWorse)$
we are pushing $C(\PrecBetter)$ down and $C(\PrecWorse)$ up.

The loss function is clearly differentiable with respect to the symbol costs,
and the symbol cost function $c$ is differentiable with respect to its trainable parameters.
This enables the use of gradient descent to find the values of the parameters of $c$
that minimize the loss value.
\todo[author=R2]{I expect that you mean "to *locally* minimize the loss value". If you do, the fix is simple (just add "locally"); if not, more explanations are needed.}

\Cref{fig:architecture} shows how the loss function is plugged into the recommender for the training.


\section{Experimental evaluation}
\label{sec:evaluation}

To demonstrate the capacity of the precedence recommender described in \cref{sec:architecture},
a series of experiments was performed.
\todo{MS: vzdycky kdyz to jen trochu jde, snaz se pouzivat cinny rod (a pritomny cas): To demonstrate ..., we performed ... Ale taky tu trochu chybi, co nas ceka dal, ve stylu: In this section, we report ... }
In this section, we describe the design and configuration of the experiments,
and then compare the performance of the trained models to baseline heuristics.

\subsection{Solver}

The empirical evaluation was performed using a modified version of the \gls{atp} Vampire 4.3.0.\cite{10.1007/978-3-642-39799-8_1}
The prover was used to generate the training data and to evaluate the trained precedence recommender.
To generate the training data,
Vampire was modified to output \gls{cnf} representations of the problems
and annotated problem signatures in a machine-readable format.
%in \gls{json} format and annotated problem signatures in \gls{csv} format.\todo{MS: Mentioning JSON or CSV seems to be too low level. ``programmer documatation'' material.}
For the evaluation of the precedences generated by the recommender,
Vampire was modified to allow the user to specify the predicate and function symbol precedences.
\todo[inline]{FB: Add a link to the modified Vampire.}

% All explicit options:
% vampire --encode on --statistics full --time_statistics on --proof off --avatar off --saturation_algorithm discount --age_weight_ratio 10 --literal_comparison_mode predicate --symbol_precedence frequency --time_limit 10

% Notable explicit options:
% vampire --avatar off --saturation_algorithm discount --age_weight_ratio 10 --literal_comparison_mode predicate --symbol_precedence frequency --time_limit 10

% Notable implicit options:
% --term_ordering kbo

\todo[inline,author=FB]{Describe the Vampire options.}
\todo{MS: Literal comparison mode is not a standard thing. Mentioning the option name is again too low level. Explaining the effect (point two) is appropriate, although a bit too vague. To kdyztak opravim ja. 
Popsat presne setup je dulezite, kvuli reprodukovatelnosti, ale ne nutne primo v clanku (budem nejspis stejne dodavat odkaz na nejake reproducibility repo).
Ted zminit AVATAR je nevhodne, pokud nevysvetlis, co to aspon ramcove je. A pokud to budes vysvetlovat, je otazka, proc tim ztracet misto, kdyz to neni bezprostredne relevantni pro dalsi vysvetlovani.}

%For the experiment, Vampire 4.3.0 was modified to convert the problems from the \gls{tptp} format to a representation that allows easy automated processing.
%More precisely, when the modified prover is run in the clausification mode (\texttt{vampire -mode clausify}),
%it outputs the \gls{cnf} form of the problem and the problem signature in formats that allow easy processing in Python.
%The \gls{cnf} clause and term structure of the problem is encoded into the \gls{json} format,
%and the signature along with additional symbol metadata is encoded into the \gls{csv} format.

\subsection{Training data}

The training data consists of examples of the form $(P, \PrecBetter, \PrecWorse)$,
where $P$ is a \gls{cnf} problem and $\PrecBetter, \PrecWorse$ are precedences of symbols of problem $P$
such that out of the two precedences, $\PrecBetter$ yields a proof in fewer iterations of the saturation loop.
\todo{FB: Justify saturation loop iterations as a proxy for success.}
\todo{FB: Reference a detailed explanation of saturation loop iterations.}

Since neither of the common \glspl{sot}, i.e., \gls{kbo} nor \gls{lpo}, ever compares a predicate symbol with a function symbol,
two separate precedences can be considered for each problem:
a predicate precedence and a function precedence.
A predicate precedence recommender is trained separately from a function precedence recommender
to simplify the training process and to isolate the effects of the predicate and function precedences.
This section describes how the training data for training a \emph{predicate} precedence recommender is generated.
Training data for training a function precedence recommender is generated analogously.

\subsubsection{Problem set}

The examples are generated by sampling the \gls{fol} part of the problem library \gls{tptp} v7.4.0.
A total of \num{17053} problems formulated in \gls{fof} and \gls{cnf} are sampled.
The problems formulated in \gls{fof} are converted to \gls{cnf} by Vampire.
% vampire --mode clausify
Let $\ProblemsTptp$ denote the set of all \num{17053} \gls{cnf} problems available for training and evaluation.

\subsubsection{Sampling}

The examples are generated by iterative sampling of $\ProblemsTptp$.
In each iteration, a problem $P \in \ProblemsTptp$ is chosen and Vampire is executed on $P$
with two predicate precedences chosen uniformly from the set of all predicate precedences on $P$.
A function precedence for $P$ is chosen uniformly and used for both executions.
\todo[author=FB]{Why do we sample function precedences randomly?}

The two executions are compared in terms of performance:
predicate precedence $\PrecBetter$ is recognized as better than precedence $\PrecWorse$,
denoted as $\Better{\PrecBetter}{\PrecWorse}$,
if the proof search finishes successfully with $\PrecBetter$ ($\SolverRun{P}{\Prec} = \top$)
and if the number of iterations of the saturation loop with $\PrecBetter$ is smaller than with $\PrecWorse$.
\todo{FB: Use notation for saturation loop iterations if the notation was established earlier. If failure is denoted by $\infty$, we can simply use $<$.}
If one of the two precedences is recognized as better,
the example $(P, \PrecBetter, \PrecWorse)$ is produced,
where $\PrecBetter$ is the better precedence,
and $\PrecWorse$ is the other precedence.

To ensure the efficiency of the sampling, the process is interpreted as an instance of Bernoulli multi-armed bandit problem,
\todo{Cite?}
with the reward of a trial being 1 in case an example is produced, and 0 otherwise.
Adaptive sampling balances
exploring problems that have been tried relatively scarcely and
exploiting problems that have yielded examples relatively often.
For each problem $P \in \ProblemsTptp$,
the generator keeps track of the number of times the problem has been tried $n_P$
and the number of examples generated from that problem $s_P$.
The ratio of $s_P$ to $n_P$ corresponds to the average reward of problem $P$ observed so far.
The problems are sampled using the allocation strategy \acrshort{ucb1} \cite{Auer2002} with a parallelizing relaxation.
In each iteration, the generator samples the problem $P$ that maximizes
$$
\frac{s_P}{n_P} + \sqrt{\frac{2 \ln n}{n_P}}
$$
where $n = \sum_{P \in \ProblemsTptp} n_P$ is the total number of tries on all problems.
The parallelizing relaxation means that the $s_P$ values are only updated once in \num{1000} iterations,
allowing up to \num{2000} parallel solver executions.
\todo{Explain in detail?}

The sampling runs until a total of \num{1000000} examples are generated.
% sftp://cluster.ciirc.cvut.cz/home/bartefil/git/vampire-ml/out/20210108-generate-predicate/questions_generated
For example, while generating \num{1000000} examples for the predicate precedence dataset,
the least explored problem was tried 19 times, and the most exploited problem was tried 504 times.
\num{5349} out of the \num{17053} problems yielded at least one example.
\todo{FB: Remove these two sentences because the give too arbitrary detail?}

\subsubsection{Split}

The problems in $\ProblemsTptp$ are split roughly in half to form the train set and the validation set.
Both training and validation sets are restricted to problems whose graph representation consists of at most \num{100000} nodes
to limit the memory requirements of the training.
The train set is further restricted to problems that correspond to at least one training example.
In total there are \num{7648} problems in the validation set $\ProblemsVal$
and \num{2571} problems in the train set $\ProblemsTrain$.
\todo{FB: Unify the terminology: validation, or train set?}
% /home/filip/projects/vampire-ml/vampire-ml/outputs/2021-02-11/16-27-02/results.csv
% all: 17053
% train: 8527
% val: 8526
% with_questions: 5349
% graphified (not all attempted): 10219
% train&with_questions: 2647
% train&with_questions&graphified: 2571
% val&graphified: 7648
\todo{MS: jeste by nebylo lepsi vzit nejdriv ty ``male'' problemy a pak teprve pulit.}

\subsection{Training procedure}

A predicate symbol cost model is trained by gradient backpropagation
\todo{FB: Explain?}
on the precedence pair classification task
\todo{FB: Reference a section.}
using the examples generated from the problems in $\ProblemsTrain$.
\todo{Add references to the sections that describe the architecture.}
To avoid redundant computations, all examples generated from a problem are processed in the same training batch.
Thus, each training batch contains up to \num{128} problems and all the examples generated from these problems.
The symbol cost model is trained using the Adam optimizer \cite{Kingma2014}.
Learning rate starts at \num{1.28e-3}
and is halved each time the train loss stagnates for 10 consecutive epochs.
% tf.keras.callbacks.ReduceLROnPlateau(monitor='loss', factor=0.5, patience=10)
A \gls{gcn} with depth 4 and message size 16 is used.\todo{MS: zase pasivum. Tady to urcite bude chtit odkaz na sekci, kde se jasne definuje, co to znamena. Jinak recenzenti byvaji dost alergicti na hyperpametry, ktere se jen tak nahodi na nevysvetli. Minimalne nejaky komentar typu: treba, zkouseli jsme u ruzne dalsi hodnoty, ale dopadalo to podobne. Nebo, pozdeji rekeneme, co se deje, kdyz se tohle meni. Nebo: nemeli jsme cas zjistovat, co se deje, kdyz se to meni, tak berem rozumne hodnoty podle blabla.}

Each of the training examples of problem $P$ contributes to the training with the weight $\frac{1}{s_P}$,
where $s_P$ is the number of examples of problem $P$ in the training set,
so that each problem contributes to the training to the same degree irrespective of the relative numbers of examples.
\todo{FB: Shall we remind that the examples are further normalized by $k(n)$ to make even across signature lengths?}

\subsection{Environment}

All the experiments were run on a computer with the following specification:

\begin{itemize}
\item CPU: Intel Xeon Gold 6140 (72 cores @ \SI{2.30}{GHz})
\item RAM: \SI{383}{GiB}
\item OS: Ubuntu 20.04
\item Kernel: GNU/Linux 5.4.0-40-generic x86\_64
\end{itemize}

% From the paper Reliable benchmarking [BenchExec]:
% – CPU model and size of RAM,
% – specified resource limits,
% – name and version of OS,
% – version of important software packages, such as the kernel or runtime environments like the Java VM,
% – version and configuration of the benchmarked tool(s), and
% – version of the benchmark set (input files)

\subsection{Results}

The final evaluation is performed on $\ProblemsVal$.
For each problem $P \in \ProblemsVal$,
a precedence is generated by the trained recommender
and Vampire is run using this precedence and a wall clock time limit of 10 seconds.
The results are averaged over 5 runs to reduce the effect of noise due to wall clock time limit.
As a baseline, the performance of Vampire with the \texttt{frequency} precedence heuristic was evaluated
with the same time limit.

To generate a precedence for a problem,
the recommender first converts the problem to a machine-friendly \gls{cnf} format,
then converts the \gls{cnf} to a graph,
then predicts symbol costs using the \gls{gcn} model
and finally orders the symbols by their costs to produce the precedence.
To simplify the experiment, the time limit is only imposed on the Vampire run
and excludes the time taken by the recommender to generate the precedence.
\todo{Update if we include the preprocessing time.}

\Cref{tab:results} shows the results.

\begin{table*}
\caption{
Results of the evaluation of various predicate precedence heuristics on $\ProblemsVal$.
Means and standard deviations over 5 runs are reported.
}
\centering
\begin{tabular}{l|ll|ll}

Model & \multicolumn{2}{l}{Successes out of \num{7648}} & \multicolumn{2}{l}{Successes rate} \\
& Mean & Std & Mean & Std \\
\hline

\acrshort{gcn} (predicate) & \num{3923.6} & \num{2.24} & \SI{51.30}{\percent} & \num{2.93e-4} \\
% Success rate: mean: 0.513023013, std: 0.000292887
% Evaluation results: https://ui.neptune.ai/filipbartek/vampire-ml/e/VML-553
% Total: validation_solver_eval/all/problems/measured&split&category: 7648
% Success mean: validation_solver_eval/all/success/count/mean: 3923.6
% Success std: validation_solver_eval/all/success/count/std 2.24
% Success rate: 0.513023013
% Difference in success count from baseline: 154 ~ 0.020135983
% Estimated difference from baseline (estimate on 891 problems): 0.021099888
% Checkpoint: outputs/2021-02-06/14-55-41/tf_ckpts/epoch/weights.00079-0.61.tf VML-540 0.511785

\acrshort{gcn} (function) & ? & ? & ? & ? \\

\texttt{vampire -lcm predicate} & \num{3769.6} & \num{3.07} & \SI{49.29}{\percent} & \num{4.01e-4} \\
% Success rate: mean: 0.492887029, std: 0.000401412
% https://ui.neptune.ai/filipbartek/vampire-ml/e/VML-490
% /home/filip/projects/vampire-ml/vampire-ml/outputs/2021-02-09/12-13-43
% Row: 'val&graphified&solver_eval'
% Problems total: 7648
% Success rate: 0.492887029

\texttt{vampire -lcm standard} & ? & ? & ? & ? \\

\end{tabular}
\label{tab:results}
\end{table*}
\todo{FB: Add links to Neptune runs?}
\todo{FB: Add results for function precedences.}
\todo{FB: Try to evaluate a recommender that uses both predicate and function models.}

\todo{FB: Add discussion.}
\todo{FB: Evaluate on problems larger than 100k nodes.}


\section{Related work}

MS: my sami (PAAR paper) jsem si related workem. Budem se vuci tomu muset vymezit.
Jinak me ale nic moc bezprostredniho nenepada a nevim tedy, jestli se tedy primo
nutit do sekce s timto nazvem!

\section{Conclusion}

\subsection{Future work}

Generalization from trail acc to val acc may be improved.

Highest val accuracy does not transfer to highest ATP success rate. Generalization to ATP is sub-par.

Larger graphs

Joint model (trained jointly on predicate and function precedences)

Differentiate questions due to satiterations from questions due to success and failure

MS: Is there a knowledge transfer possible between related setups when we in vampire change
options (seemingly?) orthogonal to precedence (saturation algorithm, additional inference rules / reductions, AVATAR on/off,
clause selection strategies, etc.)?

KBO is a also parametrized by weights (along with the precedence). Could we learn those as well? (Jointly?)

Reinforcement learning will allow using a trained recommender to generate data from harder problems.

In our experiments, too much attention is spent on uninteresting examples.
We learn to order pairs of non-optimal precedences.
However, if the recommender ordered a non-optimal pair wrong, it could still produce the optimum precedence correctly.
Besides, we train from pairs of successful runs that only differ in number of iterations.

\section*{Acknowledgments}
% > Acknowledgements should generally be placed in an unnumbered subsection at the end of the paper.

% https://sgs.cvut.cz/index.php?action=faq
% This work was supported by the Grant Agency of the Czech Technical University in Prague, grant No. SGS...

\todo{MS: zvazit, jestli nerozepsat zvlast Martin, zvlast Filip.
Pripadne se zeptat Hanky, jestli bychom nemeli nejak vypichnout
jen GACR.}

This work was supported by
the Czech Science Foundation project 20-06390Y,
the project RICAIP no. 857306 under the EU-H2020 programme,
and
the Grant Agency of the Czech Technical University in Prague, grant\\
no.~SGS20/215/OHK3/3T/37.


% > LaTeX users should avoid self-defined environments and use the bibliographic style MathPhySci for computer science proceedings.
% > It is not possible to have hyperlinks in references.
\bibliographystyle{splncs04}
\bibliography{main}

\appendix

\section{Graph structure}

\begin{itemize}
\item The graph representation of problem $P$ contains exactly one root node of type \ntype{problem}.
\item Each clause is represented by a \ntype{clause} node connected to the root \ntype{problem} node.
\item Each atom is represented by an \ntype{atom} node (in case the atom is not an equality)
or an \ntype{equality} node (in case the atom is an equality).
For each literal occurrence there is an edge connecting the respective atom to the respective clause.
The type of the edge corresponds to the polarity of the literal:
\epos{} for positive literal and \eneg{} for negative literal.
\item Each \ntype{equality} node is connected to two nodes that represent the operands,
each of which is of type \ntype{term} or \ntype{variable}.
The commutativity of the equality operator is reflected by the fact that the operand edges are not ordered.
\item Each \ntype{atom} node is connected to one \ntype{predicate} node that represents the predicate symbol being applied by this atom.
Note that these edges connect the applications of a predicate symbol across the whole problem.
\item Each \ntype{atom} node is connected to zero or more \ntype{argument} nodes that represent the argument positions of the atom.
\item Each pair of \ntype{argument} nodes that correspond to consecutive argument positions is connected by an edge.
\item Each \ntype{argument} node is connected to a node that represents the argument term,
which is either a \ntype{term} node or a \ntype{variable} node.
\item Each \ntype{term} node is connected to one \ntype{function} node that represents the function symbol being applied by this term.
\item For each variable, there is an edge connecting the \ntype{clause} node of the clause that binds the variable to the \ntype{variable} node that represents the variable.
\end{itemize}

\section{Loss derivative}

The loss is differentiable with respect to the symbol costs:
\begin{align*}
\frac{\partial \loss}{\partial c_i}
&= -\sigmoid(-C(\Better{\PrecBetter}{\PrecWorse}{P})) \cdot k(n) \cdot (\inv{\PrecWorse}_i - \inv{\PrecBetter}_i) \\
&= (p(\Better{\PrecBetter}{\PrecWorse}{P}) - 1) \cdot k(n) \cdot (\inv{\PrecWorse}_i - \inv{\PrecBetter}_i)
\end{align*}

This means that it is possible to backpropagate the loss gradient into the symbol cost model. \todo{MS: tohle bude tezky,
ale ackoliv ML people by tohle uz asi chapali, ATP crowd spis bude
potrebovat obsirnejsi vysvetleni.}

\section{Experiment details}

\begin{figure}[h]
\caption{The dependence of preprocessing time on the number of nodes in the problem graph on 1000 random validation problems}
\label{fig:preprocessing}
\centering
% \includesvg[width=\textwidth]{preprocessing}
% Source: https://docs.google.com/spreadsheets/d/1GujYNEtETpC3jk4iyENLjptv8mDmI5f1ZEhbmwtqnPM/edit#gid=394425425&fvid=659935521
% Experiment: VML-715
\end{figure}

\end{document}
