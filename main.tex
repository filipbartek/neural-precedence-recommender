\documentclass[runningheads]{llncs}
\usepackage[utf8]{inputenc}
\usepackage[T1]{fontenc}
\usepackage[english]{babel}
\usepackage[binary-units=true,detect-weight=true]{siunitx}

% How to write a paper:
% https://mj.ucw.cz/papers/jakpsat.pdf
% Jones 2016: https://www.microsoft.com/en-us/research/academic-program/write-great-research-paper/
% AWR: https://jazyky.fel.cvut.cz/vyuka/RPP/BE9M04AKP/

% PAAR paper: https://github.com/filipbartek/learning-precedences-from-elementary-symbol-features/releases/download/paar2020%2Fceur-2/Learning_Precedences_from_Simple_Symbol_Features.pdf

\usepackage{amsfonts}
\usepackage{amsmath}
\usepackage{amssymb}
\usepackage{mathtools}
\usepackage{stmaryrd}
%\usepackage[disable]{todonotes}
\usepackage{todonotes}

% Multi-letter identifier
\newcommand{\mli}[1]{\mathit{#1}}

\DeclareMathOperator{\re}{\mathbb{R}}
\DeclareMathOperator{\nat}{\mathbb{N}}
\DeclareMathOperator{\logit}{logit}
\DeclareMathOperator{\sigmoid}{sigmoid}
\DeclareMathOperator*{\argmin}{argmin}
\DeclareMathOperator{\argsort}{argsort}
\newcommand{\inv}[1]{#1^{-1}}
\DeclarePairedDelimiter{\card}{\lvert}{\rvert}
\DeclarePairedDelimiter{\SquareBracket}{[}{]}
\DeclarePairedDelimiter{\Parentheses}{(}{)}
\newcommand{\DotProd}[2]{\left<#1,#2\right>}
\newcommand{\Better}[3]{#1 \prec_{#3} #2}
\newcommand{\Prob}[1]{\mathrm{Prob}(#1)}

\DeclareMathOperator{\symbols}{\Sigma}
\newcommand{\Problems}[1]{\mathcal{P}_{#1}}
\DeclareMathOperator{\cnf}{\Problems{\mathrm{CNF}}}
\DeclareMathOperator{\ProblemsTptp}{\Problems{0}}
\DeclareMathOperator{\ProblemsTrain}{\Problems{\mathrm{train}}}
\DeclareMathOperator{\ProblemsTrainEx}{\ProblemsTrain'}
\DeclareMathOperator{\ProblemsVal}{\Problems{\mathrm{val}}}
\DeclareMathOperator{\ProblemsValEx}{\ProblemsVal'}
\newcommand{\signature}[1]{\Sigma_#1}

% Precedences
\newcommand{\PrecBetter}{\pi}
\newcommand{\PrecWorse}{\rho}

\newcommand{\Vampire}{\textsc{Vampire}}

\newcommand{\loss}{\ell}
% Inspiration: The Elements of Statistical Learning, p. 120

\usepackage{glossaries}

\newacronym{ann}{ANN}{artificial \acrlong{nn}}
\newacronym{atp}{ATP}{automated theorem prover}
\newacronym{casc}{CASC}{CADE ATP System Competition}
% http://www.tptp.org/CASC/
\newacronym{ciirc}{CIIRC}{Czech Institute of Informatics, Robotics and Cybernetics}
\newacronym{cnf}{CNF}{clause normal form}
\newacronym{csv}{CSV}{comma-separated values}
\newacronym{ctu}{CTU}{Czech Technical University in Prague}
\newacronym{dag}{dag}{directed acyclic graph}
\newacronym{fel}{FEL}{Faculty of Electrical Engineering}
\newacronym{fof}{FOF}{first-order form}
% http://www.tptp.org/TPTP/TR/TPTPTR.shtml
\newacronym{fol}{FOL}{first-order logic}
\newacronym{gcn}{GCN}{graph convolutional network}
\newacronym{gnn}{GNN}{graph \acrlong{nn}}
\newacronym{hin}{HIN}{heterogeneous information network}
\newacronym{json}{JSON}{JavaScript Object Notation}
\newacronym{kbo}{KBO}{Knuth-Bendix ordering}
\newacronym{lpo}{LPO}{lexicographic path ordering}
\newacronym{ml}{ML}{machine learning}
\newacronym{nn}{NN}{neural network}
\newacronym{relu}{ReLU}{Rectified Linear Unit}
\newacronym{rgcn}{R-GCN}{relational \acrlong{gcn}}
\newacronym{tkbo}{TKBO}{transfinite \acrlong{kbo}}
\newacronym{tptp}{TPTP}{Thousands of Problems for Theorem Provers}
% http://www.tptp.org/
\newacronym{ucb}{UCB}{Upper Confidence Bound}
\newacronym{ucb1}{UCB1}{\gls{ucb}1}

\newglossaryentry{sot}{
name={simplification ordering on terms},
description={},
plural={simplification orderings on terms}
}

% !TEX root = main.tex

\title{Neural Precedence Recommender}
% TODO: Consider prepending an article: "A Neural Precedence Recommender"

% > The corresponding author, i.e., the author responsible for checking the final proof and for signing the copyright form on behalf of all of the authors, should be clearly marked in the header of the paper.
% > The inclusion of the corresponding author’s email address is  mandatory.

% > We strongly recommend that all authors include their email addresses in their papers.

\author{
Filip Bártek\inst{1,2}\orcidID{0000-0002-1822-2651} \and
Martin Suda\inst{1}\orcidID{0000-0003-0989-5800}
}

\authorrunning{F. Bártek \and M. Suda}

% > The affiliated institutions, including town/city and country
\institute{
\acrlong{ciirc}\\
\acrlong{ctu}\\
Jugoslávských partyzánů 1580/3, 160 00 Praha 6 -- Dejvice, Czech Republic\\
\email{\{filip.bartek,martin.suda\}@cvut.cz}\\
% TODO: Consider removing Martin's email.
% TODO: Consider changing Martin's email to martin.suda@gmail.com.
%\url{https://www.ciirc.cvut.cz/}
\and
\acrlong{fel}\\
\acrlong{ctu}\\
Technická 2, 166 27 Praha 6 -- Dejvice, Czech Republic\\
%\url{http://www.fel.cvut.cz/}
}


\usepackage{svg}

\usepackage[pdf]{graphviz}
\usepackage{tikz}

\usepackage{hyperref}
\hypersetup{
pdftitle={Neural Precedence Recommender},
pdfauthor={Filip Bártek, Martin Suda},
pdfkeywords={saturation-based theorem proving, simplification ordering, symbol precedence, machine learning, graph convolutional network}
}

\usepackage{cleveref}

% Some mild shrinkage:

% for Section and subsection headings:
%% Save the class definition of \subparagraph
\let\llncssubparagraph\subparagraph
%% Provide a definition to \subparagraph to keep titlesec happy
\let\subparagraph\paragraph
%% Load titlesec
\usepackage{titlesec}
%% Revert \subparagraph to the llncs definition
\let\subparagraph\llncssubparagraph
\titlespacing*{\subsection}
  {0pt}{0.9\baselineskip}{0.8\baselineskip}

% For figures
% https://tex.stackexchange.com/questions/60477/remove-space-after-figure-and-before-text
\setlength{\textfloatsep}{0.7\baselineskip plus 0.2\baselineskip minus 0.2\baselineskip} % between a float at the top and text
\setlength{\floatsep}{0.3\baselineskip plus 0.2\baselineskip minus 0.2\baselineskip} % between two floats at the top
\setlength{\intextsep}{0.7\baselineskip plus 0.2\baselineskip minus 0.2\baselineskip} % between a float in text and text

\begin{document}

\maketitle

\begin{abstract}
% !TEX root = main.tex

The state-of-the-art superposition-based theorem provers for \acrlong{fol}
rely on \glspl{sot} to constrain the applicability of inference rules,
which in turn shapes the ensuing search space.
The popular Knuth-Bendix simplification ordering is parameterized by 
a \emph{symbol precedence}---a permutation of the predicate and function symbols
of the input problem's signature.
Thus, the choice of a precedence has an indirect yet often substantial impact
on the amount of work required to successfully complete a proof search.

This paper describes and evaluates a symbol precedence recommender,
a \acrlong{ml} system that estimates the best possible precedence
based on observations of prover performance on a set of problems and random precedences.
Using the graph convolutional neural network technology,
the system does not presuppose the problems to be related or share a common signature. 
When coupled with the theorem prover \Vampire{} and evaluated on the \acrshort{tptp} problem library,
the recommender is found to outperform a state-of-the-art heuristic by more than \SI{4}{\percent}
on unseen problems.

\keywords{saturation-based theorem proving \and
simplification ordering \and
symbol precedence \and
\acrlong{ml} \and
\acrlong{gcn}}

% AWR: Guidelines (Unit 7, page 5):
% At most 300 words
% Content:
% Purpose of the study
% Research problems
% Basic design of the study
% Summary of interpretations and conclusions

% AWR: Parts of an abstract (checklist) (Unit 7, page 6):
% Motivation
% Problem statement
% Approach
% Results
% Conclusions

% Motivation and purpose of the study are excluded because the audience at CADE need not be reminded of the motivation for making the provers faster.

\end{abstract}

\section{Introduction}

% !TEX root = main.tex

% General motivation for FOL ATPing:
% Vampire is used as a sledgehammer in Isabelle/HOL:
% 1. https://people.mpi-inf.mpg.de/~jblanche/life.pdf
% 2. https://www.cl.cam.ac.uk/~lp15/papers/Automation/paar.pdf
% Formal methods for verification (survey): https://arxiv.org/pdf/1912.03028.pdf

% Jones 2016 guidelines:
% 1. Problem (explain by example)
% 2. Contributions (refutable, forward references to sections)

% AWR structure (Unit 7, page 1):
% 1. Attention-getter (lead-in) [1-2 sentences]
% 2. Set up for the thesis [minimum: 2-3 sentences]
% 3. Thesis statement (essay map) [1 sentence]

% AWR guidelines (Unit 7, page 5):
% Whet the reader's appetite
% Set the context
% State why the main idea is important
% State your thesis/claim

Modern saturation-based Automatic Theorem Provers (ATPs) such as E \cite{Schulz2019}, SPASS \cite{DBLP:conf/cade/WeidenbachDFKSW09},
or \Vampire{} \cite{DBLP:conf/cav/KovacsV13}
employ the superposition calculus \cite{DBLP:journals/logcom/BachmairG94,DBLP:books/el/RV01/NieuwenhuisR01} as their underlying inference system.
Integrating the flavors of resolution \cite{DBLP:books/el/RV01/BachmairG01}, paramodulation \cite{Robinson1983}, and 
the unfailing completion \cite{Bachmair89completionwithout},
superposition is a powerful calculus with
native support for equational reasoning.
The calculus is parameterized by a simplification ordering on terms % and literals,
and uses it to constrain the applicability of inferences, with a significant impact on performance.

Both main classes of simplification orderings used in practice,
the \acrlong*{kbo} \cite{Knuth1983}
and the \acrlong*{lpo} \cite{Kamin1980},
are specified with the help of a 
\emph{symbol precedence}, an ordering on the signature symbols. %.\footnote{KBO is further parameterized by symbol weights.}
% but our reference implementation in \Vampire{}~\cite{DBLP:conf/cav/KovacsV13} 
% uses for efficiency reasons only weights equal to one \cite{DBLP:conf/cade/KovacsMV11} and so we do not consider this parameter here.}
While the superposition calculus is refutationally complete for any simplification ordering \cite{DBLP:journals/logcom/BachmairG94},
the choice of the precedence has a significant impact on how long it takes to solve a given problem.

It is well known that giving the highest precedence to the predicate symbols introduced as sub-formula names 
during clausification \cite{DBLP:books/el/RV01/NonnengartW01}
% during the Tseitin transformation % of the input formula \cite{Tseitin1983} 
can immediately make the saturation produce the exponential 
set of clauses that the transformation is designed to avoid \cite{Reger2016}.
Also, certain orderings help to make the superposition a decision procedure on specific fragments of first-order logic 
(see, e.g., \cite{DBLP:conf/lics/GanzingerN99,DBLP:conf/cade/HustadtKS05}).
However, the precise way by which the choice of a precedence 
influences the follow-up proof search on a general problem is extremely hard to predict. % indirect

% neslo by rict na tvrdo, ze je to takovy ten mytus o tom, ze se uzivatel zamysli a zvoli si tu skvelou precedenci pro svuj problem?

Several general-purpose precedence generating schemes are available to ATP users,
such as the successful \texttt{invfreq} scheme in E \cite{E-manual}, which orders the symbols 
by the number of occurrences in the input problem. However, experiments with random precedences
indicate that the existing schemes often fail to come close to the optimum precedence \cite{RegerSuda2017},
suggesting room for further improvements.

In this work, we propose a \acrlong{ml} system that learns to predict for an ATP
whether one precedence will lead to a faster proof search on a given problem than another.
Given a previously unseen problem, it can then be asked to recommend the best possible precedence for an ATP to run with.
Relying only on the logical structure of the problems, % for the learning, 
the system generalizes the knowledge about favorable precedences across problems with different signatures.

Our recommender uses a relational graph convolutional neural network \cite{Schlichtkrull2017}
to represent the problem structure. It learns from the ATP performance on selected problems
and pairs of randomly sampled precedences. This information is used to train
a \emph{symbol cost model}, which then realizes the recommendation by simply sorting 
the problem's symbols according to the obtained costs. 

This work strictly improves on our previous experiments with linear regression models and simple hand-crafted symbol features \cite{DBLP:conf/cade/Bartek020}
and is, to the best of our knowledge, the first method able to propose good symbol precedences automatically 
using a non-linear transformation of the input problem structure.

The rest of this paper is organized as follows.
\Cref{sec:preliminaries} exposes the basic terminology used throughout the remaining sections.
\Cref{sec:architecture} proposes a structure of the precedence recommender that can be trained on pairs of symbol precedences,
as described in \cref{sec:training}.
\Cref{sec:evaluation} summarizes and discusses experiments performed
using an implementation of the precedence recommender.
\Cref{sec:related} compares the system proposed in this work with notable related works.
\Cref{sec:conclusion} concludes the investigation and outlines possible directions for future research.

%\todo{MS: idea maybe to work out more in the introduction: 
%We could stress how the impact or the choice of a precedence is \emph{indirect}, 
%(as it's already obvious from the explanation here)
%all the more interesting one can learn from observing just this indirect impact 
%which precendences are good and which are bad!}

% Two aspects: 
% -selection of \emph{maximals} 
% -rewriting from \emph{large to small}


\section{Preliminaries}
\label{sec:preliminaries}
% Terminology

\todo[inline,author=FB]{Get inspiration from PAAR paper.}

\subsection{Saturation-based theorem proving}

A \emph{\acrfull{fol} problem} consists of a set of premise formulas and a conjecture formula.
In a \emph{refutation-based} \emph{\acrfull{atp}},
proving that the premises entail the conjecture
is reduced to proving that the premises together with the negated conjecture entail a \emph{contradiction}.
The most popular \gls{fol} \glspl{atp}, such as Vampire \cite{10.1007/978-3-642-39799-8_1}, E \cite{10.1007/978-3-030-29436-6_29}, or SPASS \cite{},
start the proof search by converting the input \gls{fol} formulas to an equisatisfiable representation in \emph{quantifier-free \gls{cnf}}.\cite{Harrison2009}
We denote a problem in \gls{cnf} as $P = (\Sigma, \mathrm{Cl})$,
where $\Sigma$ is a list of all non-logical symbols in the problem,
and $\mathrm{Cl}$ is the set of the clauses of the problem (including the negated conjecture).

Given a problem $P$ in \gls{cnf},
a \emph{saturation-based} \gls{atp} searches for a refutational proof
by iteratively applying the \emph{inference rules} from the given \emph{calculus}
to infer new clauses entailed by $\mathrm{Cl}$.
As soon as the empty clause, denoted by $\square$, is inferred,
the prover concludes that the premises entail the conjecture,
the sequence of inferences taken constituting a proof.
In case the premises do not entail the conjecture,
the proof search continues until
the set of the inferred clauses is saturated with respect to the inference rules.
In the common setting of time-restricted proof search, a timeout may end the process prematurely.

\todo[inline,author=FB]{Add terminology: saturation iteration, $\bot$.}

%\vampire{} is a saturation-based prover.
%Given a \gls{fol} problem,
%\vampire{} converts the problem into \gls{cnf}.
%and proceeds to saturate the set of clauses
%by repeated application of inference rules.

\subsection{Superposition calculus}

\todo[inline,author=FB]{Mention: Vampire, superposition calculus, inference rules (resolution, superposition), simplification ordering on terms, KBO, symbol precedence. How does the STO influence the rules? Mention transfinite KBO (lcm predicate), which is only relevant to literal selection. Cite Voronkov: transfinite KBO (see section Notes on implementation). We use TKBO to increase the impact of predicate precedence. Vampire sets all KBO weights to 1.}

%\vampire{} searches for a proof by applying the rules
%of the superposition inference system
%\cite{10.1007/978-3-642-39799-8_1}.

\subsubsection{\Gls{sot}}

% AWR candidate
Simplification ordering on terms
influences the proof search in Vampire on two levels.
First, the inferences on each clause are limited
to the selected literals.
\todo{Cite Bachmair Ganzinger or Handbook of AR, chapter 2. For literal selection. Inspiration: Selecting the selection.}
In each clause,
either a negative literal or all the maximal literals are selected.
The maximality is evaluated
according to the simplification ordering.
% Note: The selection function Total does not use the simplification ordering.
Second, the simplification ordering orients some of the equalities
to prevent superposition and equality factoring
from inferring redundant complex conclusions.
In each of these two roles,
the simplification ordering may impact the direction and,
in effect, the length of the proof search.

\subsubsection{\Acrfull{kbo}}

% Example: It seems that a classical example for KBO an orienting equations is group theory axioms.

\subsubsection{Symbol precedences}

\todo[inline,author=FB]{Define terminology.}

\subsection{\Acrlongpl{gcn}}

\todo[inline,author=FB]{GCN? Backprop?, gradient descent, loss, gradient, epoch}


\section{Architecture}
\label{sec:architecture}
% !TEX root = main.tex

\todo{Advantage of using pairs of precedences: we need to solve classification rather than regression.}
\todo{Intuice: snazime se nastavit costy tak, aby co nejvic dvojic dopadlo spravne.}
\todo{Some pairs of precedences are not informative. We hope that the non-systematic examples will cancel each other out.}

\subsection{System overview}

The precedence recommender is a system that takes
a \gls{cnf} problem $P = (\Sigma, \mathit{Cl})$ as the input,
and produces a precedence $\pi^*$ over the symbols $\Sigma$ as the output.
For the recommender to be useful, it needs to produce a precedence
that solves the input problem in a timely fashion.

The recommender described in this section
first computes a cost value for each symbol of the input problem,
and then orders the symbols by their costs in a non-increasing order.
In this manner, the task of finding good precedences is reduced to the task
of finding a good symbol cost function.
Note that the symbol precedence heuristics commonly used in the \glspl{atp}
conform to this general scheme:
they order the symbols by a numeric property, such as arity or number of occurrences.

The architecture is designed so that the recommender can be trained
on the results of \gls{atp} executions on various problems with random precedences.
Since the recommender contains a neural network,
it is parameterized by weight tensors
whose values can be trained by gradient descent.
\todo[inline]{MS: chapu ucel (a vidim, ze dalsi veta navazuje na ``parameters'', ale je to kostrbate.
Recommender = system, chapu jako program. Programy vetsinou neparametrizujeme tenzorama a gradient descent zije na jine urovni,
nez aby parametrizovane programy primo ucil. Uceni je spis nejaka technologie, kterou recommender pouziva, aby ... ``produces precedences
that are likely to yield a successful proof''.  }
The goal of the training is to find parameter values such that the recommender produces precedences
that are likely to yield a successful proof in as few iterations of the saturation loop as possible.

The recommender consists of modules that perform specific subtasks,
each of which is described in detail in one of the following
\iftoggle{LONG}{
sections (see also \cref{fig:architecture}):
\begin{itemize}
\item \Cref{sec:graphifier}: A graph constructor converts the input \gls{cnf} problem into a problem graph.
\item \Cref{sec:gcn}: \Gls{gcn} converts a problem graph into symbol embeddings.
\item \Cref{sec:output}: An output function converts the symbol embeddings into a vector of symbol costs.
\item \Cref{sec:ranking}: Sorting converts the symbol costs into a symbol precedence.
It is only used in generating mode.
\item \Cref{sec:training}: Loss function converts the symbol costs and a pair of precedences into a loss value.
It is only used in training mode.
\end{itemize}
}{
sections (see also \cref{fig:architecture}).
}

\begin{figure}[h]
\caption{Recommender architecture overview.
When recommending a precedence, the input is problem $P$ and the output is precedence $\pi^*$.
When training on an example, the input is problem $P$ and precedences $\pi$ and $\rho$,
and the output is the loss value.}
\label{fig:architecture}
\centering
\digraph[scale=0.4]{ArchitectureOverview}{
graph [splines=ortho,ranksep=0.2];
node [shape=box, fontsize=20];
edge [fontsize=20];

Problem [label="CNF problem P"];
Graphifier [style=rounded, label="Graph constructor"];
g [label="Graph"];
GCN [style=rounded, label="GCN"];
SymbolEmbedding [label="Symbol embeddings"];
SymbolCostModel [style=rounded, label="Projection"];
SymbolCost [label="Symbol costs"];
Sort [style=rounded, label="Sort"];
Precedence [label=<Output precedence &pi;*>];

Problem -> Graphifier -> g -> GCN -> SymbolEmbedding -> SymbolCostModel -> SymbolCost -> Sort -> Precedence;

pi [label=<Precedence &pi;>];
rho [label=<Precedence &rho;>];
LossFunction [style=rounded, label="Loss function"];
Loss [label="Loss value"];

pi -> LossFunction;
rho -> LossFunction;
SymbolCost -> LossFunction;
LossFunction -> Loss;
}

\iffalse

\usetikzlibrary{shapes}
\tikzstyle{object} = [rectangle, draw]
\tikzstyle{input} = [ellipse, draw]
\tikzstyle{output} = [ellipse, draw]

\begin{tikzpicture}[node distance = 0.5 and 1, ->]
% https://tex.stackexchange.com/a/332796/202639

\node (problem) [input] {\gls{cnf} problem $P$};
\node (symbol embeddings) [object] [below=of problem] {Symbol embeddings};
\node (symbol costs) [object] [below=of symbol embeddings] {Symbol costs};
\node (symbol precedence) [output] [below=of symbol costs] {Symbol precedence};

\draw (problem) to node [left] {GCN} (symbol embeddings);
\draw (symbol embeddings) to node [left] {MLP} (symbol costs);
\draw (symbol costs) to node [left] {Order symbols by their costs} (symbol precedence);

\node (PrecBetter) [input] [right=of problem] {Precedence $\PrecBetter$};
\node (PrecBetterInv) [object] [below=of PrecBetter] {$\inv{\PrecBetter}$};
\draw (PrecBetter) to node [right] {Invert} (PrecBetterInv);

\node (PrecWorse) [input] [right=of PrecBetter] {Precedence $\PrecWorse$};
\node (PrecWorseInv) [object] [below=of PrecWorse] {$\inv{\PrecWorse}$};
\draw (PrecWorse) to node [right] {Invert} (PrecWorseInv);

\node (PrecDiff) [object] [right=2 of symbol costs] {$\inv{\PrecWorse} - \inv{\PrecBetter}$};
\node (PrecDiffNormalized) [object] [below=of PrecDiff] {Normalized};
\node (PrecPairCost) [object] [below=of PrecDiffNormalized] {Precedence pair cost};
\node (loss) [output] [below=of PrecPairCost] {Loss};

\draw (PrecBetterInv) to (PrecDiff);
\draw (PrecWorseInv) to (PrecDiff);
\draw (PrecDiff) to node [right] {Normalize} (PrecDiffNormalized);
\draw (PrecDiffNormalized) to (PrecPairCost);
\draw (PrecPairCost) to (loss);

\draw (symbol costs) to (PrecPairCost);

\end{tikzpicture}

\fi

\end{figure}
\todo{FB: Improve the diagram. Consider using tikz. See the disabled tikz figure in the source file.}

\subsection{From \gls{cnf} to graphs}
\label{sec:graphifier}

As the first step of the recommender processing pipeline,
the input problem is converted from a \gls{cnf} representation
to a \emph{heterogeneous (directed) graph} \cite{Zhou2018}.
% Zhou uses the term "heterogeneous graph".
%\footnote{Data mining literature often uses the term \gls{hin} \cite{Shi2015} for heterogeneous graphs.
%We prefer to conform to the terminology common in literature studying \glspl{gnn}.}
Each of the nodes of the graph is labeled with a node type,
and each edge is labeled with an edge type,
defining the heterogeneous nature of the graph.
Each node corresponds to one of the elements that constitute the \gls{cnf} formula,
such as a clause, an atom or a predicate symbol.
Each such category of elements corresponds to one node type.
The edges represent the (oriented) relations between the elements,
for example the incidence relation between a clause and one of its (literals') atoms,
or the relation between an atom and its predicate symbol.
$\mathcal{R}$ denotes the set of all relations in the graph.
\Cref{fig:CnfSchema} shows the types of nodes and edges used in our graph representation.
\Cref{fig:GcnExample} shows an example of a graph representation of a simple problem.

\newcommand{\ntype}[1]{\texttt{#1}}
\newcommand{\etype}[1]{\texttt{#1}}
\newcommand{\epos}{\etype{pos}}
\newcommand{\eneg}{\etype{neg}}

\begin{figure}[h]
\caption{CNF network schema}
\label{fig:CnfSchema}
\centering
\tikzstyle{token} = [rectangle, draw]

\begin{tikzpicture}[node distance = 1 and 2, ->]
% https://tex.stackexchange.com/a/332796/202639

% Node and edge types:
% https://docs.google.com/spreadsheets/d/1PCPHEgk6vLxpdpcvB_PGoLx7p4DID6WtvVWy2mDuv4A/edit?usp=sharing

\node (formula) [token] {\ntype{problem}};
% TODO: Consider removing the Formula nodes.
\node (clause) [token, below=of formula] {\ntype{clause}};
\node (atom) [token, below left=of clause] {\ntype{atom}};
\node (equality) [token, below right=of clause] {\ntype{equality}};
\node (predicate) [token, left=of atom] {\ntype{predicate}};
\node (argument) [token, below=of atom] {\ntype{argument}};
\node (term) [token, below=of argument] {\ntype{term}};
\node (function) [token, left=of term] {\ntype{function}};
\node (variable) [token, right=of term] {\ntype{variable}};

\draw (formula) to node [right] {\etype{contains}} (clause);
\draw (clause) to [bend right] node [above] {\epos{}} (atom);
\draw (clause) to [bend left] node [below] {\eneg{}} (atom);
\draw (clause) to [bend left] node [above] {\epos{}} (equality);
\draw (clause) to [bend right] node [below] {\eneg{}} (equality);
\draw (clause) to node [above] {\etype{binds}} (variable);
\draw (atom) to node [above] {\etype{atom\_applies}} (predicate);
\draw (atom) to node [left] {\etype{atom\_has}} (argument);
\draw (equality) to node {\etype{equalizes}} (term);
\draw (equality) to node {\etype{equalizes}} (variable);
\draw (argument) to [bend right] node [left] {\etype{is}} (term);
\draw (argument) to [loop left] node [left] {\etype{precedes}} (argument);
\draw (argument) to node [below] {\etype{is}} (variable);
\draw (term) to node [above] {\etype{term\_applies}} (function);
\draw (term) to [bend right] node [right] {\etype{term\_has}} (argument);

\end{tikzpicture}
\end{figure}
\todo{FB: Clean up the diagram.}

\begin{figure}[h]
\caption{Graph representation of the \gls{cnf} formula $a=b \wedge f(a,b) \neq f(b,b)$.}
\label{fig:GcnExample}
\centering
\digraph[scale=0.3]{GcnExample}{
	node [fontsize=32, shape=record];
	edge [fontsize=32, dir=both, arrowtail=empty];
	problem [label=<problem|a=b &and; f(a,b)&ne;f(b,b)>];
	c0 [label="clause|a=b"];
	c1 [label=<clause|f(a,b)&ne;f(b,b)>];
	problem -> c0;
	problem -> c1;
	t0 [label="equality atom|a=b"];
	t1 [label="equality atom|f(a,b)=f(b,b)"];
	c0 -> t0 [label=" + "];
	c1 -> t1 [label=" &ndash; "];
	ta [label="term|a"];
	tb [label="term|b"];
	ff [label="function|f", style=bold];
	fa [label="function|a", style=bold];
	fb [label="function|b", style=bold];
	tfab [label="term|f(a,b)"];
	tfbb [label="term|f(b,b)"];
	tfab0 [label="argument|1", style=dotted];
	tfab1 [label="argument|2", style=dotted];
	tfbb0 [label="argument|1", style=dotted];
	tfbb1 [label="argument|2", style=dotted];
	t0 -> ta;
	t0 -> tb;
	t1 -> tfab;
	t1 -> tfbb;
	tfab -> ff;
	tfab -> tfab0;
	tfab0 -> tfab1;
	tfab0 -> ta;
	tfab1 -> tb;
	tfbb -> ff;
	tfbb -> tfbb0;
	tfbb0 -> tfbb1;
	tfbb0 -> tb;
	tfbb1 -> tb;
	ta -> fa;
	tb -> fb;
}
\end{figure}

The graph representation has the following properties:
\begin{itemize}
\item Lossless: The original problem can be faithfully reconstructed from the corresponding graph representation
(up to logical equivalence).
\item Signature agnostic: Renaming the symbols and variables in the input problem yields an isomorphic graph.
\item For each relation $r \in \mathcal{R}$, its inverse $\inv{r}$ is also present in the graph,
typically represented by a different edge type.
\todo{MS: tohle jsem nepochopil. Nemuzeme rict, ze (az na nejakou vyjimkou) jsou edges v obou smerech?\\
FB: Chci vyjasnit, ze inverzni edge ma jiny edge type.}
\item A singleton node of type \ntype{problem} is connected to all the clauses of the problem.
\item The polarity of the literals is expressed by the type of the edge (\epos{} or \eneg{})
connecting the respective atom to the clause it occurs in.
\item For every non-equality atom and term, the order of its arguments is captured by a sequence of \ntype{argument} nodes chained by edges \cite{Rawson2020}.
\item The two operands of equality are not ordered.
This reflects the symmetry of equality.
\item Perfect sub-expression sharing:
\todo[inline]{MS: Predpokladam ze dva ruzne ground literaly p(c) tam budou taky jen jendnou?\\
FB: Literaly nemaji nody. Atomy maji nody a polarita literalu je zakodovana v typu hrany, ktery jej spojuje s klauzuli.
MS: O to my, neslo. Slo o to, jestli se sdily ground termy mezi klauzulemi, ted uz dobry! :)}
Redundant atoms and terms share a node representation.
\todo{FB: Cite some text about sub-expression sharing. Perhaps Rawson2020?}
\todo{MS: Ale mezitim se to tu zmenilo a ja prestavam chapat, co je ``Redundant''.}
Note that since each variable is bound by a clause,
ground terms are shared across clauses,
but non-ground terms are only shared within the context of a clause.
\todo{FB: Remove as of little interest?}
\end{itemize}
\todo{FB: Reference appendix if we describe the representation in more detail there.}
\todo[inline]{MS: Filosoficka k ``can be represented''; mozna by se mohlo nekde objevit:
Snazime se do struktury HIN ``otisknout'' celou strukturu CNF (jak jsem rikal,
znamena to, ze zobrazeni je proste), abychom siti nevzali moznost ``vedet o problemu
uplne vsechno'' (az na veci, na kterych nezalezi jako poradi literaly v klauzulich, etc).
Ta filosoficka poznamka je o tom, ze pro dobre uceni tohle vlastne vubec nemusi byt treba,
ze nejaky ``hrubsi otisk'' by mohl stacit tez (a byt treba mensi), ale my nevime, jak takovy zvolit.\\
FB: Zkusim to popsat, kdyz zbyde cas.}
\todo{FB: Mention other encodings that have been proposed by Mirek and Michael.}

\subsection{\Gls{gcn}: From graphs to symbol embeddings}
\label{sec:gcn}

For each symbol in the input problem $P$,
we seek to find a vector representation, i.e., an \emph{embedding},
that captures the symbol's properties that are relevant
for correctly ranking the symbol in the symbol precedences over $P$.

The symbol embeddings are output by a \gls{rgcn} \cite{Schlichtkrull2017},
which is a stack of \emph{graph convolutional layers}.
Each layer consists of a collection of differentiable modules---one module per edge type.
The computation of the \gls{gcn} starts with assigning each node an initial embedding
and then iteratively updates the embeddings by passing them through the convolutional layers.

The initial embedding $h_a^{(0)}$ of a node $a$ is a concatenation of two vectors:
a \emph{feature vector} specific for that node (typically empty)
and a trainable vector shared by all nodes of the same type.
In our particular implementation,
feature vectors are used in nodes that correspond to clauses and symbols.
Each clause node has a feature vector with a one-hot encoding of the role of the clause,
which can be either axiom, assumption, or negated conjecture \cite{TptpSyntax,Sutcliffe2017}.
Each symbol node has a feature vector with two bits of data:
whether the symbol was introduced into the problem during preprocessing,
and whether the symbol appears in a conjecture clause.

One pass through the convolutional layer
updates the node embeddings by passing a message along each of the edges.
For an edge of type $r \in \mathcal{R}$ going from source node $s$ to destination node $d$ at layer $l$,
the message is composed by converting the embedding of the source node $h_s^{(l)}$
using the module associated with the edge type $r$.
In the simple case that the module is a fully connected layer with weight matrix $W_r^{(l)}$ and bias vector $b_r^{(l)}$,
the message is $W_r^{(l)} h_s^{(l)} + b_r^{(l)}$.
% Notation is from Kipf: RGCN paper.
Each message is then divided by the normalization constant
$c_{s,d} = \sqrt{\card{\mathcal{N}_s^r}} \sqrt{\card{\mathcal{N}_d^r}}$ \cite{kipf2017semisupervised},
where $\mathcal{N}_a^r$ is the set of neighbors of node $a$ under the relation $r$.
\todo{FB: Mention that we mean both in- and out-neighbors?}

Once all the messages are computed,
they are aggregated at the destination nodes to form the new node embeddings.
Each node $d$ aggregates all the incoming messages of a given edge type $r$ by summation,
then passes the sum through an activation function $\sigma$ such as the \gls{relu},
and finally aggregates the messages across the edge types by summation,
yielding the new embedding $h_d^{(l+1)}$.

The following formula captures the complete update of embedding of node $d$ by layer $l$:
$$
h_d^{(l+1)} =
\sum_{r \in \mathcal{R}} \sigma \Parentheses{\sum_{s \in \mathcal{N}_d^r} \frac{1}{c_{s,d}} (W_r^{(l)} h_s^{(l)} + b_r^{(l)})}
$$

\todo{FB: Include LayerNorm and Residual if we end up using them.}
%$$
%h_i^{(l+1)} =
%\mathrm{LayerNorm} \Parentheses{h_i^{(l)} + \sum_{r \in \mathcal{R}} \sigma \Parentheses{\sum_{j \in %\mathcal{N}_i^r} \frac{1}{c_{ji}} h_j^{(l)} W_r^{(l)}}}
%$$
% Inspiration: https://ufal.mff.cuni.cz/~straka/courses/npfl114/1920/slides.pdf/npfl114-07.pdf - slide 27 - Transformer
%\todo{FB: Mention inspiration: Transformer. See Straka's slides.}

\todo[inline]{MS: neni uplne jiste, ze ``CADE people'' oceni presny zapis 
strutury site, vcetne prvku jako $\mathrm{LayerNorm}$ ve vzorecku.
Nerikam, ze bychom to meli uplne zatajit, ale popis, ktery je zde, spis
patri do appendixu (ML paperu). Na te siti vlastne nic originalniho neni,
tak neni treba vsechno vysvetlit do detailu. Zajimavejsi tu
potom je az to, jak se problem $P$ ``otiskne'' v topologii grafu. 
To uz originalni je (a soutezi to s representacemi jako ta Mirkova)
a popisujes to pak dal.}
\todo{Compare our GCN to Michael, Mirek etc.}

\subsection{Output layer: From symbol embeddings to symbol costs}
\label{sec:output}
% Terminology: Deep Learning Book

The symbol cost of each symbol is computed by passing the symbol's embedding through a linear output unit,
which is an affine transformation with no activation function.
\todo{FB: Is this clear at this point?}

It is possible to use a more complex output layer in place of the linear unit,
e.g., a feedforward network with one or more hidden layers.
Our experiments showed no significant improvement when a hidden layer was added,
likely because the underlying \gls{gcn} learns a sufficiently complex transformation.

Let $\theta$ denote the vector of all the parameters of the whole \acrlong{nn} consisting of the \gls{gcn} and the output unit.
Given an input problem $P$ with signature $\Sigma = (s_1, \ldots, s_n)$,
we denote the cost of symbol $s_i$ predicted by the \acrlong{nn} as $c(i, P; \theta)$.
In the rest of this text,
we refer to the predicted cost of $s_i$ simply as $c(i)$
because the problem $P$ and the parameters $\theta$ are fixed in each respective context.

\subsection{Sorting: From symbol costs to precedence}
\label{sec:ranking}

\subsubsection{Precedence cost}
We extend the notion of cost from symbols to precedences
by summing up the symbol costs
weighted by their positions in the given precedence $\pi$:
$$
C(\pi) = Z_n \sum_{i=1}^n i \cdot c(\pi(i))% We call the normalization factor Z_n to follow convention from Foundations of Machine Learning, p. 122.
$$
$Z_n = \frac{2}{n(n+1)}$ is a normalization factor
that ensures the commensurability of precedence costs across signature sizes.
More precisely, normalizing by $Z_n$ makes the expected value of the precedence cost
independent of the signature size $n$:
\begin{align*}
\mathbb{E}_\pi [C(\pi)]
&= \mathbb{E}_\pi \SquareBracket{Z_n \sum_{i=1}^n i \cdot c(\pi(i))} \\
&= Z_n \sum_{i=1}^n i \cdot \mathbb{E}_{\pi, i} [c(\pi(i))] \\
&= Z_n \Parentheses{\sum_{i=1}^n i} \mathbb{E}_i [c(i)] \\
&= \frac{2}{n(n+1)} \frac{n(n+1)}{2} \mathbb{E}_i [c(i)] \\
&= \mathbb{E}_i [c(i)]
\end{align*}

\todo[inline]{FB: Consider accessing elements $\pi$ as vector elements $\pi_i$ instead of $\pi(i)$. Keep in mind that $\inv{\pi}$ must be updated as well. The motivation for using $\pi(i)$ is that $\inv{\pi}(i)$ looks clearer than $\inv{\pi}_i$. How about using $(\inv{\pi})_i$? We may also omit the inversion completely.}

\subsubsection{Cost minimization by sorting}
\begin{lemma}
Precedence cost $C$ is minimized by the precedence that sorts the symbols by their costs in non-increasing order:
$$
\argmin_\pi C(\pi) \in \argsort^- (c(1), \ldots, c(n))
$$
where $\argsort^-(x)$ is the set of permutations $\rho$ that sort vector $x$ in non-increasing order ($x_{\rho(1)} \geq x_{\rho(2)} \geq \ldots \geq x_{\rho(n)}$).
\end{lemma}

\begin{proof}
We prove the lemma by contradiction.
Let $\pi$ minimize $C$ and let $\pi$ not sort the costs in non-increasing order.
Then there exist $k < l$ such that $c(\pi(k)) < c(\pi(l))$.
Let $\bar{\pi}$ be a precedence obtained from $\pi$ by swapping the elements $k$ and $l$.
Then we obtain
%$\bar{\pi} = (\pi(1), \ldots, \pi(k-1), \pi(l), \pi(k+1), \ldots, \pi(l-1), \pi(k), \pi(l+1), \ldots, \pi(n))$.
\begin{align*}
\frac{C(\bar{\pi}) - C(\pi)}{Z_n}
&= kc(\bar{\pi}(k)) + lc(\bar{\pi}(l)) - kc(\pi(k)) - lc(\pi(l)) \\
&= kc(\pi(l)) + lc(\pi(k)) - kc(\pi(k)) - lc(\pi(l)) \\
&= k(c(\pi(l)) - c(\pi(k))) - l(c(\pi(l)) - c(\pi(k))) \\
&= (k-l) (c(\pi(l)) - c(\pi(k))) \\
&< 0
\end{align*}
\todo{FB: Isn't the inclusion of $Z_n$ too confusing?}
The final inequality is because $k-l < 0$ and $c(\pi(l)) - c(\pi(k)) > 0$.
Clearly, $Z_n > 0$ for any $n \geq 0$.
Thus, $C(\bar{\pi}) < C(\pi)$, which is contradicts the assumption that $\pi$ minimizes $C$. \qed
\end{proof}

Once the symbol costs are known,
producing a precedence that minimizes the precedence cost is cheap
because sorting is a fast operation.

Note that a precedence that minimizes $C$
orders the symbols with the lowest cost as the last, prioritizing them for early inferences.

\todo[inline]{FB: Why is the equality predicate always the first in the precedence?
MS: Andrei Voronokov reasons. It's typically better in practice to postpone equational reasoning,
so we want equalities to be generally small.}
\todo[inline]{FB: Mention that the common heuristics fit this scheme.}

\subsection{Training: Classification of pairs of precedences}
\label{sec:training}

In \cref{sec:graphifier,sec:gcn,sec:output,sec:ranking} we described the structure of a recommender system that generates a precedence for an arbitrary input problem.
The efficacy of the recommender depends on the quality of the underlying symbol cost function.
In theory, the symbol cost function can assign the costs so that
sorting the symbols by their costs yields an optimum precedence.
Furthermore, all the information necessary to determine the optimum precedence is present in the graph representation of the input problem
thanks to the lossless property of the graph encoding mentioned before.
\todo{MS: Hlavne tady ale jakoby chybi druha pulky ty vety. Neco jako ``In practice, however, ...'' Nemyslis?} 
Our approach to defining an appropriate symbol cost function is based on statistical learning
from executions of an \gls{atp} on a set of problems with random precedences.

Our ultimate goal is to train the precedence cost function $C$ so that it is minimized by the best precedence,
measuring the quality of a precedence by the number of iterations of the saturation loop taken to solve the problem.
Since the number of possible precedences grows with the factorial of the signature size,
it is not feasible to determine the best precedence conclusively on most practical problems.
Instead of learning to classify one or more precedences on each problem as the best,
we train the system to decide which of a pair of arbitrary precedences is better.
We do this by training $C$ so that better precedences are assigned lower costs.
The motivation for learning to order pairs of precedences
is that it may allow the system to generalize to precedences that are better than any of those seen during training.
\todo{FB: We should test this hypothesis by training on one problem in isolation.}
\todo{FB: Argue better.}

\subsubsection{Training data}

Each training example has the form $(P, \PrecBetter, \PrecWorse)$,
where $P = (\Sigma, \mathit{Cl})$ is a problem
and $\PrecBetter, \PrecWorse$ are precedences over $\Sigma$
such that the prover using $\PrecBetter$ solves $P$ in fewer iterations of the saturation loop than with $\PrecWorse$,
denoted as $\Better{\PrecBetter}{\PrecWorse}{P}$.
\todo{FB: Is this sufficient?}
\todo{FB: Describe the distribution of problems and precedences.}

\subsubsection{Loss function}

Let $(P, \PrecBetter, \PrecWorse)$ be a training example ($\Better{\PrecBetter}{\PrecWorse}{P}$).
The precedence cost classifies this example correctly if $C(\PrecBetter) < C(\PrecWorse)$,
or alternatively $S(\PrecBetter, \PrecWorse) = C(\PrecWorse) - C(\PrecBetter) > 0$.
% We have positive S score for correctly predicted examples.
% The loss function transorms that into a loss value close to 0.
We approach this problem as an instance of binary classification with the logistic loss \cite{Mohri2018},
% p. 128
a loss function routinely used in classification tasks in \acrlong{ml}:
\todo[inline]{MS: First mention of a ``binary classifier''!
We classify the examples, sure, but you did not say it yet.\\
FB: Is it ok to mention binary classification for the first time here if we cite \cite{Mohri2018} immediately?}
\todo[inline]{MS: [...] pisem
``binary'' a ``classification'' a neni tu jasny, co je ``binary''
a co jsou ``classes''. Z pohledu ctenare, ktery o tom zatim neslysel, se misto ``binary classficiation model'' mohlo
psat ``housenka'' a vyslo by to pro nej na stejno.}
\begin{align*}
\loss(P, \PrecBetter, \PrecWorse)
&= - \log \sigmoid S(\PrecBetter, \PrecWorse) \\
&= - \log \sigmoid (C(\PrecWorse) - C(\PrecBetter)) \\
&= - \log \sigmoid Z_n \sum_{i=1}^n i (c(\PrecWorse(i)) - c(\PrecBetter(i)))
%&= - \log \sigmoid Z_n \sum_{i=1}^n c(i) (\inv{\PrecWorse}(i) - \inv{\PrecBetter}(i))
\end{align*}
% FB: Note: I have included the full expansion to make it visible how the loss depends on $c$.
% To a skilled NN practitioner, it should now be obvious that the loss is differentiable w.r.t. $c$.

Note that the classifier cannot simply train $S$ to output a positive number on all pairs of precedences
because $S$ is defined as a difference of two precedence costs.
Intuitively, by training on the example $(P, \PrecBetter, \PrecWorse)$
we are pushing $C(\PrecBetter)$ down and $C(\PrecWorse)$ up.

The loss function is clearly differentiable with respect to the symbol costs,
\todo{Shall we include the derivative?}
and the symbol cost function $c$ is differentiable with respect to its trainable parameters.
This enables the use of gradient descent to find the values of the parameters of $c$
that minimize the loss value.


\section{Experimental evaluation}
\label{sec:evaluation}
% !TEX root = main.tex

To demonstrate the capacity of the trainable precedence recommender described in \cref{sec:architecture},
we performed a series of experiments.
In this section, we describe the design and configuration of the experiments,
and then compare the performance of several trained models to a number of baseline heuristics.

\subsection{Environment}

\subsubsection{System}

All the experiments were run on a computer with the following specification:

\begin{itemize}
\item CPU: Intel Xeon Gold 6140 (72 cores @ \SI{2.30}{GHz})
\item RAM: \SI{383}{GiB}
\item OS: Ubuntu 20.04
\item Kernel: GNU/Linux 5.4.0-40-generic x86\_64
\end{itemize}

% From the paper Reliable benchmarking [BenchExec]:
% – CPU model and size of RAM,
% – specified resource limits,
% – name and version of OS,
% – version of important software packages, such as the kernel or runtime environments like the Java VM,
% – version and configuration of the benchmarked tool(s), and
% – version of the benchmark set (input files)

\subsubsection{Solver}

The empirical evaluation was performed using a modified version of the \gls{atp} Vampire 4.3.0 \cite{10.1007/978-3-642-39799-8_1}.
The prover was used to generate the training data and to evaluate the trained precedence recommender.
To generate the training data,
Vampire was modified to output \gls{cnf} representations of the problems
and annotated problem signatures in a machine-readable format.
%in \gls{json} format and annotated problem signatures in \gls{csv} format.\todo{MS: Mentioning JSON or CSV seems to be too low level. ``programmer documatation'' material.}
For the evaluation of the precedences generated by the recommender,
Vampire was modified to allow the user to supply explicit predicate and function symbol precedences for the proof search
(normally, the user only picks a precedence generation heuristic).
\todo[inline]{FB: Add a link to the modified Vampire.}

\todo[inline,author=MS]{Literal comparison mode is not a standard thing. Mentioning the option name is again too low level. Explaining the effect (point two) is appropriate, although a bit too vague. To kdyztak opravim ja. 
Popsat presne setup je dulezite, kvuli reprodukovatelnosti, ale ne nutne primo v clanku (budem nejspis stejne dodavat odkaz na nejake reproducibility repo).
Ted zminit AVATAR je nevhodne, pokud nevysvetlis, co to aspon ramcove je. A pokud to budes vysvetlovat, je otazka, proc tim ztracet misto, kdyz to neni bezprostredne relevantni pro dalsi vysvetlovani.}

Vampire was run with a time limit of 10 seconds.
To increase the potential impact of predicate precedences,
we used a simple \gls{tkbo} \cite{Ludwig2007,Kovacs2011}
that compares atoms according to the predicate precedence first,
Using the ordinary \gls{kbo} as a tie-breaker.
Ordinary \gls{kbo} was used to compare terms.
\todo[inline,author=FB]{Describe the detailed configuration in an appendix and reference the appendix here.}

% All explicit options:
% vampire --encode on --statistics full --time_statistics on --proof off --avatar off --saturation_algorithm discount --age_weight_ratio 10 --literal_comparison_mode predicate --symbol_precedence frequency --time_limit 10

% Notable explicit options:
% vampire --avatar off --saturation_algorithm discount --age_weight_ratio 10 --literal_comparison_mode predicate --symbol_precedence frequency --time_limit 10

% Notable implicit options:
% --term_ordering kbo

\subsection{Training data}

The training data consists of examples of the form $(P, \PrecBetter, \PrecWorse)$,
where $P$ is a \gls{cnf} problem and $\PrecBetter, \PrecWorse$ are precedences of symbols of problem $P$
such that out of the two precedences, $\PrecBetter$ yields a proof in fewer iterations of the saturation loop (see \cref{sec:saturation}).
\todo{FB: Justify saturation loop iterations as a proxy for success. MS: Ano, ale driv nez zde!}

Since the \gls{tkbo} never compares a predicate symbol with a function symbol,
\todo{FB: Consider referencing an earlier section that explains KBO or TKBO.}
two separate precedences can be considered for each problem:
a predicate precedence and a function precedence.
We trained a predicate precedence recommender separately from a function precedence recommender
to simplify the training process and to isolate the effects of the predicate and function precedences.
This section describes how the training data for training a \emph{predicate} precedence recommender was generated.
Training data for training a function precedence recommender was generated analogously.

\subsubsection{Problem set}

\todo{MS: mozna bych dal propagoval cinny rod a minuly cas ...}

The examples are generated by sampling the \gls{fol} part of the problem library \gls{tptp} v7.4.0 \cite{10.1007/978-3-030-29436-6_29}.
A total of \num{17053} problems formulated in \gls{fof} and \gls{cnf} are sampled.
The problems formulated in \gls{fof} are converted to \gls{cnf} by Vampire.
% vampire --mode clausify
Let $\ProblemsTptp$ denote the set of all \num{17053} \gls{cnf} problems available for training and evaluation.

\subsubsection{Sampling}

The examples are generated by iterative sampling of $\ProblemsTptp$.
In each iteration, a problem $P \in \ProblemsTptp$ is chosen and Vampire is executed twice on $P$
with two (uniformly) random predicate precedences and a common random function precedence.
The ``background'' random function precedence serves as additional noise (in addition to the variability 
contained in \gls{tptp}) and makes sure that the predicate precedence recommender
will not be able to rely on any specificities that would come from fixing function precedences in the training data.

The two executions are compared in terms of performance:
predicate precedence $\PrecBetter$ is recognized as better than predicate precedence $\PrecWorse$,
denoted as $\Better{\PrecBetter}{\PrecWorse}$,
if the proof search finishes successfully with $\PrecBetter$ ($\SolverRun{P}{\Prec} = \top$)
and if the number of iterations of the saturation loop with $\PrecBetter$ is smaller than with $\PrecWorse$.
\todo{FB: Use notation for saturation loop iterations if the notation was established earlier. 
If failure is denoted by $\infty$, we can simply use $<$.}
If one of the two precedences is recognized as better,
the example $(P, \PrecBetter, \PrecWorse)$ is produced,
where $\PrecBetter$ is the better precedence,
and $\PrecWorse$ is the other precedence.
Otherwise, for example, if Vampire fails on both precedences, we go back to sampling another problem.

To ensure the efficiency of the sampling, the process is interpreted as an instance of Bernoulli multi-armed bandit problem,
\todo{Cite? MS: kdyz bude dobra citace, sem by se urcite sikla:) Pri nejhorsim Sutton \& Barto to jisti.}
with the reward of a trial being 1 in case an example is produced, and 0 otherwise.
Adaptive sampling balances
exploring problems that have been tried relatively scarcely and
exploiting problems that have yielded examples relatively often.
For each problem $P \in \ProblemsTptp$,
the generator keeps track of the number of times the problem has been tried $n_P$
and the number of examples generated from that problem $s_P$.
The ratio of $s_P$ to $n_P$ corresponds to the average reward of problem $P$ observed so far.
The problems are sampled using the allocation strategy \acrshort{ucb1} \cite{Auer2002} with a parallelizing relaxation.
In each iteration, the generator samples the problem $P$ that maximizes
$$
\frac{s_P}{n_P} + \sqrt{\frac{2 \ln n}{n_P}}
$$
where $n = \sum_{P \in \ProblemsTptp} n_P$ is the total number of tries on all problems.
The parallelizing relaxation means that the $s_P$ values are only updated once in \num{1000} iterations,
allowing up to \num{2000} parallel solver executions.
\todo{Explain in detail?}

The sampling runs until a target number of examples is generated---\num{1000000} predicate precedence examples and \num{800000} function precedence examples.
% sftp://cluster.ciirc.cvut.cz/home/bartefil/git/vampire-ml/out/20210108-generate-predicate/questions_generated
For example, while generating \num{1000000} examples for the predicate precedence dataset,
the least explored problem was tried 19 times, and the most exploited problem was tried 504 times.
\num{5349} out of the \num{17053} problems yielded at least one example.

\subsubsection{Split}

The problems in $\ProblemsTptp$ are split roughly in half to form the train set and the validation set.
Both training and validation sets are restricted to problems whose graph representation consists of at most \num{100000} nodes
to limit the memory requirements of the training.
The train set is further restricted to problems that correspond to at least one training example.
In total there are \num{7648} problems in the validation set $\ProblemsVal$
and \num{2571} problems in the train set $\ProblemsTrain$.
\todo{MS: A kdyz pouzivas validation set behem uceni, ma taky mensi podmnozinu s limitem \num{100000} nodes?}
\todo{FB: Unify the terminology: validation, or train set?
MS: Co by znamenalo unify? Nebo myslis test set?}
% /home/filip/projects/vampire-ml/vampire-ml/outputs/2021-02-11/16-27-02/results.csv
% all: 17053
% train: 8527
% val: 8526
% with_questions: 5349
% graphified (not all attempted): 10219
% train&with_questions: 2647
% train&with_questions&graphified: 2571
% val&graphified: 7648

\subsection{Model}

We used a \gls{gcn} as described in \cref{sec:architecture}
with depth 4, message size 16, ReLU activation function,
residual connections and layer normalization.
\todo[inline]{MS: zase pasivum. Tady to urcite bude chtit odkaz na sekci, kde se jasne definuje, co to znamena. Jinak recenzenti byvaji dost alergicti na hyperpametry, ktere se jen tak nahodi na nevysvetli. Minimalne nejaky komentar typu: treba, zkouseli jsme u ruzne dalsi hodnoty, ale dopadalo to podobne. Nebo, pozdeji rekeneme, co se deje, kdyz se tohle meni. Nebo: nemeli jsme cas zjistovat, co se deje, kdyz se to meni, tak berem rozumne hodnoty podle blabla.}

\subsection{Training procedure}

A predicate symbol cost model is trained by gradient backpropagation
\todo{FB: Explain? MS: tohle by melo byt k pochopeni z predchozich sekci.}
on the precedence pair classification task
\todo{FB: Reference a section. MS: Ano!}
using the examples generated from the problems in $\ProblemsTrain$.
\todo{Add references to the sections that describe the architecture.}
To avoid redundant computations, all examples generated from a problem are processed in the same training batch.
Thus, each training batch contains up to \num{128} problems and all the examples generated from these problems.
The symbol cost model is trained using the Adam optimizer \cite{Kingma2014}.
Learning rate starts at \num{1.28e-3}
and is halved each time the train loss stagnates for 10 consecutive epochs.
% tf.keras.callbacks.ReduceLROnPlateau(monitor='loss', factor=0.5, patience=10)

\subsubsection{Sample weighting}
Each of the training examples of problem $P$ contributes to the training with the weight $\frac{1}{s_P}$,
where $s_P$ is the number of examples of problem $P$ in the training set,
so that each problem contributes to the training to the same degree irrespective of the relative numbers of examples.
\todo{FB: Shall we remind that the examples are further normalized by $k(n)$ to make even across signature lengths?}
\todo{MS: asi dobry, pokud to bude jasne receny nejaky predchozi sekci.}

\subsubsection{Metrics}
To analyze the training process,
we tracked the dynamics of three metrics on $\ProblemsTrain$ and $\ProblemsVal$:
loss, accuracy and solver performance.
The solver performance was estimated once per 10 training epochs by running Vampire
on \num{1000} problems of the respective problem set
and counting the number of problems successfully solved within the time limit.
Since the accuracy is a proxy measure of solver performance and the loss is a proxy measure of the accuracy,
we are interested in generalization from loss to accuracy and from both loss and accuracy to solver performance.
Similarly, we are interested in generalization from training set to validation set.

\subsubsection{Termination}
We run the training until the validation accuracy stopped increasing for 100 consecutive epochs.

\subsection{Final evaluation}

After the training finished,
we performed a final evaluation of the most promising intermediate trained model on the whole $\ProblemsVal$.
The model that manifested the best estimated solver performance was chosen as the most promising.
\todo{Watch out: This way we may overfit on the 1000 problems used for sample solver performance evaluation.}

\subsection{Results}

A predicate precedence recommender was trained on approximately \num{500000} examples,
% Half of the 1M predicate examples was used for validation.
and a function precedence recommender was trained on approximately \num{400000} examples.
% Half of the 800k function examples was used for validation.
For each problem $P \in \ProblemsVal$,
a pair of symbol precedences was generated by the respective trained recommender
and Vampire was run using these precedences and a wall clock time limit of 10 seconds.
% Note: We have mentioned the time limit above and here we repeat it for readers who only read Results.
The results are averaged over 5 runs to reduce the effect of noise due to the wall clock time limit.
As a baseline, the performance of Vampire with the \texttt{frequency} precedence heuristic was evaluated
with the same time limit.
For comparison, the two trained recommenders are evaluated separately,
with the predicate precedence recommender using the \texttt{frequency} heuristic to generate the function precedences, and vice versa.

To generate a precedence for a problem,
the recommender first converts the problem to a machine-friendly \gls{cnf} format,
then converts the \gls{cnf} to a graph,
then predicts symbol costs using the \gls{gcn} model
and finally orders the symbols by their costs to produce the precedence.
To simplify the experiment, the time limit is only imposed on the Vampire run
and excludes the time taken by the recommender to generate the precedence.
\todo{MS: and if we don't, this would be the perfect place to summarize
how long the just described procedure took on an average problem,
or how it (roughly) scales with the graph/cnf/tptp\_file size.}
When run with 1 thread on 1000 problems,
the preprocessing took 1 second on average and at most 5 seconds,
increasing linearly with the number of nodes of the graph representation of the problem.
\todo{FB: Measure the times with joint recommender and TF restricted to 1 thread. Fix the numbers once we have a solid measurement. Add a comment with a reference to the data.}

\Cref{tab:results} shows the results.

\begin{table*}
\caption{
Results of the evaluation of various predicate precedence heuristics on $\ProblemsVal$.
Means and standard deviations over 5 runs are reported.
}
\centering
\begin{tabular}{l|ll|ll}

Model & \multicolumn{2}{l}{Successes out of \num{7648}} & Mean success rate & Baseline multiple \\
& Mean & Std & & \\

\hline


\acrshort{gcn} (predicate and function) &&&&\\
% VML-706



\acrshort{gcn} (predicate only) &
% Success rate: mean: 0.513023013, std: 0.000292887
% Evaluation results: https://ui.neptune.ai/filipbartek/vampire-ml/e/VML-553
% Total: validation_solver_eval/all/problems/measured&split&category: 7648
% Difference in success count from baseline: 154 ~ 0.020135983
% Estimated difference from baseline (estimate on 891 problems): 0.021099888
% Checkpoint: outputs/2021-02-06/14-55-41/tf_ckpts/epoch/weights.00079-0.61.tf VML-540 0.511785

\num{3923.6} &
% Success mean: validation_solver_eval/all/success/count/mean: 3923.6

\num{2.24} &
% Success std: validation_solver_eval/all/success/count/std 2.24

\SI{51.30}{\percent} &
% Success rate: 0.513023013

\num[round-mode=places,round-precision=4]{1.040853141} \\


\acrshort{gcn} (function only) &
% Final evaluation: VML-677
% Evaluated checkpoint: outputs/2021-02-16/12-28-14/tf_ckpts/epoch/weights.00289.tf
% Results file: sftp://cluster.ciirc.cvut.cz/home/bartefil/git/vampire-ml/outputs/2021-02-17/12-01-09/solver_eval/symbol_cost/epoch_-1/logs.yaml
% Total: val/all/problems/measured&split&category: 7648

\num{3874.2} &
% Success mean: val/all/success/count/mean: 3874.2

\num[round-mode=places,round-precision=2]{1.8330302779823362} &
% Success std: val/all/success/count/mean: 1.8330302779823362

\SI{50.66}{\percent} &
% Success rate: val/all/success/count/mean: 0.5065638075313807

\num[round-mode=places,round-precision=4]{1.027748302} \\


\texttt{vampire -lcm predicate} &
% Success rate: mean: 0.492887029, std: 0.000401412
% https://ui.neptune.ai/filipbartek/vampire-ml/e/VML-490
% /home/filip/projects/vampire-ml/vampire-ml/outputs/2021-02-09/12-13-43
% Row: 'val&graphified&solver_eval'
% Problems total: 7648
% Success rate: 0.492887029

\num{3769.6} &

\num{3.07} &

\SI{49.29}{\percent} &

\num[round-mode=places,round-precision=4]{1.0} \\


\texttt{vampire -lcm standard} & ? & ? & ? & ? \\

\end{tabular}
\label{tab:results}
\end{table*}
\todo{FB: Add links to Neptune runs?}
\todo{FB: Finalize results for function precedences.}
\todo{FB: Try to evaluate a recommender that uses both predicate and function models.}

\todo{FB: Add discussion.}
\todo{FB: Evaluate on problems larger than 100k nodes.}

The loss is merely a proxy measure of the accuracy,
and the accuracy is merely a proxy measure of the solver performance.
During the training, the validation loss typically started increasing after approximately 50 epochs,
while both the train loss and validation accuracy continued improving.
A natural possible explanation is that ...
\todo{Finish.}
While a higher granularity in the solver evaluation would be necessary to draw a clear conclusion,
the trend in the solver performance seems mostly aligned with the validation accuracy.

\todo[inline,author=FB]{Plot some learning curves?}


\section{Related work}
\label{sec:related}

Our previous text \cite{DBLP:conf/cade/Bartek020} marked the initial investigation of applying techniques of \gls{ml}
to generating good symbol precedences.
The neural recommender presented here uses a \gls{gnn} to model symbol costs,
while \cite{DBLP:conf/cade/Bartek020} used a linear combination of symbol features readily available in the \gls{atp} \Vampire{}.
The \gls{gnn}-based approach yields more performant precedences at the cost of longer training and preprocessing time.

In \cite{Olsak2019} and \cite{Rawson2020}, the authors propose similar \gls{gnn} architectures to solve tasks on \gls{fol} problems.
They use the \glspl{gnn} to solve classification tasks such as premise selection.
While our system is trained on a proxy classification task,
the main task it is evaluated on is the generation of useful precedences.

\todo[inline]{Related problem: Ranking. See Mohri etc.}

\section{Conclusion and future work}
\label{sec:conclusion}

We have described a system that extracts useful symbol precedences from the graph representations of \gls{cnf} problems.
Comparison against a conventional symbol precedence heuristic shows that using a \gls{gcn}
to consider the whole structure of the input problem is beneficial.
\todo{FB: Should we boast that the reduction of ranking to training of differentiable element scores is innovative?}

An analysis of the trained recommender could reveal new insights into how the symbol precedence influences the proof search,
and how a symbol precedence should be chosen in the context of a problem so that it is likely to lead to an efficient proof search.
This might lead to a design of a new precedence heuristic
that focuses relies on aspects of input problems
that have been overlooked so far in the context of generating symbol precedences.

While using the \gls{kbo} as our simplification ordering scheme of choice,
we restricted ourselves to generating the symbol precedences,
leaving the symbol weights fixed to the constant 1 (the default behavior of \Vampire{}).
Learning to recommend the symbol weights in addition to the symbol precedences would be interesting.
\todo{FB: Be more fancy in wording.}

Training both the predicate and the function precedence using a single \gls{gcn} would yield a recommender
that only needs one pass through a \gls{gcn} to generate both predicate and function precedences for a problem.
Moreover, the model would have the possibility to exploit the properties of the function precedence
when generating the predicate precedence, and vice versa.
Finally, higher training data efficiency could be achieved by considering all pairs of measured executions on a problem
in one training batch.
\todo{FB: Confusing?}

A more detailed analysis of the training process might yield insights into how the performance generalizes
from the training set to the validation set
and from proxy metrics, such as the loss and the accuracy, to the final solver performance.
Ultimately, this might reveal modifications that make the training more efficient and the final recommender more performant.

The training examples take the form of pairs of precedences,
where one of the precedences is preferable over the other.
In this work, the amount by which the quality of the precedences differs is ignored.
Taking the exact or estimated numbers of saturation loop iterations into account
might help the recommender to prioritize the more significant gains.

Applying techniques of reinforcement learning to focus the training data on the cases that yield high performance gains
could further improve the recommender performance.

We restricted our experiments to problems whose graph representation does not exceed \num{100000} nodes.
Evaluating the recommender on larger problems would be an interesting test of generalization.
Training on larger problems may be facilitated by employing a lossy graphification.

We performed all experiments with a fixed \gls{atp} \Vampire{} and a fixed configuration
we consider to be a reasonable baseline.
Introducing other prover configurations (namely other \glspl{sot})
or other provers could make the resulting recommender more robust
and possibly help extract more general knowledge about the structure of good symbol precedences.

Finally, the approach outlined in \cref{sec:architecture}
could be adjusted to solve tasks on \gls{fol} formulas other than recommending symbol precedences,
such as premise selection of given clause selection.
On the other hand, we could try to solve other tasks that involve generating permutations of arbitrary length.
\todo{FB: TSP? Probably not because it is too combinatorial.}

\section*{Acknowledgments}
% > Acknowledgements should generally be placed in an unnumbered subsection at the end of the paper.

% https://sgs.cvut.cz/index.php?action=faq
% This work was supported by the Grant Agency of the Czech Technical University in Prague, grant No. SGS...

\todo{MS: zvazit, jestli nerozepsat zvlast Martin, zvlast Filip.
Pripadne se zeptat Hanky, jestli bychom nemeli nejak vypichnout
jen GACR.}

This work was supported by
the Czech Science Foundation project 20-06390Y,
the project RICAIP no. 857306 under the EU-H2020 programme,
and
the Grant Agency of the Czech Technical University in Prague, grant\\
no.~SGS20/215/OHK3/3T/37.


% > LaTeX users should avoid self-defined environments and use the bibliographic style MathPhySci for computer science proceedings.
% > It is not possible to have hyperlinks in references.
\bibliographystyle{splncs04}
\bibliography{main}

%\appendix
%\section{Graph structure}

\begin{itemize}
\item The graph representation of problem $P$ contains exactly one root node of type \ntype{problem}.
\item Each clause is represented by a \ntype{clause} node connected to the root \ntype{problem} node.
\item Each atom is represented by an \ntype{atom} node (in case the atom is not an equality)
or an \ntype{equality} node (in case the atom is an equality).
For each literal occurrence there is an edge connecting the respective atom to the respective clause.
The type of the edge corresponds to the polarity of the literal:
\epos{} for positive literal and \eneg{} for negative literal.
\item Each \ntype{equality} node is connected to two nodes that represent the operands,
each of which is of type \ntype{term} or \ntype{variable}.
The commutativity of the equality operator is reflected by the fact that the operand edges are not ordered.
\item Each \ntype{atom} node is connected to one \ntype{predicate} node that represents the predicate symbol being applied by this atom.
Note that these edges connect the applications of a predicate symbol across the whole problem.
\item Each \ntype{atom} node is connected to zero or more \ntype{argument} nodes that represent the argument positions of the atom.
\item Each pair of \ntype{argument} nodes that correspond to consecutive argument positions is connected by an edge.
\item Each \ntype{argument} node is connected to a node that represents the argument term,
which is either a \ntype{term} node or a \ntype{variable} node.
\item Each \ntype{term} node is connected to one \ntype{function} node that represents the function symbol being applied by this term.
\item For each variable, there is an edge connecting the \ntype{clause} node of the clause that binds the variable to the \ntype{variable} node that represents the variable.
\end{itemize}

\section{Loss derivative}

The loss is differentiable with respect to the symbol costs:
\begin{align*}
\frac{\partial \loss}{\partial c_i}
&= -\sigmoid(-C(\Better{\PrecBetter}{\PrecWorse}{P})) \cdot k(n) \cdot (\inv{\PrecWorse}_i - \inv{\PrecBetter}_i) \\
&= (p(\Better{\PrecBetter}{\PrecWorse}{P}) - 1) \cdot k(n) \cdot (\inv{\PrecWorse}_i - \inv{\PrecBetter}_i)
\end{align*}

This means that it is possible to backpropagate the loss gradient into the symbol cost model. \todo{MS: tohle bude tezky,
ale ackoliv ML people by tohle uz asi chapali, ATP crowd spis bude
potrebovat obsirnejsi vysvetleni.}

\section{Experiment details}

\begin{figure}[h]
\caption{The dependence of preprocessing time on the number of nodes in the problem graph on 1000 random validation problems}
\label{fig:preprocessing}
\centering
% \includesvg[width=\textwidth]{preprocessing}
% Source: https://docs.google.com/spreadsheets/d/1GujYNEtETpC3jk4iyENLjptv8mDmI5f1ZEhbmwtqnPM/edit#gid=394425425&fvid=659935521
% Experiment: VML-715
\end{figure}


\end{document}
