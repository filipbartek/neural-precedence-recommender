\documentclass[runningheads]{llncs}
\usepackage[utf8]{inputenc}
\usepackage[T1]{fontenc}
\usepackage[english]{babel}
\usepackage[binary-units=true,detect-weight=true]{siunitx}

% How to write a paper:
% https://mj.ucw.cz/papers/jakpsat.pdf
% Jones 2016: https://www.microsoft.com/en-us/research/academic-program/write-great-research-paper/
% AWR: https://jazyky.fel.cvut.cz/vyuka/RPP/BE9M04AKP/

% PAAR paper: https://github.com/filipbartek/learning-precedences-from-elementary-symbol-features/releases/download/paar2020%2Fceur-2/Learning_Precedences_from_Simple_Symbol_Features.pdf

\usepackage{amsfonts}
\usepackage{amsmath}
\usepackage{amssymb}
\usepackage{mathtools}
\usepackage{stmaryrd}
%\usepackage[disable]{todonotes}
\usepackage{todonotes}

% Multi-letter identifier
\newcommand{\mli}[1]{\mathit{#1}}

\DeclareMathOperator{\re}{\mathbb{R}}
\DeclareMathOperator{\nat}{\mathbb{N}}
\DeclareMathOperator{\logit}{logit}
\DeclareMathOperator{\sigmoid}{sigmoid}
\DeclareMathOperator*{\argmin}{argmin}
\DeclareMathOperator{\argsort}{argsort}
\newcommand{\inv}[1]{#1^{-1}}
\DeclarePairedDelimiter{\card}{\lvert}{\rvert}
\DeclarePairedDelimiter{\SquareBracket}{[}{]}
\DeclarePairedDelimiter{\Parentheses}{(}{)}
\newcommand{\DotProd}[2]{\left<#1,#2\right>}
\newcommand{\Better}[3]{#1 \prec_{#3} #2}
\newcommand{\Prob}[1]{\mathrm{Prob}(#1)}

\DeclareMathOperator{\symbols}{\Sigma}
\newcommand{\Problems}[1]{\mathcal{P}_{#1}}
\DeclareMathOperator{\cnf}{\Problems{\mathrm{CNF}}}
\DeclareMathOperator{\ProblemsTptp}{\Problems{0}}
\DeclareMathOperator{\ProblemsTrain}{\Problems{\mathrm{train}}}
\DeclareMathOperator{\ProblemsTrainEx}{\ProblemsTrain'}
\DeclareMathOperator{\ProblemsVal}{\Problems{\mathrm{val}}}
\DeclareMathOperator{\ProblemsValEx}{\ProblemsVal'}
\newcommand{\signature}[1]{\Sigma_#1}

% Precedences
\newcommand{\PrecBetter}{\pi}
\newcommand{\PrecWorse}{\rho}

\newcommand{\Vampire}{\textsc{Vampire}}

\newcommand{\loss}{\ell}
% Inspiration: The Elements of Statistical Learning, p. 120

\usepackage{glossaries}

\newacronym{ann}{ANN}{artificial \acrlong{nn}}
\newacronym{atp}{ATP}{automated theorem prover}
\newacronym{casc}{CASC}{CADE ATP System Competition}
% http://www.tptp.org/CASC/
\newacronym{ciirc}{CIIRC}{Czech Institute of Informatics, Robotics and Cybernetics}
\newacronym{cnf}{CNF}{clause normal form}
\newacronym{csv}{CSV}{comma-separated values}
\newacronym{ctu}{CTU}{Czech Technical University in Prague}
\newacronym{dag}{dag}{directed acyclic graph}
\newacronym{fel}{FEL}{Faculty of Electrical Engineering}
\newacronym{fof}{FOF}{first-order form}
% http://www.tptp.org/TPTP/TR/TPTPTR.shtml
\newacronym{fol}{FOL}{first-order logic}
\newacronym{gcn}{GCN}{graph convolutional network}
\newacronym{gnn}{GNN}{graph \acrlong{nn}}
\newacronym{hin}{HIN}{heterogeneous information network}
\newacronym{json}{JSON}{JavaScript Object Notation}
\newacronym{kbo}{KBO}{Knuth-Bendix ordering}
\newacronym{lpo}{LPO}{lexicographic path ordering}
\newacronym{ml}{ML}{machine learning}
\newacronym{nn}{NN}{neural network}
\newacronym{relu}{ReLU}{Rectified Linear Unit}
\newacronym{rgcn}{R-GCN}{relational \acrlong{gcn}}
\newacronym{tkbo}{TKBO}{transfinite \acrlong{kbo}}
\newacronym{tptp}{TPTP}{Thousands of Problems for Theorem Provers}
% http://www.tptp.org/
\newacronym{ucb}{UCB}{Upper Confidence Bound}
\newacronym{ucb1}{UCB1}{\gls{ucb}1}

\newglossaryentry{sot}{
name={simplification ordering on terms},
description={},
plural={simplification orderings on terms}
}

% !TEX root = main.tex

\title{Neural Precedence Recommender}
% TODO: Consider prepending an article: "A Neural Precedence Recommender"

% > The corresponding author, i.e., the author responsible for checking the final proof and for signing the copyright form on behalf of all of the authors, should be clearly marked in the header of the paper.
% > The inclusion of the corresponding author’s email address is  mandatory.

% > We strongly recommend that all authors include their email addresses in their papers.

\author{
Filip Bártek\inst{1,2}\orcidID{0000-0002-1822-2651} \and
Martin Suda\inst{1}\orcidID{0000-0003-0989-5800}
}

\authorrunning{F. Bártek \and M. Suda}

% > The affiliated institutions, including town/city and country
\institute{
\acrlong{ciirc}\\
\acrlong{ctu}\\
Jugoslávských partyzánů 1580/3, 160 00 Praha 6 -- Dejvice, Czech Republic\\
\email{\{filip.bartek,martin.suda\}@cvut.cz}\\
% TODO: Consider removing Martin's email.
% TODO: Consider changing Martin's email to martin.suda@gmail.com.
%\url{https://www.ciirc.cvut.cz/}
\and
\acrlong{fel}\\
\acrlong{ctu}\\
Technická 2, 166 27 Praha 6 -- Dejvice, Czech Republic\\
%\url{http://www.fel.cvut.cz/}
}


\usepackage[pdf]{graphviz}
\usepackage{tikz}

\begin{document}

\maketitle

\begin{abstract}
% !TEX root = main.tex

The state-of-the-art superposition-based theorem provers for \acrlong{fol}
rely on \glspl{sot} to constrain the applicability of inference rules,
which in turn shapes the ensuing search space.
The popular Knuth-Bendix simplification ordering is parameterized by 
a \emph{symbol precedence}---a permutation of the predicate and function symbols
of the input problem's signature.
Thus, the choice of a precedence has an indirect yet often substantial impact
on the amount of work required to successfully complete a proof search.

This paper describes and evaluates a symbol precedence recommender,
a \acrlong{ml} system that estimates the best possible precedence
based on observations of prover performance on a set of problems and random precedences.
Using the graph convolutional neural network technology,
the system does not presuppose the problems to be related or share a common signature. 
When coupled with the theorem prover \Vampire{} and evaluated on the \acrshort{tptp} problem library,
the recommender is found to outperform a state-of-the-art heuristic by more than \SI{4}{\percent}
on unseen problems.

\keywords{saturation-based theorem proving \and
simplification ordering \and
symbol precedence \and
\acrlong{ml} \and
\acrlong{gcn}}

% AWR: Guidelines (Unit 7, page 5):
% At most 300 words
% Content:
% Purpose of the study
% Research problems
% Basic design of the study
% Summary of interpretations and conclusions

% AWR: Parts of an abstract (checklist) (Unit 7, page 6):
% Motivation
% Problem statement
% Approach
% Results
% Conclusions

% Motivation and purpose of the study are excluded because the audience at CADE need not be reminded of the motivation for making the provers faster.

\end{abstract}

\section{Introduction}

% !TEX root = main.tex

% General motivation for FOL ATPing:
% Vampire is used as a sledgehammer in Isabelle/HOL:
% 1. https://people.mpi-inf.mpg.de/~jblanche/life.pdf
% 2. https://www.cl.cam.ac.uk/~lp15/papers/Automation/paar.pdf
% Formal methods for verification (survey): https://arxiv.org/pdf/1912.03028.pdf

% Jones 2016 guidelines:
% 1. Problem (explain by example)
% 2. Contributions (refutable, forward references to sections)

% AWR structure (Unit 7, page 1):
% 1. Attention-getter (lead-in) [1-2 sentences]
% 2. Set up for the thesis [minimum: 2-3 sentences]
% 3. Thesis statement (essay map) [1 sentence]

% AWR guidelines (Unit 7, page 5):
% Whet the reader's appetite
% Set the context
% State why the main idea is important
% State your thesis/claim

Modern saturation-based Automatic Theorem Provers (ATPs) such as E \cite{Schulz2019}, SPASS \cite{DBLP:conf/cade/WeidenbachDFKSW09},
or \Vampire{} \cite{DBLP:conf/cav/KovacsV13}
employ the superposition calculus \cite{DBLP:journals/logcom/BachmairG94,DBLP:books/el/RV01/NieuwenhuisR01} as their underlying inference system.
Integrating the flavors of resolution \cite{DBLP:books/el/RV01/BachmairG01}, paramodulation \cite{Robinson1983}, and 
the unfailing completion \cite{Bachmair89completionwithout},
superposition is a powerful calculus with
native support for equational reasoning.
The calculus is parameterized by a simplification ordering on terms % and literals,
and uses it to constrain the applicability of inferences, with a significant impact on performance.

Both main classes of simplification orderings used in practice,
the \acrlong*{kbo} \cite{Knuth1983}
and the \acrlong*{lpo} \cite{Kamin1980},
are specified with the help of a 
\emph{symbol precedence}, an ordering on the signature symbols. %.\footnote{KBO is further parameterized by symbol weights.}
% but our reference implementation in \Vampire{}~\cite{DBLP:conf/cav/KovacsV13} 
% uses for efficiency reasons only weights equal to one \cite{DBLP:conf/cade/KovacsMV11} and so we do not consider this parameter here.}
While the superposition calculus is refutationally complete for any simplification ordering \cite{DBLP:journals/logcom/BachmairG94},
the choice of the precedence has a significant impact on how long it takes to solve a given problem.

It is well known that giving the highest precedence to the predicate symbols introduced as sub-formula names 
during clausification \cite{DBLP:books/el/RV01/NonnengartW01}
% during the Tseitin transformation % of the input formula \cite{Tseitin1983} 
can immediately make the saturation produce the exponential 
set of clauses that the transformation is designed to avoid \cite{Reger2016}.
Also, certain orderings help to make the superposition a decision procedure on specific fragments of first-order logic 
(see, e.g., \cite{DBLP:conf/lics/GanzingerN99,DBLP:conf/cade/HustadtKS05}).
However, the precise way by which the choice of a precedence 
influences the follow-up proof search on a general problem is extremely hard to predict. % indirect

% neslo by rict na tvrdo, ze je to takovy ten mytus o tom, ze se uzivatel zamysli a zvoli si tu skvelou precedenci pro svuj problem?

Several general-purpose precedence generating schemes are available to ATP users,
such as the successful \texttt{invfreq} scheme in E \cite{E-manual}, which orders the symbols 
by the number of occurrences in the input problem. However, experiments with random precedences
indicate that the existing schemes often fail to come close to the optimum precedence \cite{RegerSuda2017},
suggesting room for further improvements.

In this work, we propose a \acrlong{ml} system that learns to predict for an ATP
whether one precedence will lead to a faster proof search on a given problem than another.
Given a previously unseen problem, it can then be asked to recommend the best possible precedence for an ATP to run with.
Relying only on the logical structure of the problems, % for the learning, 
the system generalizes the knowledge about favorable precedences across problems with different signatures.

Our recommender uses a relational graph convolutional neural network \cite{Schlichtkrull2017}
to represent the problem structure. It learns from the ATP performance on selected problems
and pairs of randomly sampled precedences. This information is used to train
a \emph{symbol cost model}, which then realizes the recommendation by simply sorting 
the problem's symbols according to the obtained costs. 

This work strictly improves on our previous experiments with linear regression models and simple hand-crafted symbol features \cite{DBLP:conf/cade/Bartek020}
and is, to the best of our knowledge, the first method able to propose good symbol precedences automatically 
using a non-linear transformation of the input problem structure.

The rest of this paper is organized as follows.
\Cref{sec:preliminaries} exposes the basic terminology used throughout the remaining sections.
\Cref{sec:architecture} proposes a structure of the precedence recommender that can be trained on pairs of symbol precedences,
as described in \cref{sec:training}.
\Cref{sec:evaluation} summarizes and discusses experiments performed
using an implementation of the precedence recommender.
\Cref{sec:related} compares the system proposed in this work with notable related works.
\Cref{sec:conclusion} concludes the investigation and outlines possible directions for future research.

%\todo{MS: idea maybe to work out more in the introduction: 
%We could stress how the impact or the choice of a precedence is \emph{indirect}, 
%(as it's already obvious from the explanation here)
%all the more interesting one can learn from observing just this indirect impact 
%which precendences are good and which are bad!}

% Two aspects: 
% -selection of \emph{maximals} 
% -rewriting from \emph{large to small}


\section{Preliminaries}
% Terminology

\todo[inline]{Add: brief exposure of FOL and saturation-based proving with terms possibly emphasized.}
\todo[inline]{If space is left: define KBO.}
\todo[inline]{If space is left: show resolution and superposition rules. Explain how KBO influences them.}
\todo[inline]{Mention lcm predicate. Cite the Vampire article. Voronkov: transfinite KBO - see section Notes on implementation

Mention: Vampire sets all KBO symbol weights to 1.

In Vampire: simple TKBO - level

We use TKBO to increase the impact of predicate precedence.}
\todo[inline]{Get inspiration from PAAR paper.}

\vampire{} \cite{10.1007/978-3-642-39799-8_1}
is a representative
of the state-of-the-art \gls{fol} \gls{atp} design.
\todo{Move into Introduction.}

%\vampire{} is a saturation-based prover.
%Given a \gls{fol} problem,
%\vampire{} converts the problem into \gls{cnf}.
%and proceeds to saturate the set of clauses
%by repeated application of inference rules.

%\vampire{} searches for a proof by applying the rules
%of the superposition inference system
%\cite{10.1007/978-3-642-39799-8_1}.

% AWR candidate
Proof search in Vampire is crucially constrained
by a simplification ordering on terms.
Being refutation-based and saturation-based,
Vampire searches for a contradiction
by iteratively inferring clauses provable from the input clauses.
The inferences 

explores the space of clauses provable
from the input formulas
by applying the rules of superposition inference system.
This process continues until a contradiction is inferred,
the processed clause set is saturated,
or the execution hits a resource limit.

During the proof search,
Vampire explores the space of provable clauses
by applying inference rules to clauses
that have already been proven.

% AWR candidate
Simplification ordering on terms
influences the proof search in Vampire on two levels.
First, the inferences on each clause are limited
to the selected literals.
\todo{Cite Bachmair Ganzinger or Handbook of AR, chapter 2. For literal selection. Inspiration: Selecting the selection.}
In each clause,
either a negative literal or all the maximal literals are selected.
The maximality is evaluated
according to the simplification ordering.
% Note: The selection function Total does not use the simplification ordering.
Second, the simplification ordering orients some of the equalities
to prevent superposition and equality factoring
from inferring redundant complex conclusions.
In each of these two roles,
the simplification ordering may impact the direction and,
in effect, the length of the proof search.

The inference system used in the state-of-the-art \gls{fol} \gls{atp} \vampire{}
is parameterized by a simplification ordering on terms.
\vampire{} uses the superposition inference system \cite{}.

The \gls{atp} \vampire{} uses superposition inference system.


Vampire supports two simplification term ordering classes:
\gls{lpo} and \gls{kbo}.

The superposition inference system \cite{} used in Vampire is parameterized
by symbol precedence.

% Example: It seems that a classical example for KBO an orienting equations is group theory axioms.

The choice of symbol precedence may affect the length of a proof search to a great extent.

Given a \gls{fol} problem,
determining a symbol precedence that leads to a fast proof search is a nontrivial task.

\todo[inline]{Define "CNF" as quantifier-free CNF FOL with equality.}
\todo[inline]{Define "signature".}
% https://en.wikipedia.org/wiki/First-order_logic

\todo[inline]{GCN? Backprop?, gradient descent, loss, gradient}

\section{Architecture}
\label{sec:architecture}

\subsection{Learning to generate good permutations}

Generating good symbol precedences is an instance of a more general task
of generating permutations of arbitrary length.
Each input problem corresponds to an input object that specifies the length of a permutation.

Let $X$ be a set of objects.
Let $l: O \to \nat$ assign each object $x \in X$ its signature length $l(x)$.
Let $\{x_1, \ldots, x_n\}$ be the training samples.\todo{MS: je mi jasny, ze tu neni poradek, ale i tak; posledni dva odstavce jsou dost zmateny.}

\subsubsection{Binary classification}

Let $p_\theta : X \to \re$ be a binary classification model.\todo{MS: Dobry priklad na to, ktere pojmy je treba vysvetlovat. model je znamy pojem v logice, tady ale pouzivame jiny, ML vyznam. To by si zaslouzilo komentar. Navic pisem
``binary'' a ``classification'' a neni tu jasny, co je ``binary''
a co jsou ``classes''. Z pohledu ctenare, ktery o tom zatim neslysel, se misto ``binary classficiation model'' mohlo
psat ``housenka'' a vyslo by to pro nej na stejno.}
The model predicts the probability of label $1$.
Let $(x_i, y_i)$ be the training samples,
where $x_i \in X$ and $y_i \in Y = \{0, 1\}$.\todo{MS:
``training'', ``sample'' a ``loss'' jsou dalsi priklady pojmu,
ktere najednou spadly z hury.}

For a training sample $(x, y) \in \re \times \{0, 1\}$, the binary cross-entropy loss is:
$$
L(\theta) = y \log{f_\theta(x)} + (1-y) \log{(1-f_\theta(x))}
$$

If $y_i = 1$ for all $i$, then the loss can be written in a simpler form:
$$
L(\theta) = \log{f_\theta(x)}
$$

\subsubsection{Permutations}

Let $O$ be a set of objects (or contexts).
Given an object $o \in O$, let $\Sigma(o)$ be its signature.
Let $x_i = (o_i, \pi_i, \rho_i)$, where $o_i \in O$
and $\pi_i, \rho_i \in \Perm(\Sigma(o_i))$.
Let $y_i \in \{0, 1\}$.

Let $p_\theta: X \to Y$.

Loss:
$$
L(\theta) = \log{p_\theta(x)}
$$

Let $c_\theta(o, s)$ be the predicted cost of symbol $s$.
Then:
$$
p_\theta(o, \pi, \rho) = C_\theta(o, \pi) - C_\theta(o, \rho)
$$

$$
C_\theta(o, \pi) = \sum_{i=1}^{l(o)}{c(o, \pi_i)}
$$

\begin{align*}
L(\theta) &= \log{p_\theta(o, \pi, \rho)} \\
&= \log{(C_\theta(o, \pi) - C_\theta(o, \rho))} \\
&= \log{(\sum_{i=1}^{l(o)}{c(o, \pi_i)} - \sum_{i=1}^{l(o)}{c(o, \rho_i)})} \\
&= \log{\sum_{i=1}^{l(o)}{c(o, \pi_i) - c(o, \rho_i)}} \\
\end{align*}

If $c$ is differentiable,
we can propagate loss gradients into it.

\subsection{Notation}

We denote the set of all finite vectors over $\re$ by $\re^* = \bigcup_{n=1}^\infty{\re^n}$.

Dot product: $\DotProd{x}{y} = \sum_{i=1}^n x_i \cdot y_i$ for $x, y \in \re^n$.

For any $n \in \nat$, we denote the set of all permutations over the set $\{1, \ldots, n\}$ by $\Perm{n}$.
Permuting a vector $x = (x_1, \ldots, x_n) \in \re^n$ by a permutation $\pi = (\pi_1, \ldots, \pi_n) \in \Perm{n}$ yields the vector $\pi(x) = (x_{\pi_1}, \ldots, x_{\pi_n})$.
$\inv{\pi}$ denotes the permutation inverse to $\pi$.
$\inv{\pi} = ...$.
\todo{Define permutation inversion.}
\todo{MS: Mozna ani ne. Ono je obcas tezky zavest a pouzivat presnout matematickou notaci, ktera se navic muze stavat neprehlednou, kvuli te presnosti. Na druhou stranu, intuitivne, kazdy vi, co by to mela byt inverzni permutace. Takze nerikam psat nepresne veci, ale obcas jde
zustak vagni s tim, ze vim, ze existuje zpusob jak to udelat presnym.
A povrchni ctenar to nemusi resit a matematicky zdatny by si ten detail odvodil spravne ...}

\newcommand{\Prec}{\pi}
\newcommand{\PrecBetter}{\pi}
\newcommand{\PrecWorse}{\rho}

\subsection{Overview}

Let $\cnf$ be the set of all \gls{cnf} problems.
Let $P \in \cnf$ be an arbitrary problem.
\todo{Consider using $F$ (as formula) instead of $P$ to free up P for probability.}
Let $\signature{P}$ be the signature of $P$, that is a list of all non-logical symbols (predicates and functions)
\todo{Should we rather use either predicates or functions? Or use a compound signature?}
that appear in the problem.
Let $n = \card{\signature{P}}$.
A symbol precedence $\Prec \in \Perm{n}$ for problem $P$ is a permutation of the symbols of $P$.

\newcommand{\Solver}{S}
\newcommand{\SolverRun}[2]{\Solver(#1, #2)}

Let $\Solver$ be an \gls{atp} with a fixed computation environment, configuration and time limit.
The computation of $\Solver$ is assumed to be parameterized by symbol precedence.
\todo{Discuss. What does this mean?}
$\SolverRun{P}{\Prec}$ denotes the result of a proof attempt on problem $P$ with symbol precedence $\Prec$.
$\SolverRun{P}{\Prec} = \top$ if $S$ solves $P$ within the time limit, and $\SolverRun{P}{\Prec} = \bot$ otherwise.
\todo{MS: vadi zminit pocet iteraci smycky tak, jako v minulem paperu?}
Since the result is generally non-deterministic in case the time limit is specified in wall clock units,
we consider $\SolverRun{P}{\Prec}$ to be a random variable with a Bernoulli distribution.
\todo{Cite?}
We denote the probability of a successful proof search $\Prob{\SolverRun{P}{\Prec} = \top}$.
\todo{MS: asi bych se primlouval za to v teto fazi o Bernoullim pomlcet.
Otazky jsme generovali tak, ze proste porovnavame pocty iteraci saturatcni smycky a pocitame s tim, ze beh je deterministicky.
To, ze se pozdeji, pri evalu, pouzije perspektiva nahodnosti
bych uz chapal jako metodiku experimentu a teorii tim nekomplikoval.}

The main task this paper deals with is this:
Given a problem $P \in \cnf$, generate a symbol precedence $\Prec \in \Perm{n}$ that maximizes $\Prob{\SolverRun{P}{\Prec} = \top}$.
MS: (Nehlede na to, ze main task by nejspis mel byt vysvetlen mnohem drive...) Asi bych to takhle nestavel. Jak pisu vyse, precijen je
beh dokazovace deterministicky a brat to nejak Bayesovsky by sice mohlo 
byt zajimave, ale jeste mnohem tezsi na obhajeni jako ``svetonazor''.
Co proste priznat nasledujici assumption:
verime, ze pokud bude model spravne odpovidat na otazky $(P, \pi_1, \pi_2)$ (s pravdepodobnosti $> \SI{50}{\percent}$), povede podle nej vybrana permutace minimalizujici \eqref{eq:model_cost} k rychlemu vyreseni problemu (a tim i idealne k vyreseni problemu, ktere byly drive out of reach.)\todo{MS: koukni prosim, jestli nejaky takovy mame. Tj zadna random permutace ho nevyresila, ale naucena sit pak ano!}

Instead of modeling $\Prob{\SolverRun{P}{\Prec} = \top}$ directly,
the system described in this paper predicts, given a problem $P$ and a pair of precedences $\PrecBetter, \PrecWorse \in \Perm{n}$,
the probability\todo{MS: tohle je jeden z oficialnich zpusobu,
jak vysvetlovat neuronky, ale neni nijak jasne do jake miry byva
$\sigma(\mathit{logits})$ blizko nejake pravdepodobnosti. Za mne by klidne stacilo mluvit o tom, ze se snazime klasifikovat co nejvic dvojic
spravne a ze na to pouzivame standard techologii neuronek (Binary Cross Entropy loss (against the final layer’s sigmoid non-linearity)). Tj opet hlasuji za to nemluvit zde o pravdepodobnosti. :)}
\todo{MS: jo, ale pak bude nekde dobre zminit, ze jsme si vedomi, ze dvojice skoro jiste neobsahuji stejne mnozstvi ``signalu'', ze ktereho je mozne se ucit. Je to proste vlastnost vybrane metody a v prumeru to nevadi.}
that $\PrecBetter$ outperforms $\PrecWorse$ in terms of success of proof search. This proxy task allows learning from pairs of precedences that both yield a successful proof search but differ in the number of steps taken in the proof search.
\todo{Explain more.}

% By taking the pairs, we get more data esp. from easy problems.
% Consider weighting the samples by the result type: quantitative / qualitative etc.

% TODO: Experiments: Increase depth and message size

\begin{figure}[h]
\caption{System overview}
\label{fig:SystemOverview}
\centering
\usetikzlibrary{shapes}
\tikzstyle{object} = [rectangle, draw]
\tikzstyle{input} = [ellipse, draw]
\tikzstyle{output} = [ellipse, draw]

\begin{tikzpicture}[node distance = 0.5 and 1, ->]
% https://tex.stackexchange.com/a/332796/202639

\node (problem) [input] {\gls{cnf} problem $P$};
\node (symbol embeddings) [object] [below=of problem] {Symbol embeddings};
\node (symbol costs) [object] [below=of symbol embeddings] {Symbol costs};
\node (symbol precedence) [output] [below=of symbol costs] {Symbol precedence};

\draw (problem) to node [left] {GCN} (symbol embeddings);
\draw (symbol embeddings) to node [left] {MLP} (symbol costs);
\draw (symbol costs) to node [left] {Order symbols by their costs} (symbol precedence);

\node (PrecBetter) [input] [right=of problem] {Precedence $\PrecBetter$};
\node (PrecBetterInv) [object] [below=of PrecBetter] {$\inv{\PrecBetter}$};
\draw (PrecBetter) to node [right] {Invert} (PrecBetterInv);

\node (PrecWorse) [input] [right=of PrecBetter] {Precedence $\PrecWorse$};
\node (PrecWorseInv) [object] [below=of PrecWorse] {$\inv{\PrecWorse}$};
\draw (PrecWorse) to node [right] {Invert} (PrecWorseInv);

\node (PrecDiff) [object] [right=2 of symbol costs] {$\inv{\PrecWorse} - \inv{\PrecBetter}$};
\node (PrecDiffNormalized) [object] [below=of PrecDiff] {Normalized};
\node (PrecPairCost) [object] [below=of PrecDiffNormalized] {Precedence pair cost};
\node (loss) [output] [below=of PrecPairCost] {Loss};

\draw (PrecBetterInv) to (PrecDiff);
\draw (PrecWorseInv) to (PrecDiff);
\draw (PrecDiff) to node [right] {Normalize} (PrecDiffNormalized);
\draw (PrecDiffNormalized) to (PrecPairCost);
\draw (PrecPairCost) to (loss);

\draw (symbol costs) to (PrecPairCost);

\end{tikzpicture}
\end{figure}

Fig.~\ref{fig:SystemOverview} \todo{MS: Na figures/tables bez referneci v textu se pohlizi obzvlast nelibe. Predpokladam, zes planoval pozdeji popsat, presto zmninuji. Konkretne k Fig.~\ref{fig:SystemOverview}:
Napada me: neslo by precedenci definovat tak, aby uz byla tim $\pi^{-1}$? Invertovani zabira v obrazku skoustu mista, ale pritom
jde o technicky detail a hlavni myslenku nijak neosvetluje...}

\subsection{Symbol cost model}

The symbol costs are modeled by a \gls{rgcn}.\cite{Schlichtkrull2017}
\todo{Can I use the term \gls{rgcn} when I use a modified version? Namely, the activation is called earlier.}
The model is a stack of graph convolutional layers.
Each layer consists of one differentiable module for each edge type.
Each module is a dense layer.
$$
h_i^{(l+1)} =
\mathrm{LayerNorm} \Parentheses{h_i^{(l)} + \sum_{r \in \mathcal{R}} \sigma \Parentheses{\sum_{j \in \mathcal{N}_i^r} \frac{1}{c_{ji}} h_j^{(l)} W_r^{(l)}}}
$$
\todo{Include dropout?}
% Inspiration: https://ufal.mff.cuni.cz/~straka/courses/npfl114/1920/slides.pdf/npfl114-07.pdf - slide 27 - Transformer
$h_i^{(l)}$ denotes the embedding of node $i$ at layer $l$.
$\mathcal{R}$ denotes the set of all relations in the graph.
$\sigma$ denotes the activation function, for example the \gls{relu}.
$\mathcal{N}_i^r$ denotes the set of neighbors of node $i$ under relation $r$.
$c_{j, i, r} = \sqrt{\card{\mathcal{N}_j^r}} \sqrt{\card{\mathcal{N}_i^r}}$ is a normalization constant.\cite{kipf2017semisupervised}
$W_r^{(l)}$ is a trainable weight matrix representing relation $r$ at layer $l$.

Residual...
Layer norm...

\todo{MS: neni uplne jiste, ze ``CADE people'' oceni presny zapis 
strutury site, vcetne prvku jako $\mathrm{LayerNorm}$ ve vzorecku.
Nerikam, ze bychom to meli uplne zatajit, ale popis, ktery je zde, spis
patri do appendixu (ML paperu). Na te siti vlastne nic originalniho neni,
tak neni treba vsechno vysvetlit do detailu. Zajimavejsi tu
potom je az to, jak se problem $P$ ``otiskne'' v topologii grafu. 
To uz originalni je (a soutezi to s representacemi jako ta Mirkova)
a popisujes to pak dal.}

\subsection{Loss function} \todo{MS: tady jsem zatim skoncil...}
\label{sec:loss}

Let $\cnf$ be the set of all \gls{cnf} problems.
Let $P \in \cnf$ be an arbitrary problem.
Let $\signature{P}$ be the signature of $P$.
Let $n = \card{\signature{P}}$.

Given a problem $P$, a symbol cost model $c : \cnf \to \re^*$
\todo{Confusing?}
returns a vector of costs of the symbols of $P$:
$$
c(P) = (c_1, \ldots, c_n)
$$

Let $\Prec \in \Perm{n}$ be a precedence of symbols of problem $P$.
Let $k(n) = \frac{2}{n(n+1)}$ be a normalization factor for precedences of length $n$.
The symbol cost model $c$ can be extended to a precedence cost model $C$
by taking the weighted sum of symbol costs:
$$
C(\Prec | P)
= k(n) \sum_{i=1}^n c_{\Prec_i} \cdot i
= k(n) \sum_{i=1}^n c_i \cdot \inv{\Prec}_i
= k(n) \DotProd{c(P)}{\inv{\Prec}}
$$
Note that the value of $C$ is minimized by a precedence that orders the symbols by their costs in non-increasing order.
\todo{Discuss, prove.}

Fixing the normalization factor $k(n)$ to the value $\frac{2}{n(n+1)}$
ensures that costs of precedences of various lengths are commensurable.
Observe namely that if $c$ assigns each symbol the same cost $c_0$,
then the precedence cost is equal to $c_0$ irrespective of the signature length $n$:
\begin{equation} \label{eq:model_cost}
C(\Prec | P) = \frac{2}{n(n+1)} \sum_{i=1}^n c_0 \cdot i = c_0
\end{equation}

Let $\PrecBetter, \PrecWorse \in \Perm{n}$ be two symbol precedences over $P$.
Let $\PrecBetter$ yield a faster proof search on $P$ than $\PrecWorse$, denoted as $\Better{\PrecBetter}{\PrecWorse}$.
The precedence cost model can be extended to predict the probability that $\PrecBetter$ is better than $\PrecWorse$:
\todo{Fix this. Ensure the polarity is sound.}
\begin{align*}
p(\Better{\PrecBetter}{\PrecWorse} | P)
&= \sigmoid \SquareBracket{C(\PrecWorse | P) - C(\PrecBetter | P)} \\
%&= \sigmoid \SquareBracket{k(n) \sum_{i=1}^n c_i \cdot (\inv{\PrecWorse}_i - \inv{\PrecBetter}_i)} \\
&= \sigmoid \SquareBracket{k(n) \DotProd{c(P)}{\inv{\PrecWorse} - \inv{\PrecBetter}}}
\end{align*}
It is easy to see\todo{MS: Bez znalosti $\sigmoid$ uplne ne ;)} that $p(\Better{\PrecBetter}{\PrecWorse} | P) > 0.5$ if and only if $C(\PrecWorse | P) > C(\PrecBetter | P)$,
and that the value of $p$ can be increased by increasing $C(\PrecWorse | P)$ or decreasing $C(\PrecBetter | P)$.

We consider the difference $C(\PrecWorse | P) - C(\PrecBetter | P)$
to be the "logit"\todo{MS: Logit je taky tezky pojem, urcite ne automaticky znamy u ATP people.} score of the pair of precedences $\PrecBetter, \PrecWorse$,
and denote it by $C(\Better{\PrecBetter}{\PrecWorse} | P)$.

We define the loss of $c$ on the training sample $(\PrecBetter, \PrecWorse, P)$ as binary cross-entropy
with ground truth $\Better{\PrecBetter}{\PrecWorse}$:

\begin{align*}
L(\PrecBetter, \PrecWorse, P)
&= - \log p(\Better{\PrecBetter}{\PrecWorse} | P) \\
&= - \log \sigmoid \SquareBracket{k(n) \DotProd{c(P)}{\inv{\PrecWorse} - \inv{\PrecBetter}}} \\
&= - \log \sigmoid \SquareBracket{k(n) \sum_{i=1}^n c_i \cdot (\inv{\PrecWorse}_i - \inv{\PrecBetter}_i)}
\end{align*}

The loss is differentiable with respect to the symbol costs:
\begin{align*}
\frac{\partial L}{\partial c_i}
&= -\sigmoid(-C(\Better{\PrecBetter}{\PrecWorse} | P)) \cdot k(n) \cdot (\inv{\PrecWorse}_i - \inv{\PrecBetter}_i) \\
&= (p(\Better{\PrecBetter}{\PrecWorse} | P) - 1) \cdot k(n) \cdot (\inv{\PrecWorse}_i - \inv{\PrecBetter}_i)
\end{align*}

This means that it is possible to backpropagate the loss gradient into the symbol cost model. \todo{MS: tohle bude tezky,
ale ackoliv ML people by tohle uz asi chapali, ATP crowd spis bude
potrebovat obsirnejsi vysvetleni.}

\subsubsection{Generalization}

Note that this approach is generally applicable to all problems where it is necessary to generate permutations of arbitrary length.

\todo[inline]{Finish.}

\subsection{Symbol cost model}



\subsection{Graph representation of problems}

\newcommand{\ntype}[1]{\texttt{#1}}
\newcommand{\etype}[1]{\texttt{#1}}
\newcommand{\epos}{\etype{pos}}
\newcommand{\eneg}{\etype{neg}}

Every \gls{cnf} problem can be represented by a \gls{hin} with the network schema shown in \cref{fig:CnfSchema}.\todo{MS: Mel jsem chut to opravit na velke "Fig.", ale s makrem se mi hadat nechce ;)}
% Cite: https://www.kdd.org/exploration_files/V14-02-03-Sun.pdf
% Referenced from: https://docs.dgl.ai/en/0.5.x/generated/dgl.DGLHeteroGraph.metagraph.html

\begin{figure}[ht]
\caption{CNF network schema}
\label{fig:CnfSchema}
\centering
\tikzstyle{token} = [rectangle, draw]

\begin{tikzpicture}[node distance = 1 and 2, ->]
% https://tex.stackexchange.com/a/332796/202639

% Node and edge types:
% https://docs.google.com/spreadsheets/d/1PCPHEgk6vLxpdpcvB_PGoLx7p4DID6WtvVWy2mDuv4A/edit?usp=sharing

\node (formula) [token] {\ntype{problem}};
% TODO: Consider removing the Formula nodes.
\node (clause) [token, below=of formula] {\ntype{clause}};
\node (atom) [token, below left=of clause] {\ntype{atom}};
\node (equality) [token, below right=of clause] {\ntype{equality}};
\node (predicate) [token, left=of atom] {\ntype{predicate}};
\node (argument) [token, below=of atom] {\ntype{argument}};
\node (term) [token, below=of argument] {\ntype{term}};
\node (function) [token, left=of term] {\ntype{function}};
\node (variable) [token, right=of term] {\ntype{variable}};

\draw (formula) to node [right] {\etype{contains}} (clause);
\draw (clause) to [bend right] node [above] {\epos{}} (atom);
\draw (clause) to [bend left] node [below] {\eneg{}} (atom);
\draw (clause) to [bend left] node [above] {\epos{}} (equality);
\draw (clause) to [bend right] node [below] {\eneg{}} (equality);
\draw (clause) to node [above] {\etype{binds}} (variable);
\draw (atom) to node [above] {\etype{atom\_applies}} (predicate);
\draw (atom) to node [left] {\etype{atom\_has}} (argument);
\draw (equality) to node {\etype{equalizes}} (term);
\draw (equality) to node {\etype{equalizes}} (variable);
\draw (argument) to [bend right] node [left] {\etype{is}} (term);
\draw (argument) to [loop left] node [left] {\etype{precedes}} (argument);
\draw (argument) to node [below] {\etype{is}} (variable);
\draw (term) to node [above] {\etype{term\_applies}} (function);
\draw (term) to [bend right] node [right] {\etype{term\_has}} (argument);

\end{tikzpicture}
\end{figure}

The network of a problem is a directed graph $G = (V, E)$.
The function $\tau: V \rightarrow A$ maps the nodes to their types:

$$
A = \{\ntype{problem}, \ntype{predicate}, \ntype{function}, \ntype{clause}, \ntype{atom}, \ntype{equality}, \ntype{term}, \ntype{variable}, \ntype{argument}\}
$$

The function $\phi: E \rightarrow E$ maps the edges to their types.

\begin{itemize}
\item The graph representation of problem $P$ contains exactly one root node of type \ntype{problem}.
\item Each clause is represented by a \ntype{clause} node connected to the root \ntype{problem} node.
\item Each atom is represented by an \ntype{atom} node (in case the atom is not an equality)
or an \ntype{equality} node (in case the atom is an equality).
For each literal occurrence there is an edge connecting the respective atom to the respective clause.
The type of the edge corresponds to the polarity of the literal:
\epos{} for positive literal and \eneg{} for negative literal.
\item Each \ntype{equality} node is connected to two nodes that represent the operands,
each of which is of type \ntype{term} or \ntype{variable}.
The commutativity of the equality operator is reflected by the fact that the operand edges are not ordered.
\item Each \ntype{atom} node is connected to one \ntype{predicate} node that represents the predicate symbol being applied by this atom.
Note that these edges connect the applications of a predicate symbol across the whole problem.
\item Each \ntype{atom} node is connected to zero or more \ntype{argument} nodes that represent the argument positions of the atom.
\item Each pair of \ntype{argument} nodes that correspond to consecutive argument positions is connected by an edge.
\item Each \ntype{argument} node is connected to a node that represents the argument term,
which is either a \ntype{term} node or a \ntype{variable} node.
\item Each \ntype{term} node is connected to one \ntype{function} node that represents the function symbol being applied by this term.
\item For each variable, there is an edge connecting the \ntype{clause} node of the clause that binds the variable to the \ntype{variable} node that represents the variable.
\end{itemize}

\todo[inline]{Abstract from the edge types. Focus on the main idea. Technical details may be in technical report or appendix.}

Various levels of term sharing are possible. [continue]

\todo[inline]{Give an example of a problem and its graph.}

\subsection{Generating precedences}
\label{sec:generating}

When presented with a \gls{cnf} problem $P$ with signature $\symbols$,
the recommender produces a precedence $\pi$ of symbols from $\symbols$.
In order to generate the precedence,
the recommender first assigns a cost value to each symbol
by invoking a symbol cost model (a trained \gls{gcn}) on a graph representation of $P$.
Ordering the symbols by their predicted costs in nondecreasing order yields the precedence $\pi$.

\subsection{Training}

The training is performed using a fixed \acrlong{atp}.
The configuration of the prover,
including for example saturation algorithm, age-weight ratio, and term ordering scheme,
is arbitrary\todo{MS: tohle slovo je prilis silny. Veta by spis mela vysvetlit, ze jsme si neco vybrali, protoze je nam to v postate jedno. Ale zaroven si uvedomujeme, ze to nejak ovlivni vysledek a ze v ultimatnim super-toolu, by se na vsech ostatni paramatrech dalo conditionovat.} and fixed.
The prover needs to support refutational reasoning on \gls{cnf} problems
and use a term simplification ordering parameterized by a symbol precedence.
For example, the \glspl{atp} \vampire{} and E satisfy this requirement.
Note that both of these provers support the \gls{kbo} and \gls{lpo} term simplification ordering schemes,
and that each of these schemes is parameterized by a symbol precedence.

As outlined in \cref{sec:generating},
it is necessary to train a model of symbol costs.
The model is a \gls{gcn}.

\todo[inline]{Include a metagraph and an example graph of a problem.}
\todo[inline]{Compare to HINs}
% https://medium.com/@jason_trost/heterogeneous-information-networks-and-applications-to-cyber-security-23b245461adb

Since it is not obvious what the target symbol cost values should be,
the symbol cost model is not trained directly.
Instead, it is plugged into a classifier that predicts,
given a problem $P$ and a pair of precedences $\pi_0, \pi_1$,
which of the two precedences yields a better performance on $P$.

The symbol cost model is trained by plugging the model into a classifier
that is trained to predict which of a pair of precedences
yields a better performance on an arbitrary problem.

\begin{figure}[ht]
\caption{Architecture overview}
\centering
\digraph[scale=0.4]{precedencepairclassifierdetailed}{
	graph [splines=ortho];
	node [shape=renctangle, fontsize=20];
	edge [fontsize=20];
	fol [label="FOL problem", shape=oval];
	pi0 [shape=oval, label=<&pi;<SUB>0</SUB>>];
	pi1 [shape=oval, label=<&pi;<SUB>1</SUB>>];
	invpi0 [label=<&pi;<SUB>0</SUB><SUP>-1</SUP>>];
	invpi1 [label=<&pi;<SUB>1</SUB><SUP>-1</SUP>>];
	cnf [label="Clause normal form (CNF)"];
	symbolembeddings [label="Symbol embeddings"];
	symbolcosts [label="Symbol costs"];
	pi1pi0 [label="Inverse precedence difference"];
	normalized [label="Normalized inverse precedence difference"];
	paircost [label="Precedence pair cost"];
	fol -> cnf [xlabel=" Vampire "];
	cnf -> symbolembeddings [xlabel=< <B>Graph Convolution Network</B> >];
	symbolembeddings -> symbolcosts [xlabel=< <B>Feed-forward neural network</B> >, style=bold];
	symbolcosts -> paircost [style=bold];
	paircost -> loss [xlabel=" Binary cross-entropy ", style=bold];
	loss [label="Loss", shape=oval];
	pi0 -> invpi0 [xlabel=" Invert "];
	pi1 -> invpi1 [xlabel=" Invert "];
	invpi0 -> pi1pi0;
	invpi1 -> pi1pi0;
	pi1pi0 -> normalized [xlabel=" Normalize "];
	normalized -> paircost;
	symbolprecedence [label="Symbol precedence", style=dashed];
	symbolcosts -> symbolprecedence [xlabel=" Order symbols by their costs ", style=dashed];
}
\end{figure}

\subsection{Layers}
'
\begin{enumerate}
\item Problem -> symbol embeddings
\item Symbol embedding -> symbol cost
\item Symbol costs -> precedence cost
\end{enumerate}

\subsection{Cost models}

%Let $\CostSym: \symbols \rightarrow \re$ be a differentiable symbol cost model.

We define the precedence cost:
$$
\CostPrec(\pi) =
C \sum_{1 \leq i \leq n} \CostSym(\pi(i)) \cdot i =
C \sum_{1 \leq i \leq n} \CostSym(s_i) \cdot \inv{\pi}(s_i)
$$
Precedence cost is minimized by $\pi$ that orders the symbols by their costs in non-increasing order
($\forall (1 \leq i < j \leq n) . (\CostSym(\pi(i)) \geq \CostSym(\pi(j)))$).

Note that we can weight the symbols with an arbitrary nondecreasing function $f$ of symbol index:
$$
\CostPrec(\pi) =
C \sum_{1 \leq i \leq n} \CostSym(\pi(i)) \cdot f(i) =
C \sum_{1 \leq i \leq n} \CostSym(s_i) \cdot f(\inv{\pi}(s_i))
$$

We set $C = \frac{2}{n(n+1)}$ so that $\CostSym(s) = 1$ for all $s$ implies $\CostPrec(\pi) = 1$ for all $\pi$.

% Note that we use this orientation because the TensorFlow metric BinaryCrossentropy classifies 0 as negative and we use the value 0 for "failed to classify" logits.
Given a pair of precedences $\pi_0, \pi_1$,
we define the log-odds of the event "$\pi_0$ is better than $\pi_1$":
$$
\CostPrecPair(\pi_0, \pi_1) =
\CostPrec(\pi_1) - \CostPrec(\pi_0) =
C \sum_{1 \leq i \leq n} \CostSym(s_i) \cdot [\inv{\pi_1}(s_i) - \inv{\pi_0}(s_i)]
$$
Clearly $\CostPrecPair(\pi_0, \pi_1) > 0$ iff $\CostPrec(\pi_0) < \CostPrec(\pi_1)$.
For a pair of precedences about which we know that $\pi_0$ is better than $\pi_1$,
we want $\CostPrecPair(\pi_0, \pi_1) > 0$.

We model the probability of the event "$\pi_0$ is better than $\pi_1$"
by the sigmoid of $\CostPrecPair(\pi_0, \pi_1)$:
$$
p(\pi_0, \pi_1) = \sigmoid(\CostPrecPair(\pi_0, \pi_1))
$$

We use the binary cross-entropy loss to train the model.
Given a pair of precedences such that $\pi_0$ is better than $\pi_1$,
the loss is as follows:
$$
Loss(\pi_0, \pi_1) = -\log(\sigmoid(\CostPrecPair(\pi_0, \pi_1)))
$$

\section{Experimental evaluation}
\label{sec:evaluation}

% !TEX root = main.tex

To demonstrate the capacity of the trainable precedence recommender described in \cref{sec:architecture},
we performed a series of experiments.
In this section, we describe the design and configuration of the experiments,
and then compare the performance of several trained models to a number of baseline heuristics.

\subsection{Environment}

\subsubsection{System}

All the experiments were run on a computer with the following specification:

\begin{itemize}
\item CPU: Intel Xeon Gold 6140 (72 cores @ \SI{2.30}{GHz})
\item RAM: \SI{383}{GiB}
\item OS: Ubuntu 20.04
\item Kernel: GNU/Linux 5.4.0-40-generic x86\_64
\end{itemize}

% From the paper Reliable benchmarking [BenchExec]:
% – CPU model and size of RAM,
% – specified resource limits,
% – name and version of OS,
% – version of important software packages, such as the kernel or runtime environments like the Java VM,
% – version and configuration of the benchmarked tool(s), and
% – version of the benchmark set (input files)

\subsubsection{Solver}

The empirical evaluation was performed using a modified version of the \gls{atp} Vampire 4.3.0 \cite{10.1007/978-3-642-39799-8_1}.
The prover was used to generate the training data and to evaluate the trained precedence recommender.
To generate the training data,
Vampire was modified to output \gls{cnf} representations of the problems
and annotated problem signatures in a machine-readable format.
%in \gls{json} format and annotated problem signatures in \gls{csv} format.\todo{MS: Mentioning JSON or CSV seems to be too low level. ``programmer documatation'' material.}
For the evaluation of the precedences generated by the recommender,
Vampire was modified to allow the user to supply explicit predicate and function symbol precedences for the proof search
(normally, the user only picks a precedence generation heuristic).
\todo[inline]{FB: Add a link to the modified Vampire.}

\todo[inline,author=MS]{Literal comparison mode is not a standard thing. Mentioning the option name is again too low level. Explaining the effect (point two) is appropriate, although a bit too vague. To kdyztak opravim ja. 
Popsat presne setup je dulezite, kvuli reprodukovatelnosti, ale ne nutne primo v clanku (budem nejspis stejne dodavat odkaz na nejake reproducibility repo).
Ted zminit AVATAR je nevhodne, pokud nevysvetlis, co to aspon ramcove je. A pokud to budes vysvetlovat, je otazka, proc tim ztracet misto, kdyz to neni bezprostredne relevantni pro dalsi vysvetlovani.}

Vampire was run with a time limit of 10 seconds.
To increase the potential impact of predicate precedences,
we used a simple \gls{tkbo} \cite{Ludwig2007,Kovacs2011}
that compares atoms according to the predicate precedence first,
Using the ordinary \gls{kbo} as a tie-breaker.
Ordinary \gls{kbo} was used to compare terms.
\todo[inline,author=FB]{Describe the detailed configuration in an appendix and reference the appendix here.}

% All explicit options:
% vampire --encode on --statistics full --time_statistics on --proof off --avatar off --saturation_algorithm discount --age_weight_ratio 10 --literal_comparison_mode predicate --symbol_precedence frequency --time_limit 10

% Notable explicit options:
% vampire --avatar off --saturation_algorithm discount --age_weight_ratio 10 --literal_comparison_mode predicate --symbol_precedence frequency --time_limit 10

% Notable implicit options:
% --term_ordering kbo

\subsection{Training data}

The training data consists of examples of the form $(P, \PrecBetter, \PrecWorse)$,
where $P$ is a \gls{cnf} problem and $\PrecBetter, \PrecWorse$ are precedences of symbols of problem $P$
such that out of the two precedences, $\PrecBetter$ yields a proof in fewer iterations of the saturation loop (see \cref{sec:saturation}).
\todo{FB: Justify saturation loop iterations as a proxy for success. MS: Ano, ale driv nez zde!}

Since the \gls{tkbo} never compares a predicate symbol with a function symbol,
\todo{FB: Consider referencing an earlier section that explains KBO or TKBO.}
two separate precedences can be considered for each problem:
a predicate precedence and a function precedence.
We trained a predicate precedence recommender separately from a function precedence recommender
to simplify the training process and to isolate the effects of the predicate and function precedences.
This section describes how the training data for training a \emph{predicate} precedence recommender was generated.
Training data for training a function precedence recommender was generated analogously.

\subsubsection{Problem set}

\todo{MS: mozna bych dal propagoval cinny rod a minuly cas ...}

The examples are generated by sampling the \gls{fol} part of the problem library \gls{tptp} v7.4.0 \cite{10.1007/978-3-030-29436-6_29}.
A total of \num{17053} problems formulated in \gls{fof} and \gls{cnf} are sampled.
The problems formulated in \gls{fof} are converted to \gls{cnf} by Vampire.
% vampire --mode clausify
Let $\ProblemsTptp$ denote the set of all \num{17053} \gls{cnf} problems available for training and evaluation.

\subsubsection{Sampling}

The examples are generated by iterative sampling of $\ProblemsTptp$.
In each iteration, a problem $P \in \ProblemsTptp$ is chosen and Vampire is executed twice on $P$
with two (uniformly) random predicate precedences and a common random function precedence.
The ``background'' random function precedence serves as additional noise (in addition to the variability 
contained in \gls{tptp}) and makes sure that the predicate precedence recommender
will not be able to rely on any specificities that would come from fixing function precedences in the training data.

The two executions are compared in terms of performance:
predicate precedence $\PrecBetter$ is recognized as better than predicate precedence $\PrecWorse$,
denoted as $\Better{\PrecBetter}{\PrecWorse}$,
if the proof search finishes successfully with $\PrecBetter$ ($\SolverRun{P}{\Prec} = \top$)
and if the number of iterations of the saturation loop with $\PrecBetter$ is smaller than with $\PrecWorse$.
\todo{FB: Use notation for saturation loop iterations if the notation was established earlier. 
If failure is denoted by $\infty$, we can simply use $<$.}
If one of the two precedences is recognized as better,
the example $(P, \PrecBetter, \PrecWorse)$ is produced,
where $\PrecBetter$ is the better precedence,
and $\PrecWorse$ is the other precedence.
Otherwise, for example, if Vampire fails on both precedences, we go back to sampling another problem.

To ensure the efficiency of the sampling, the process is interpreted as an instance of Bernoulli multi-armed bandit problem,
\todo{Cite? MS: kdyz bude dobra citace, sem by se urcite sikla:) Pri nejhorsim Sutton \& Barto to jisti.}
with the reward of a trial being 1 in case an example is produced, and 0 otherwise.
Adaptive sampling balances
exploring problems that have been tried relatively scarcely and
exploiting problems that have yielded examples relatively often.
For each problem $P \in \ProblemsTptp$,
the generator keeps track of the number of times the problem has been tried $n_P$
and the number of examples generated from that problem $s_P$.
The ratio of $s_P$ to $n_P$ corresponds to the average reward of problem $P$ observed so far.
The problems are sampled using the allocation strategy \acrshort{ucb1} \cite{Auer2002} with a parallelizing relaxation.
In each iteration, the generator samples the problem $P$ that maximizes
$$
\frac{s_P}{n_P} + \sqrt{\frac{2 \ln n}{n_P}}
$$
where $n = \sum_{P \in \ProblemsTptp} n_P$ is the total number of tries on all problems.
The parallelizing relaxation means that the $s_P$ values are only updated once in \num{1000} iterations,
allowing up to \num{2000} parallel solver executions.
\todo{Explain in detail?}

The sampling runs until a target number of examples is generated---\num{1000000} predicate precedence examples and \num{800000} function precedence examples.
% sftp://cluster.ciirc.cvut.cz/home/bartefil/git/vampire-ml/out/20210108-generate-predicate/questions_generated
For example, while generating \num{1000000} examples for the predicate precedence dataset,
the least explored problem was tried 19 times, and the most exploited problem was tried 504 times.
\num{5349} out of the \num{17053} problems yielded at least one example.

\subsubsection{Split}

The problems in $\ProblemsTptp$ are split roughly in half to form the train set and the validation set.
Both training and validation sets are restricted to problems whose graph representation consists of at most \num{100000} nodes
to limit the memory requirements of the training.
The train set is further restricted to problems that correspond to at least one training example.
In total there are \num{7648} problems in the validation set $\ProblemsVal$
and \num{2571} problems in the train set $\ProblemsTrain$.
\todo{MS: A kdyz pouzivas validation set behem uceni, ma taky mensi podmnozinu s limitem \num{100000} nodes?}
\todo{FB: Unify the terminology: validation, or train set?
MS: Co by znamenalo unify? Nebo myslis test set?}
% /home/filip/projects/vampire-ml/vampire-ml/outputs/2021-02-11/16-27-02/results.csv
% all: 17053
% train: 8527
% val: 8526
% with_questions: 5349
% graphified (not all attempted): 10219
% train&with_questions: 2647
% train&with_questions&graphified: 2571
% val&graphified: 7648

\subsection{Model}

We used a \gls{gcn} as described in \cref{sec:architecture}
with depth 4, message size 16, ReLU activation function,
residual connections and layer normalization.
\todo[inline]{MS: zase pasivum. Tady to urcite bude chtit odkaz na sekci, kde se jasne definuje, co to znamena. Jinak recenzenti byvaji dost alergicti na hyperpametry, ktere se jen tak nahodi na nevysvetli. Minimalne nejaky komentar typu: treba, zkouseli jsme u ruzne dalsi hodnoty, ale dopadalo to podobne. Nebo, pozdeji rekeneme, co se deje, kdyz se tohle meni. Nebo: nemeli jsme cas zjistovat, co se deje, kdyz se to meni, tak berem rozumne hodnoty podle blabla.}

\subsection{Training procedure}

A predicate symbol cost model is trained by gradient backpropagation
\todo{FB: Explain? MS: tohle by melo byt k pochopeni z predchozich sekci.}
on the precedence pair classification task
\todo{FB: Reference a section. MS: Ano!}
using the examples generated from the problems in $\ProblemsTrain$.
\todo{Add references to the sections that describe the architecture.}
To avoid redundant computations, all examples generated from a problem are processed in the same training batch.
Thus, each training batch contains up to \num{128} problems and all the examples generated from these problems.
The symbol cost model is trained using the Adam optimizer \cite{Kingma2014}.
Learning rate starts at \num{1.28e-3}
and is halved each time the train loss stagnates for 10 consecutive epochs.
% tf.keras.callbacks.ReduceLROnPlateau(monitor='loss', factor=0.5, patience=10)

\subsubsection{Sample weighting}
Each of the training examples of problem $P$ contributes to the training with the weight $\frac{1}{s_P}$,
where $s_P$ is the number of examples of problem $P$ in the training set,
so that each problem contributes to the training to the same degree irrespective of the relative numbers of examples.
\todo{FB: Shall we remind that the examples are further normalized by $k(n)$ to make even across signature lengths?}
\todo{MS: asi dobry, pokud to bude jasne receny nejaky predchozi sekci.}

\subsubsection{Metrics}
To analyze the training process,
we tracked the dynamics of three metrics on $\ProblemsTrain$ and $\ProblemsVal$:
loss, accuracy and solver performance.
The solver performance was estimated once per 10 training epochs by running Vampire
on \num{1000} problems of the respective problem set
and counting the number of problems successfully solved within the time limit.
Since the accuracy is a proxy measure of solver performance and the loss is a proxy measure of the accuracy,
we are interested in generalization from loss to accuracy and from both loss and accuracy to solver performance.
Similarly, we are interested in generalization from training set to validation set.

\subsubsection{Termination}
We run the training until the validation accuracy stopped increasing for 100 consecutive epochs.

\subsection{Final evaluation}

After the training finished,
we performed a final evaluation of the most promising intermediate trained model on the whole $\ProblemsVal$.
The model that manifested the best estimated solver performance was chosen as the most promising.
\todo{Watch out: This way we may overfit on the 1000 problems used for sample solver performance evaluation.}

\subsection{Results}

A predicate precedence recommender was trained on approximately \num{500000} examples,
% Half of the 1M predicate examples was used for validation.
and a function precedence recommender was trained on approximately \num{400000} examples.
% Half of the 800k function examples was used for validation.
For each problem $P \in \ProblemsVal$,
a pair of symbol precedences was generated by the respective trained recommender
and Vampire was run using these precedences and a wall clock time limit of 10 seconds.
% Note: We have mentioned the time limit above and here we repeat it for readers who only read Results.
The results are averaged over 5 runs to reduce the effect of noise due to the wall clock time limit.
As a baseline, the performance of Vampire with the \texttt{frequency} precedence heuristic was evaluated
with the same time limit.
For comparison, the two trained recommenders are evaluated separately,
with the predicate precedence recommender using the \texttt{frequency} heuristic to generate the function precedences, and vice versa.

To generate a precedence for a problem,
the recommender first converts the problem to a machine-friendly \gls{cnf} format,
then converts the \gls{cnf} to a graph,
then predicts symbol costs using the \gls{gcn} model
and finally orders the symbols by their costs to produce the precedence.
To simplify the experiment, the time limit is only imposed on the Vampire run
and excludes the time taken by the recommender to generate the precedence.
\todo{MS: and if we don't, this would be the perfect place to summarize
how long the just described procedure took on an average problem,
or how it (roughly) scales with the graph/cnf/tptp\_file size.}
When run with 1 thread on 1000 problems,
the preprocessing took 1 second on average and at most 5 seconds,
increasing linearly with the number of nodes of the graph representation of the problem.
\todo{FB: Measure the times with joint recommender and TF restricted to 1 thread. Fix the numbers once we have a solid measurement. Add a comment with a reference to the data.}

\Cref{tab:results} shows the results.

\begin{table*}
\caption{
Results of the evaluation of various predicate precedence heuristics on $\ProblemsVal$.
Means and standard deviations over 5 runs are reported.
}
\centering
\begin{tabular}{l|ll|ll}

Model & \multicolumn{2}{l}{Successes out of \num{7648}} & Mean success rate & Baseline multiple \\
& Mean & Std & & \\

\hline


\acrshort{gcn} (predicate and function) &&&&\\
% VML-706



\acrshort{gcn} (predicate only) &
% Success rate: mean: 0.513023013, std: 0.000292887
% Evaluation results: https://ui.neptune.ai/filipbartek/vampire-ml/e/VML-553
% Total: validation_solver_eval/all/problems/measured&split&category: 7648
% Difference in success count from baseline: 154 ~ 0.020135983
% Estimated difference from baseline (estimate on 891 problems): 0.021099888
% Checkpoint: outputs/2021-02-06/14-55-41/tf_ckpts/epoch/weights.00079-0.61.tf VML-540 0.511785

\num{3923.6} &
% Success mean: validation_solver_eval/all/success/count/mean: 3923.6

\num{2.24} &
% Success std: validation_solver_eval/all/success/count/std 2.24

\SI{51.30}{\percent} &
% Success rate: 0.513023013

\num[round-mode=places,round-precision=4]{1.040853141} \\


\acrshort{gcn} (function only) &
% Final evaluation: VML-677
% Evaluated checkpoint: outputs/2021-02-16/12-28-14/tf_ckpts/epoch/weights.00289.tf
% Results file: sftp://cluster.ciirc.cvut.cz/home/bartefil/git/vampire-ml/outputs/2021-02-17/12-01-09/solver_eval/symbol_cost/epoch_-1/logs.yaml
% Total: val/all/problems/measured&split&category: 7648

\num{3874.2} &
% Success mean: val/all/success/count/mean: 3874.2

\num[round-mode=places,round-precision=2]{1.8330302779823362} &
% Success std: val/all/success/count/mean: 1.8330302779823362

\SI{50.66}{\percent} &
% Success rate: val/all/success/count/mean: 0.5065638075313807

\num[round-mode=places,round-precision=4]{1.027748302} \\


\texttt{vampire -lcm predicate} &
% Success rate: mean: 0.492887029, std: 0.000401412
% https://ui.neptune.ai/filipbartek/vampire-ml/e/VML-490
% /home/filip/projects/vampire-ml/vampire-ml/outputs/2021-02-09/12-13-43
% Row: 'val&graphified&solver_eval'
% Problems total: 7648
% Success rate: 0.492887029

\num{3769.6} &

\num{3.07} &

\SI{49.29}{\percent} &

\num[round-mode=places,round-precision=4]{1.0} \\


\texttt{vampire -lcm standard} & ? & ? & ? & ? \\

\end{tabular}
\label{tab:results}
\end{table*}
\todo{FB: Add links to Neptune runs?}
\todo{FB: Finalize results for function precedences.}
\todo{FB: Try to evaluate a recommender that uses both predicate and function models.}

\todo{FB: Add discussion.}
\todo{FB: Evaluate on problems larger than 100k nodes.}

The loss is merely a proxy measure of the accuracy,
and the accuracy is merely a proxy measure of the solver performance.
During the training, the validation loss typically started increasing after approximately 50 epochs,
while both the train loss and validation accuracy continued improving.
A natural possible explanation is that ...
\todo{Finish.}
While a higher granularity in the solver evaluation would be necessary to draw a clear conclusion,
the trend in the solver performance seems mostly aligned with the validation accuracy.

\todo[inline,author=FB]{Plot some learning curves?}


\section{Related work}

MS: my sami (PAAR paper) jsem si related workem. Budem se vuci tomu muset vymezit.
Jinak me ale nic moc bezprostredniho nenepada a nevim tedy, jestli se tedy primo
nutit do sekce s timto nazvem!

\section{Conclusion}

\subsection{Future work}

Generalization from trail acc to val acc may be improved.

Highest val accuracy does not transfer to highest ATP success rate. Generalization to ATP is sub-par.

\section*{Acknowledgments}
% > Acknowledgements should generally be placed in an unnumbered subsection at the end of the paper.

% https://sgs.cvut.cz/index.php?action=faq
% This work was supported by the Grant Agency of the Czech Technical University in Prague, grant No. SGS...

This work was supported by
the Czech Science Foundation project 20-06390Y
and
the Grant Agency of the Czech Technical University in Prague, grant\\
no.~SGS20/215/OHK3/3T/37.

% > LaTeX users should avoid self-defined environments and use the bibliographic style MathPhySci for computer science proceedings.
% > It is not possible to have hyperlinks in references.
\bibliographystyle{splncs04}
\bibliography{main}

\end{document}
