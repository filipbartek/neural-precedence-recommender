\documentclass{article}
\usepackage[utf8]{inputenc}
\usepackage[T1]{fontenc}
\usepackage[english]{babel}

% How to write a paper:
% https://mj.ucw.cz/papers/jakpsat.pdf
% Jones 2016: https://www.microsoft.com/en-us/research/academic-program/write-great-research-paper/
% AWR: https://jazyky.fel.cvut.cz/vyuka/RPP/BE9M04AKP/

% PAAR paper: https://github.com/filipbartek/learning-precedences-from-elementary-symbol-features/releases/download/paar2020%2Fceur-2/Learning_Precedences_from_Simple_Symbol_Features.pdf

% TODO: Compare to bag-of-words model.

\usepackage{amsfonts}
\usepackage{amsmath}
\usepackage{amssymb}
\usepackage{mathtools}
\usepackage{stmaryrd}
%\usepackage[disable]{todonotes}
\usepackage{todonotes}

% Multi-letter identifier
\newcommand{\mli}[1]{\mathit{#1}}

\DeclareMathOperator{\re}{\mathbb{R}}
\DeclareMathOperator{\nat}{\mathbb{N}}
\DeclareMathOperator{\logit}{logit}
\DeclareMathOperator{\sigmoid}{sigmoid}
\DeclareMathOperator*{\argmin}{argmin}
\DeclareMathOperator{\argsort}{argsort}
\newcommand{\inv}[1]{#1^{-1}}
\DeclarePairedDelimiter{\card}{\lvert}{\rvert}
\DeclarePairedDelimiter{\SquareBracket}{[}{]}
\DeclarePairedDelimiter{\Parentheses}{(}{)}
\newcommand{\DotProd}[2]{\left<#1,#2\right>}
\newcommand{\Better}[3]{#1 \prec_{#3} #2}
\newcommand{\Prob}[1]{\mathrm{Prob}(#1)}

\DeclareMathOperator{\symbols}{\Sigma}
\newcommand{\Problems}[1]{\mathcal{P}_{#1}}
\DeclareMathOperator{\cnf}{\Problems{\mathrm{CNF}}}
\DeclareMathOperator{\ProblemsTptp}{\Problems{0}}
\DeclareMathOperator{\ProblemsTrain}{\Problems{\mathrm{train}}}
\DeclareMathOperator{\ProblemsTrainEx}{\ProblemsTrain'}
\DeclareMathOperator{\ProblemsVal}{\Problems{\mathrm{val}}}
\DeclareMathOperator{\ProblemsValEx}{\ProblemsVal'}
\newcommand{\signature}[1]{\Sigma_#1}

% Precedences
\newcommand{\PrecBetter}{\pi}
\newcommand{\PrecWorse}{\rho}

\newcommand{\Vampire}{\textsc{Vampire}}

\newcommand{\loss}{\ell}
% Inspiration: The Elements of Statistical Learning, p. 120


\usepackage[pdf]{graphviz}

\begin{document}

\maketitle

\begin{abstract}
% !TEX root = main.tex

The state-of-the-art superposition-based theorem provers for \acrlong{fol}
rely on \glspl{sot} to constrain the applicability of inference rules,
which in turn shapes the ensuing search space.
The popular Knuth-Bendix simplification ordering is parameterized by 
a \emph{symbol precedence}---a permutation of the predicate and function symbols
of the input problem's signature.
Thus, the choice of a precedence has an indirect yet often substantial impact
on the amount of work required to successfully complete a proof search.

This paper describes and evaluates a symbol precedence recommender,
a \acrlong{ml} system that estimates the best possible precedence
based on observations of prover performance on a set of problems and random precedences.
Using the graph convolutional neural network technology,
the system does not presuppose the problems to be related or share a common signature. 
When coupled with the theorem prover \Vampire{} and evaluated on the \acrshort{tptp} problem library,
the recommender is found to outperform a state-of-the-art heuristic by more than \SI{4}{\percent}
on unseen problems.

\keywords{saturation-based theorem proving \and
simplification ordering \and
symbol precedence \and
\acrlong{ml} \and
\acrlong{gcn}}

% AWR: Guidelines (Unit 7, page 5):
% At most 300 words
% Content:
% Purpose of the study
% Research problems
% Basic design of the study
% Summary of interpretations and conclusions

% AWR: Parts of an abstract (checklist) (Unit 7, page 6):
% Motivation
% Problem statement
% Approach
% Results
% Conclusions

% Motivation and purpose of the study are excluded because the audience at CADE need not be reminded of the motivation for making the provers faster.

\end{abstract}

\section{Introduction}

% !TEX root = main.tex

% General motivation for FOL ATPing:
% Vampire is used as a sledgehammer in Isabelle/HOL:
% 1. https://people.mpi-inf.mpg.de/~jblanche/life.pdf
% 2. https://www.cl.cam.ac.uk/~lp15/papers/Automation/paar.pdf
% Formal methods for verification (survey): https://arxiv.org/pdf/1912.03028.pdf

% Jones 2016 guidelines:
% 1. Problem (explain by example)
% 2. Contributions (refutable, forward references to sections)

% AWR structure (Unit 7, page 1):
% 1. Attention-getter (lead-in) [1-2 sentences]
% 2. Set up for the thesis [minimum: 2-3 sentences]
% 3. Thesis statement (essay map) [1 sentence]

% AWR guidelines (Unit 7, page 5):
% Whet the reader's appetite
% Set the context
% State why the main idea is important
% State your thesis/claim

Modern saturation-based Automatic Theorem Provers (ATPs) such as E \cite{Schulz2019}, SPASS \cite{DBLP:conf/cade/WeidenbachDFKSW09},
or \Vampire{} \cite{DBLP:conf/cav/KovacsV13}
employ the superposition calculus \cite{DBLP:journals/logcom/BachmairG94,DBLP:books/el/RV01/NieuwenhuisR01} as their underlying inference system.
Integrating the flavors of resolution \cite{DBLP:books/el/RV01/BachmairG01}, paramodulation \cite{Robinson1983}, and 
the unfailing completion \cite{Bachmair89completionwithout},
superposition is a powerful calculus with
native support for equational reasoning.
The calculus is parameterized by a simplification ordering on terms % and literals,
and uses it to constrain the applicability of inferences, with a significant impact on performance.

Both main classes of simplification orderings used in practice,
the \acrlong*{kbo} \cite{Knuth1983}
and the \acrlong*{lpo} \cite{Kamin1980},
are specified with the help of a 
\emph{symbol precedence}, an ordering on the signature symbols. %.\footnote{KBO is further parameterized by symbol weights.}
% but our reference implementation in \Vampire{}~\cite{DBLP:conf/cav/KovacsV13} 
% uses for efficiency reasons only weights equal to one \cite{DBLP:conf/cade/KovacsMV11} and so we do not consider this parameter here.}
While the superposition calculus is refutationally complete for any simplification ordering \cite{DBLP:journals/logcom/BachmairG94},
the choice of the precedence has a significant impact on how long it takes to solve a given problem.

It is well known that giving the highest precedence to the predicate symbols introduced as sub-formula names 
during clausification \cite{DBLP:books/el/RV01/NonnengartW01}
% during the Tseitin transformation % of the input formula \cite{Tseitin1983} 
can immediately make the saturation produce the exponential 
set of clauses that the transformation is designed to avoid \cite{Reger2016}.
Also, certain orderings help to make the superposition a decision procedure on specific fragments of first-order logic 
(see, e.g., \cite{DBLP:conf/lics/GanzingerN99,DBLP:conf/cade/HustadtKS05}).
However, the precise way by which the choice of a precedence 
influences the follow-up proof search on a general problem is extremely hard to predict. % indirect

% neslo by rict na tvrdo, ze je to takovy ten mytus o tom, ze se uzivatel zamysli a zvoli si tu skvelou precedenci pro svuj problem?

Several general-purpose precedence generating schemes are available to ATP users,
such as the successful \texttt{invfreq} scheme in E \cite{E-manual}, which orders the symbols 
by the number of occurrences in the input problem. However, experiments with random precedences
indicate that the existing schemes often fail to come close to the optimum precedence \cite{RegerSuda2017},
suggesting room for further improvements.

In this work, we propose a \acrlong{ml} system that learns to predict for an ATP
whether one precedence will lead to a faster proof search on a given problem than another.
Given a previously unseen problem, it can then be asked to recommend the best possible precedence for an ATP to run with.
Relying only on the logical structure of the problems, % for the learning, 
the system generalizes the knowledge about favorable precedences across problems with different signatures.

Our recommender uses a relational graph convolutional neural network \cite{Schlichtkrull2017}
to represent the problem structure. It learns from the ATP performance on selected problems
and pairs of randomly sampled precedences. This information is used to train
a \emph{symbol cost model}, which then realizes the recommendation by simply sorting 
the problem's symbols according to the obtained costs. 

This work strictly improves on our previous experiments with linear regression models and simple hand-crafted symbol features \cite{DBLP:conf/cade/Bartek020}
and is, to the best of our knowledge, the first method able to propose good symbol precedences automatically 
using a non-linear transformation of the input problem structure.

The rest of this paper is organized as follows.
\Cref{sec:preliminaries} exposes the basic terminology used throughout the remaining sections.
\Cref{sec:architecture} proposes a structure of the precedence recommender that can be trained on pairs of symbol precedences,
as described in \cref{sec:training}.
\Cref{sec:evaluation} summarizes and discusses experiments performed
using an implementation of the precedence recommender.
\Cref{sec:related} compares the system proposed in this work with notable related works.
\Cref{sec:conclusion} concludes the investigation and outlines possible directions for future research.

%\todo{MS: idea maybe to work out more in the introduction: 
%We could stress how the impact or the choice of a precedence is \emph{indirect}, 
%(as it's already obvious from the explanation here)
%all the more interesting one can learn from observing just this indirect impact 
%which precendences are good and which are bad!}

% Two aspects: 
% -selection of \emph{maximals} 
% -rewriting from \emph{large to small}


\section{Preliminaries}
% Terminology

\vampire{} \cite{10.1007/978-3-642-39799-8_1}
is representative
of the state-of-the-art \gls{fol} \gls{atp} design.
\vampire{} was awarded the first place
in the \gls{fol} category of \gls{casc} \cite{},
an annual international competition of \glspl{atp},
in eleven times in the years between 2010 and 2020 \cite{}.
Moreover, 7 of 10 provers that have placed in the top 50 \% in 2020
are saturation-based, refutation-based and resolution-based,
similarly to \vampire{} \cite{}.

%\vampire{} is a saturation-based prover.
%Given a \gls{fol} problem,
%\vampire{} converts the problem into \gls{cnf}.
%and proceeds to saturate the set of clauses
%by repeated application of inference rules.

%\vampire{} searches for a proof by applying the rules
%of the superposition inference system
%\cite{10.1007/978-3-642-39799-8_1}.

% AWR candidate
Proof search in \vampire{} is crucially constrained
by a simplification ordering on terms.
Being refutation-based and saturation-based,
\vampire{} searches for contradiction
by iteratively inferring clauses provable from the input clauses.
The inferences 

explores the space of clauses provable
from the input formulas
by applying rules of superposition inference system.
This process continues until a contradiction is inferred,
the processed clauses set is saturated
or the execution hits a resource limit.

During proof search,
\vampire{} explores the space of provable clauses
by applying inference rules to clauses
that have already been proven.

% AWR candidate
Simplification ordering on terms
influences the proof search in \vampire{} on two levels.
First, the inferences on each clause are limited
to the selected literals.
In each clause,
either a negative literal or all the maximal literals are selected.
The maximality is evaluated
according to the simplification ordering.
% Note: The selection function Total does not use the simplification ordering.
Second, simplification ordering orients some of the equalities
to prevent superposition and equality factoring
from inferring redundant complex conclusions.
In each of these two roles,
the simplification ordering may impact the direction and,
in effect, the length of the proof search.

The inference system used in the state-of-the-art \gls{fol} \gls{atp} \vampire{}
is parameterized by a simplification ordering on terms.
\vampire{} uses superposition inference system \cite{}.

The \gls{atp} \vampire{} uses superposition inference system.


Vampire supports two simplification term ordering classes:
\gls{lpo} and \gls{kbo}.

The superposition inference system \cite{} used in Vampire is parameterized
by symbol precedence.

% Example: It seems that a classical example for KBO an orienting equations is group theory axioms.

The choice of symbol precedence may affect the length of a proof search to a great extent.

Given a \gls{fol} problem,
determining a symbol precedence that leads to a fast proof search is a non-trivial task.

\section{Architecture}
\label{sec:architecture}

This section describes the architecture of a precedence recommender system.
When presented with a \gls{fol} problem $P$ with symbols $\symbols$,
the system predicts a symbol precedence $\pi$ on $\symbols$
that is expected to lead to a successful proof search.

\subsection{Prediction pipeline}

\subsection{Training pipeline}

\begin{figure}[ht]
\caption{Architecture overview}
\centering
\digraph[scale=0.4]{precedencepairclassifierdetailed}{
	graph [splines=ortho];
	node [shape=renctangle, fontsize=20];
	edge [fontsize=20];
	fol [label="FOL problem", shape=oval];
	pi0 [shape=oval, label=<&pi;<SUB>0</SUB>>];
	pi1 [shape=oval, label=<&pi;<SUB>1</SUB>>];
	invpi0 [label=<&pi;<SUB>0</SUB><SUP>-1</SUP>>];
	invpi1 [label=<&pi;<SUB>1</SUB><SUP>-1</SUP>>];
	cnf [label="Clause normal form (CNF)"];
	symbolembeddings [label="Symbol embeddings"];
	symbolcosts [label="Symbol costs"];
	pi1pi0 [label="Inverse precedence difference"];
	normalized [label="Normalized inverse precedence difference"];
	paircost [label="Precedence pair cost"];
	fol -> cnf [xlabel=" Vampire "];
	cnf -> symbolembeddings [xlabel=< <B>Graph Convolution Network</B> >];
	symbolembeddings -> symbolcosts [xlabel=< <B>Feed-forward neural network</B> >, style=bold];
	symbolcosts -> paircost [style=bold];
	paircost -> loss [xlabel=" Binary cross-entropy ", style=bold];
	loss [label="Loss", shape=oval];
	pi0 -> invpi0 [xlabel=" Invert "];
	pi1 -> invpi1 [xlabel=" Invert "];
	invpi0 -> pi1pi0;
	invpi1 -> pi1pi0;
	pi1pi0 -> normalized [xlabel=" Normalize "];
	normalized -> paircost;
	symbolprecedence [label="Symbol precedence", style=dashed];
	symbolcosts -> symbolprecedence [xlabel=" Order symbols by their costs ", style=dashed];
}
\end{figure}

\subsection{Layers}
'
\begin{enumerate}
\item Problem -> symbol embeddings
\item Symbol embedding -> symbol cost
\item Symbol costs -> precedence cost
\end{enumerate}

\subsection{Cost models}

%Let $\CostSym: \symbols \rightarrow \re$ be a differentiable symbol cost model.

We define precedence cost:
$$
\CostPrec(\pi) =
C \sum_{1 \leq i \leq n} \CostSym(\pi(i)) \cdot i =
C \sum_{1 \leq i \leq n} \CostSym(s_i) \cdot \inv{\pi}(s_i)
$$
Precedence cost is minimized by $\pi$ that orders the symbols by their costs in non-increasing order
($\forall (1 \leq i < j \leq n) . (\CostSym(\pi(i)) \geq \CostSym(\pi(j)))$).

Note that we can weight the symbols with an arbitrary non-decreasing function $f$ of symbol index:
$$
\CostPrec(\pi) =
C \sum_{1 \leq i \leq n} \CostSym(\pi(i)) \cdot f(i) =
C \sum_{1 \leq i \leq n} \CostSym(s_i) \cdot f(\inv{\pi}(s_i))
$$

We set $C = \frac{2}{n(n+1)}$ so that $\CostSym(s) = 1$ for all $s$ implies $\CostPrec(\pi) = 1$ for all $\pi$.

% Note that we use this orientation because the TensorFlow metric BinaryCrossentropy classifies 0 as negative and we use the value 0 for "failed to classify" logits.
Given a pair of precedences $\pi_0, \pi_1$,
we define the log-odds of the event "$\pi_0$ is better than $\pi_1$":
$$
\CostPrecPair(\pi_0, \pi_1) =
\CostPrec(\pi_1) - \CostPrec(\pi_0) =
C \sum_{1 \leq i \leq n} \CostSym(s_i) \cdot [\inv{\pi_1}(s_i) - \inv{\pi_0}(s_i)]
$$
Clearly $\CostPrecPair(\pi_0, \pi_1) > 0$ iff $\CostPrec(\pi_0) < \CostPrec(\pi_1)$.
For a pair of precedences about which we know that $\pi_0$ is better than $\pi_1$,
we want $\CostPrecPair(\pi_0, \pi_1) > 0$.

We model the probability of the event "$\pi_0$ is better than $\pi_1$"
by the sigmoid of $\CostPrecPair(\pi_0, \pi_1)$:https://www.overleaf.com/project/5f75a50ba1a0930001ce1162
$$
p(\pi_0, \pi_1) = \sigmoid(\CostPrecPair(\pi_0, \pi_1))
$$

We use the binary cross-entropy loss to train the model.
Given a pair of precedences such that $\pi_0$ is better than $\pi_1$,
the loss is as follows:
$$
Loss(\pi_0, \pi_1) = -\log(\sigmoid(\CostPrecPair(\pi_0, \pi_1)))
$$

\section{Evaluation}
\label{sec:evaluation}

\section{Related work}

\section{Conclusion}

\bibliography{main}

\end{document}
