% General motivation for FOL ATPing:
% Vampire is used as a sledgehammer in Isabelle/HOL:
% 1. https://people.mpi-inf.mpg.de/~jblanche/life.pdf
% 2. https://www.cl.cam.ac.uk/~lp15/papers/Automation/paar.pdf
% Formal methods for verification (survey): https://arxiv.org/pdf/1912.03028.pdf

\todo[author=Filip,inline]{Consider an alternative approach: Start with an example problem, rewrite in two ways, describe how do state-of-the-art provers differentiate between these two ways (by symbol precedence), ...}

% Jones 2016 guidelines:
% 1. Problem (explain by example)
% 2. Contributions (refutable, forward references to sections)

% AWR structure (Unit 7, page 1):
% 1. Attention-getter (lead-in) [1-2 sentences]
% 2. Set up for the thesis [minimum: 2-3 sentences]
% 3. Thesis statement (essay map) [1 sentence]

% AWR guidelines (Unit 7, page 5):
% Whet the reader's appetite
% Set the context
% State why the main idea is important
% State your thesis/claim

\todo[inline,author=FB]{Try to use the first page for something more flashy than exposure of saturation-based automated proving. Ideas: catchy puzzle-like example, abstract description of the crucial challenge etc.}

Choosing the next inference to apply is the most important decision point in the state-of-the-art
\todo{FB: Add "saturation-based"?}
\glspl{atp} for \gls{fol}.
Since the most popular \glspl{atp} use the superposition calculus with literal selection,
the \gls{sot}, such as the \gls{kbo}, plays a key role in guiding the proof search.
The \gls{kbo} is typically specified by a symbol precedence,
that is a permutation of the non-logical symbols of the input problem.
Each of the commonly used precedence heuristics orders the symbols
lexicographically according to a small number of simple symbol properties,
such as the number of occurrences in the input problem,
% vampire --symbol_precedence frequency
the arity of the symbol, or
% vampire --symbol_precedence arity
%occurrence in the conjecture,
% vampire --symbol_precedence_boost goal
%number of occurrences in unit clauses, or
% vampire --symbol_precedence_boost units
whether or not the symbol was introduced by the Tseytin transformation or skolemization.
% vampire --introduced_symbol_precedence top
Since none of the simple heuristics fits all the problems,
choosing a symbol precedence that guides the proof search on a problem efficiently towards the refutation
remains to be an intriguing task.
This paper explores the possibility of solving this task by training a neural networks
on the results of running the \gls{atp} Vampire \cite{10.1007/978-3-642-39799-8_1} with randomly sampled precedences.

A \gls{fol} problem consists of \gls{fol} formulas divided into two categories:
one or more premises and one conjecture.
% We assume that \gls{fol} is expanded to FOL.
% For this reason the article is "an".
When presented with a \gls{fol} problem,
a \definiendum{saturation-based} \gls{atp} negates the conjecture (as a first step of proof by contradiction)
and converts all the formulas into an equisatisfiable set of clauses in \gls{cnf}.
\todo[author=Filip]{Are there any saturation-based ATPs that don't use CNF?}
The prover proceeds to iteratively infer new provable clauses
until a contradiction, represented by the empty clause, is inferred,
or until no new clauses can be inferred.
A proof of the fact that the premises and the negated conjecture entail a contradiction
constitutes a proof that the premises entail the conjecture.
\todo[author=Filip]{Cite some source that explains refutation-based automated theorem proving. Perhaps Harris?}
\todo[author=Martin]{prvni odstavec uvodu bych pro CADE publikum vynechal a zacal rovnou s druhym}

The state-of-the-art \gls{fol} \glspl{atp},
such as \vampire{} \cite{10.1007/978-3-642-39799-8_1} and E \cite{10.1007/978-3-030-29436-6_29},
are saturation-based and
infer new provable clauses by applying the rules of the \definiendum{superposition calculus}.
\todo[author=Filip]{Cite some source that explains superposition calculus.}
The inferences under superposition calculus are constrained by a \definiendum{simplification ordering on the terms}.
\todo[author=Filip]{Do we mean terms in sensu stricto (applications of functions), or generalized terms (applications of functions and predicates)?}
The two classes of simplification orderings commonly used in practice are
\gls{lpo} \cite{Kamin1980} and \gls{kbo} \cite{Knuth1983}.
In both classes the ordering is specified by a \definiendum{symbol precedence},
\todo[author=Filip]{This may be an oversimplification: KBO is further parameterized by symbol weights.}
which is a permutation of predicate and function symbols of the input problem.

The typical approach to constructing a symbol precedence orders the symbols by a property
that is relatively cheap and straightforward to compute,
such as the frequency (the number of occurrences in the input problem) \cite{E-manual} or the arity.
\todo[author=Filip]{Cite the original source of the arity heuristic.}
The structure of the problem contains much information that is relevant for the choice of symbol precedence,
but is completely ignored by the symbol properties that inform the symbol precedences.
\todo[author=Filip]{Why do we believe this?}
\todo[author=Martin]{Klicovou vetu ... by asi chtelo rozvest treba do celeho odstavce. Takhle vyzni dost krypticky a neni moc jasne, co se snazi vyjadrit: "The structure of the problem" neni predem vysvetleno. "completely ignored" - "ignored" trosku zavani nejakym agentem, pritom je zde predmetem neziva vec / abstraktum}

This paper describes the architecture (\cref{sec:architecture}) and an empirical evaluation (\cref{sec:evaluation})
of a symbol precedence heuristic based on a \gls{gcn} \cite{kipf2017semisupervised}
and trained on proof attempts with random symbol precedences.
The experiments presented in \cref{sec:evaluation} demonstrate an improvement of success rate by 20 \%
compared to the state-of-the-art heuristic.
\todo[author=Martin]{Poslednimyu ostaveci by taky asi jeste mohlo neco predhazet, aby to mohl shrnout: "the architecture" ma urcity clen, ale predtim zadna "architecture" jeste neni. "graph convolution network" taky vlastne zminujes poprve}

% Filip: I decided not to follow the AWR structure in favor of Martin and CADE.

\iffalse % AWR draft

The performance of the state-of-the-art \acrfullpl{atp} is severely limited by some of the design decisions that constitute the provers' architecture.
For example, the \frequency{} symbol precedence heuristic of the \acrfull{fol} \acrshort{atp} \vampire{} is easily outperformed by the best of 10 random precedences.\cite{}
The gap in performance suggests that a more elaborate precedence heuristic with a superior performance exists.
% what is precedence heuristic?
% Thesis statement
Training a model from proof attempts with random precedences yields a heuristic that significantly outperforms the state-of-the-art heuristic.

\fi
