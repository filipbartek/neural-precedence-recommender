% !TEX root = main.tex

State-of-the-art superposition-based theorem provers for \acrlong{fol}
rely on \glspl{sot} to constrain the applicability of inference rules,
which in turn shapes the ensuing search space.
The popular Knuth-Bendix simplification ordering is parameterized by 
a \emph{symbol precedence}---a permutation of the predicate and function symbols
of the input problem's signature.
Thus, the choice of a precedence has an indirect yet often substantial impact
on the amount of work required to successfully complete a proof search.

This paper describes and evaluates a symbol precedence recommender,
a \acrlong{ml} system that estimates the best possible precedence
based on observations of prover performance on a set of problems and random precedences.
Using the graph convolutional neural network technology,
the system does not presuppose the problems to be related or share a common signature. 
When coupled with the theorem prover \Vampire{} and evaluated on the \acrshort{tptp} problem library,
the recommender is found to outperform a state-of-the-art heuristic by more than \SI{4}{\percent}
on unseen problems.

\keywords{saturation-based theorem proving \and
simplification ordering \and
symbol precedence \and
\acrlong{ml} \and
\acrlong{gcn}}

% AWR: Guidelines (Unit 7, page 5):
% At most 300 words
% Content:
% Purpose of the study
% Research problems
% Basic design of the study
% Summary of interpretations and conclusions

% AWR: Parts of an abstract (checklist) (Unit 7, page 6):
% Motivation
% Problem statement
% Approach
% Results
% Conclusions

% Motivation and purpose of the study are excluded because the audience at CADE need not be reminded of the motivation for making the provers faster.
