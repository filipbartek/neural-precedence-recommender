% !TEX root = main.tex

% AWR: Guidelines (Unit 7, page 5):
% At most 300 words
% Content:
% Purpose of the study
% Research problems
% Basic design of the study
% Summary of interpretations and conclusions

% AWR: Parts of an abstract (checklist) (Unit 7, page 6):
% Motivation
% Problem statement
% Approach
% Results
% Conclusions


This paper describes and evaluates a symbol precedence heuristic based on a \acrlong{gcn}.
\todo{MS: I think we are more than just what one typically understands as a heuristic.
Because we learn what to recommend from the data.}
\todo{MS: From that perspectively, the learning itself seems more important than the particular use of a GCN.}
When trained on proof attempts with random symbol precedences,
the heuristic outperforms the state of the art by more than \SI{2}{\percent}.
For the purpose of the evaluation, 
\todo{MS: ``For the purpose of the evaluation'' je matouci. 
Ta heuristika prece nema vubec smysl ve svete bez ``symbol precedences for Knuth-Bendix term simplification orderings''.
Nebo tim chces naznacit, ze by pripadne sla pouzit obecneji? To by se ale muselo asi rict explicitne 
a nebude to lehke vysvetlit strucne.}
the symbol precedences predicted by the heuristic
are used to instantiate Knuth-Bendix term simplification orderings.
These orderings in turn guide the proof search in the state-of-the-art \acrlong{atp} \vampire{}
by constraining the superposition calculus inferences.
\todo{MS: uz i ty ``random symbol precedences'', kterymi to vypraveni nahore zacina potrebuji vampira.
Mozna bude lepsi, trochu nudne, zacit tim, odkud se berou a na co se pouzivaji precedence
a teprve pak rict, ze se z nich ucime a pak umime navrhovat nove, lepsi?}
The evaluation is performed on \acrlong{fol} problems from the \acrshort{tptp} problem library.

\keywords{saturation-based theorem proving \and
term simplification ordering \and
symbol precedence \and
\acrlong{ml} \and
\acrlong{gcn}}

% Motivation and purpose of the study are excluded because the audience at CADE need not be reminded of the motivation for making the provers faster.
