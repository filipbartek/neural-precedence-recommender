% !TEX root = main.tex

% AWR: Guidelines (Unit 7, page 5):
% At most 300 words
% Content:
% Purpose of the study
% Research problems
% Basic design of the study
% Summary of interpretations and conclusions

% AWR: Parts of an abstract (checklist) (Unit 7, page 6):
% Motivation
% Problem statement
% Approach
% Results
% Conclusions

The state-of-the-art \acrlongpl{atp} for \acrlong{fol}, such as Vampire,
use \glspl{sot} to constrain the proof search.
The popular Knuth-Bendix simplification ordering is parameterized by the \emph{symbol precedence}
\todo{FB: Is it ok to emphasize a word in abstract?}
-- a permutation of the predicate and function symbols occurring in the input problem.
The choice of the symbol precedence can have a substantial impact on the length of the proof search.

This paper describes and evaluates a symbol precedence recommender based on a \acrlong{gcn}.
When trained on proof attempts that use random symbol precedences,
the recommender outperforms the state-of-the-art heuristic by more than \SI{2}{\percent}.
The recommender is evaluated on \acrlong{fol} problems from the \acrshort{tptp} problem library.
\todo{FB: Should I mention Vampire here?}
\todo{FB: Try to mention that the recommender is signature-agnostic.}

\iffalse
This paper describes and evaluates a symbol precedence recommender based on a \acrlong{gcn}.
When trained on proof attempts that use random symbol precedences,
the recommender outperforms the state of the art heuristic by more than \SI{2}{\percent}.
The symbol precedences predicted by the recommender
specify Knuth-Bendix \glspl{sot}.
These orderings in turn guide the proof search in the state-of-the-art \acrlong{atp} Vampire
by constraining the inferences in the superposition calculus.
\todo{MS: uz i ty ``random symbol precedences'', kterymi to vypraveni nahore zacina potrebuji vampira.
Mozna bude lepsi, trochu nudne, zacit tim, odkud se berou a na co se pouzivaji precedence
a teprve pak rict, ze se z nich ucime a pak umime navrhovat nove, lepsi?}
The recommender is evaluated on \acrlong{fol} problems from the \acrshort{tptp} problem library.
\fi

\keywords{saturation-based theorem proving \and
\gls{sot} \and
symbol precedence \and
\acrlong{ml} \and
\acrlong{gcn}}

% Motivation and purpose of the study are excluded because the audience at CADE need not be reminded of the motivation for making the provers faster.
